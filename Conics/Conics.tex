%Document setup
\documentclass[10pt,a4paper]{amsart}
\usepackage[latin1]{inputenc}
\usepackage{mathtools}
\usepackage{mathrsfs} %gives mathscr font
\usepackage{graphicx}
%\usepackage[headsep=0.15in, left=0.5in, right=0.5in, top=0.6in, bottom=0.5in]{geometry}
\usepackage[textlf,mathlf]{MinionPro}
\usepackage{fancyhdr}
%\usepackage{xypic}
\usepackage{tikz-cd}
\usepackage{todonotes}

%Environments
\numberwithin{equation}{section}
\numberwithin{figure}{section}
\theoremstyle{definition}
\newtheorem{example}{\protect\examplename}[section]
\theoremstyle{remark}
\newtheorem*{rem*}{\protect\remarkname}
\theoremstyle{plain}
\newtheorem{thm}{\protect\theoremname}[section]
\theoremstyle{plain}
\newtheorem{cor}{\protect\corollaryname}[section]
\theoremstyle{definition}
\newtheorem{defn}{\protect\definitionname}[section]
\theoremstyle{plain}
\newtheorem{prop}{\protect\propositionname}[section]
\theoremstyle{plain}
\newtheorem{lem}{\protect\lemmaname}[section]

\providecommand{\definitionname}{Definition}
\providecommand{\examplename}{Example}
\providecommand{\lemmaname}{Lemma}
\providecommand{\propositionname}{Proposition}
\providecommand{\remarkname}{Remark}
\providecommand{\corollaryname}{Corollary}
\providecommand{\theoremname}{Theorem}

%macros, etc.
\newcommand{\legendre}[2]{\genfrac{(}{)}{}{}{#1}{#2}}
\renewcommand{\arraystretch}{1.2}

\begin{document}
\begin{rem*}
	Right now this is mostly a summary. I haven't padded it with much exposition.
\end{rem*}
\begin{rem*}
	\textcolor{red}{I think in the problem statement we want to require that $\gcd(a_1,a_2,a_3) = 1$, a stronger statement than $\gcd(a_1,\ldots,a_6)=1$, to guarantee that the ``conic'' is always actually a conic? See Example \ref{exa:all-possibilities}}
\end{rem*}
\section{The case $p = 2$}
We consider the even prime separately because the diagonalization process employed in the other cases does not work in characteristic 2.

Fortunately there are very few conics over $\mathbb{F}_2$, and the solutions to
\[
	ax^2 + bxy + cy^2 + dx + ey + f = 0
\]
are the same as the solutions to
\[
	bxy + (a+d)x + (c+e)y + f = 0.
\]
At this point, one can simply list out all the possibilities. It can be checked that the number of solutions in each case can be concisely written as $2 + b(-1)^{(a+d)(c+e) + f}$, where $b$ is either 0 or 1.

\section{The case $p \geq 3$}
Starting with the conic
\begin{equation}\label{eq:original-conic}
	ax^2 + bxy + cy^2 + dx + ey + f = 0
\end{equation}
we projectivize it to obtain the following homogeneous equation in $\mathbb{P}^2$:
\begin{equation}\label{eq:projective-conic}
	aX^2 + bXY + cY^2 + dXZ + eYZ + fZ^2 = 0.
\end{equation}
In $\{Z=0\} = \mathbb{P}^1 \subset \mathbb{P}^2$, this equation is
\begin{equation}\label{eq:projective-conic-at-infinity}
	aX^2 + bXY + cY^2 = 0.
\end{equation}

\begin{lem}
	The number of solutions to \eqref{eq:projective-conic} in $\mathbb{P}^2$ is the sum of the numbers of solutions to \eqref{eq:original-conic} in $\mathbb{A}^2$ and \eqref{eq:projective-conic-at-infinity} in $\mathbb{P}^1$.
	
	%(the below seemed a bit needlessly contrived)
	%to these three equations are related as follows:
	%\[
	%	|\{f = 0\}| = |\{F = 0\}| - |\{\bar{F} = 0\}|
	%\]	
	%where $|-|$ denotes cardinality. Note that $\{f = 0\} \subset \mathbb{A}^2$, $\{F=0\} \subset \mathbb{P}^2$, and %$\{\bar{F}=0\} \subset \mathbb{P}^1$.
\end{lem}
\begin{proof}
	A solution $[X:Y:Z]$ of \eqref{eq:projective-conic} corresponds to a solution $(X/Z,Y/Z)$ of \eqref{eq:original-conic} if $Z \neq 0$ and to a solution $[X:Y]$ of \eqref{eq:projective-conic-at-infinity} if $Z=0$.
\end{proof}

\subsection{Diagonalization of quadratic forms}
\color{blue}
% I just copied over what Oron wrote in chat; haven't done anything with it yet
\[
	aX^2 + \frac{4ac - b^2}{4a}Y^2 + \frac{4acf - ae^2 - 
	b^2 f + bde - cd^2}{4ac - b^2}Z^2 = 0
\]
Transformation matrix:
\[
	\begin{bmatrix}
		1 & 0 & 0\\
		-\dfrac{b}{2a} & 1 & 0\\
		\dfrac{be - 2cd}{4ac - b^2} & \dfrac{-2ae + bd}{4ac - b^2} & 1
	\end{bmatrix}
\]
\color{black}

In the following, let $r$ denote a quadratic nonresidue in $\mathbb{F}_p$.
\begin{lem}\label{lem:quadrics-P1}
	By a change of coordinates, the equation $aX^2 + bXY + cY^2 = 0$ can be transformed into one of the following:
	\begin{enumerate}
		\item $X^2 = 0$,
		\item $X^2 + Y^2 = 0$, \label{case:x^2+y^2=0}
		\item $X^2 + rY^2 = 0$.
	\end{enumerate}
\end{lem}

The first of these has one solution, namely $[0:1]$.

Evidently $Y$ must be nonzero for the other two cases, so we can rewrite them as $(X/Y)^2 = -1$ and $(X/Y)^2 = -r$ respectively. If $-1$ is a quadratic residue, then $-r$ is not. It follows that $X^2 + Y^2 = 0$ has two solutions and $X^2 + rY^2 = 0$ has none. If $-1$ is a nonresidue, the situation is reversed: $X^2 + Y^2 = 0$ has no solutions and $X^2 + rY^2 = 0$ has two solutions.

\begin{lem}
	Every conic in $\mathbb{P}^2$ can be brought to one of the following forms:
	\begin{enumerate}
		\item $X^2 = 0$,
		\item $X^2 + Y^2 = 0$,
		\item $X^2 + rY^2 = 0$,
		\item $X^2 + Y^2 + Z^2 = 0$,
		\item $X^2 + Y^2 + rZ^2 = 0$.
	\end{enumerate}
\end{lem}

Now we count the number of solutions in each of these cases.
\subsubsection*{Rank 1 (first case)}
The equation $X^2 = 0$ has $p + 1$ solutions.

\subsubsection*{Rank 2 (second and third cases)}
The only solution with $Y=0$ is $[0:0:1]$. So assume $Y \neq 0$, and rewrite the equations as $(X/Y)^2 = -1$ and $(X/Y)^2 = -r$ respectively.

Suppose $-1$ is a quadratic residue mod $p$. The equation $(X/Y)^2 = -1$ gives two possibilities for $X/Y$, and thus $2p$ solutions in $\mathbb{P}^2$. Hence $X^2 + Y^2 = 0$ has $2p + 1$ solutions in total. On the other hand, $(X/Y)^2 = -r$ has no solutions, meaning $[0:0:1]$ is the only solution to $X^2 + rY^2 = 0$.

If $-1$ is a nonresidue, the situation is reversed, and $X^2 + Y^2 = 0$ has one solution while $X^2 + rY^2 = 0$ has $2p + 1$ solutions.

\subsubsection*{Rank 3 (fourth and fifth cases)}
Suppose $-1$ is a quadratic residue mod $p$. If $Z = 0$, then we are in the setting of Lemma \ref{lem:quadrics-P1}, case (\ref{case:x^2+y^2=0}), which has two solutions. Otherwise, we can divide by $Z$ and obtain
\begin{align*}
	(X/Z)^2 + (Y/Z)^2 &= -1\\
	(X/Z)^2 + (Y/Z)^2 &= -r
\end{align*}
respectively. By Proposition \ref{prop:circle-solutions}, these each have $p - 1$ solutions. Therefore $X^2 + Y^2 + Z^2 = 0$ and $X^2 + Y^2 + rZ^2 = 0$ each have $p+1$ solutions.

Suppose $-1$ is a nonresidue. There are no solutions with $Z=0$, so rewrite the equations as above. By Proposition \ref{prop:circle-solutions} again, they each have $p + 1$ solutions.

\begin{example}[A zoo of possibilities]\label{exa:all-possibilities}
	
	In the table below, the first two columns reflect the preceding discussion. The remaining columns consider how many solutions lie in $\{Z = 0\} = \mathbb{P}^1 \subset \mathbb{P}^2$, which do not give solutions to the affine equation. The entries marked ``N/A'' are impossible---explanation is given afterwards. For all other entries, examples are given for the specific case $p=3$.
	\[\begin{tabular}{|c|c||c|c|c|}
		\hline
		\multicolumn{2}{|c||}{} & \multicolumn{3}{c|}{\# soln. in $\{Z=0\}$}  \\
		\hline
		rank of proj. & \# soln. in $\mathbb{P}^2$ & 0 & 1 & 2 \\ 
		\hline 
		\hline
		1 & $p+1$ & N/A & $x^2$ & N/A \\ 
		\hline 
		2 & 1 & $x^2 + y^2$ & $x^2 + 1$ & N/A \\ 
		\hline 
		2 & $2p+1$ & N/A & $x^2 + x$ & $xy$ \\ 
		\hline 
		3 & $p+1$ & $x^2 + y^2 + 1$ & $x+y^2$ & $x^2 + 2y^2 + 1$ \\ 
		\hline 
	\end{tabular} \]
	\begin{itemize}
		\item When the projectivization has rank 1, it is a double line. This line is distinct from $\{Z = 0\}$ \todo{otherwise our original conic was a constant, but maybe this can happen if we're not careful about the problem setup... ask} and thus meets it at exactly one point.
		\item When the projectivization has rank 2 and has $2p + 1$ points, it is the union of two intersecting lines, neither of which is $\{Z = 0\}$\todo{Likewise... what if the original ``conic'' was just $x$}. If these two lines intersect in $\{Z = 0\}$, then there is just one solution in $\{Z = 0\}$. Otherwise, there are two. This corresponds geometrically to parallel lines versus intersecting lines in $\mathbb{A}^2$.
	\end{itemize}
\end{example}

\appendix
\section{Number theory}
The nonzero elements $\mathbb{F}^\times_p \subset \mathbb{F}_p$ form a group under multiplication, and the squaring map
\[
	\mathbb{F}^\times_p \xrightarrow{x \mapsto x^2} \mathbb{F}^\times_p
\]
is a homomorphism. If $p$ divides $x^2 - 1 = (x+1)(x-1)$, we must have $x = \pm 1$ mod $p$, thus the kernel is $\{\pm 1\}$. Let $(\mathbb{F}^\times_p)^2$ be the image of the homomorphism. Then,
\[
	\mathbb{F}^\times_p / (\mathbb{F}^\times_p)^2 \cong \mathbb{Z}/2.
\]
This implies that the product of two nonresidues or two residues is a residue, while the product of a (nonzero) residue and nonresidue is a nonresidue. Moreover, each nonzero residue has exactly two square-roots.
\begin{prop}\label{prop:circle-solutions}
	For $a \neq 0$, the equation $x^2 + y^2 = a$ has $p - \legendre{-1}{p}$ solutions over $\mathbb{F}_p$.
\end{prop}
\begin{proof}
	\color{blue}will write soon\color{black}
\end{proof}
The proof of the following fact is omitted.
\begin{prop}
	For an odd prime $p$, the Legendre symbol $\legendre{-1}{p}$ is given by
	\[
		\legendre{-1}{p} = \left\{
			\begin{matrix}
				1 & \text{if } p \cong 1 \pmod 4 \\
				-1 & \text{if } p \cong 3 \pmod 4.
			\end{matrix}
		\right.
	\]
\end{prop}
\end{document}