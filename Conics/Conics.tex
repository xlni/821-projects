%Document setup
\documentclass[10pt,a4paper]{amsart} 

\usepackage[latin1]{inputenc}
\usepackage{mathtools} 
\usepackage{amssymb} 
\usepackage{mathrsfs} % gives mathscr font 
\usepackage{graphicx}
%\usepackage[headsep=0.15in, left=0.5in, right=0.5in, top=0.6in, bottom=0.5in]{geometry} 
%\usepackage[textlf,mathlf]{MinionPro} 
\usepackage{fancyhdr}
%\usepackage{xypic}
\usepackage{tikz-cd} 
\usepackage{todonotes}
\usepackage{verbatim}
\usepackage{filecontents}
\usepackage[colorlinks=true]{hyperref}
\usepackage{tabu}

%Environments
\numberwithin{equation}{section} 
\numberwithin{figure}{section}
\theoremstyle{definition} 
\newtheorem{example}{\protect\examplename}[section]
\theoremstyle{remark} 
\newtheorem*{rem*}{\protect\remarkname}
\theoremstyle{plain} 
\newtheorem{thm}{\protect\theoremname}[section]
\theoremstyle{plain} 
\newtheorem{cor}{\protect\corollaryname}[section]
\theoremstyle{definition} 
\newtheorem{defn}{\protect\definitionname}[section]
\theoremstyle{plain} 
\newtheorem{prop}{\protect\propositionname}[section]
\theoremstyle{plain} 
\newtheorem{lem}{\protect\lemmaname}[section]

\providecommand{\definitionname}{Definition}
\providecommand{\examplename}{Example} 
\providecommand{\lemmaname}{Lemma}
\providecommand{\propositionname}{Proposition}
\providecommand{\remarkname}{Remark} 
\providecommand{\corollaryname}{Corollary}
\providecommand{\theoremname}{Theorem}

%macros, etc.
\newcommand{\legendre}[2]{\genfrac{(}{)}{}{}{#1}{#2}}
\renewcommand{\arraystretch}{1.2} 

\newcommand{\A}{\mathbb{A}}
\renewcommand{\P}{\mathbb{P}} 
\newcommand{\F}{\mathbb{F}}
\newcommand{\Z}{\mathbb{Z}}
\newcommand{\Q}{\mathbb{Q}}
\newcommand{\R}{\mathbb{R}}
\newcommand{\GL}{\operatorname{GL}}
\newcommand{\Span}{\operatorname{Span}}

\begin{document} 

    \title{Conics (title tbd)} 

    \maketitle

    \begin{abstract} 
        We count the number of solutions of any affine plane conic modulo a prime.
        We use this result to determine whether such a conic has rational solutions. 
    \end{abstract}

    \tableofcontents

    \section{Introduction} 
    
    
    
    \section{Projectivization of the problem}

    Let us  begin by fixing a conic
    \begin{equation}\label{eq:original-conic} 
        C \colon ax^2 + bxy + cy^2 + dx + ey + f = 0 
    \end{equation} 
    and a choice of ground field $\F_p$. We further require that not all of 
    $a$, $b$, and $c$ be zero, and that the coefficients have gcd $1$.
    \textcolor{red}{Might be worth explaining why we've chosen to impose these conditions.}
    Unless otherwise specified, all primes are odd.
    (We defer treatment of the case $p = 2$ to \S\ref{sec:affine-solutions}.)
    \begin{defn}
        Let $\A^n_{\F_p}$ denote $n$-dimensional affine space over $\F_p$. Define 
        $n$-dimensional projective space over $\F_p$ as the quotient
        \[ \P^n_{\F_p} = (\A^{n+1}_{\F_p}\setminus \{0\})/{\sim}, \]
        where we declare that $(x_1,\ldots,x_{n+1}) \sim 
        (\lambda x_1,\ldots,\lambda x_{n+1})$ for $\lambda$ a nonzero scalar. We
        will primarially work with $n = 2$, and suppress the ground field from the 
        notation. We will write $[X:Y:Z]\in \P^2$ to denote the equivalence 
        class of $(X,Y,Z)\in \A^3 \setminus \{0\}$.
    \end{defn}

    With this construction in mind, we can \emph{projectivize} our conic
    \eqref{eq:original-conic} from $\A^2$ to $\P^2$, by writing it as a 
    homogenous equation: 
    \begin{equation}\label{eq:projective-conic} 
        \widetilde{C} \colon aX^2 + bXY + cY^2 + dXZ + eYZ + fZ^2 = 0. 
    \end{equation}
    Note that homogeneity ensures it is well-defined to ask when this
    equation has a solution in $\P^2$.

    Let $C(\F_p)$ denote the solution set of \eqref{eq:original-conic} in $\A^2$. Our aim
    is to characterize $|C(\F_p)|$ as a function of the coefficients $a,b,c,d,e,f$ and 
    the prime $p$. The solutions of \eqref{eq:projective-conic} define a set
    $\widetilde{C}(\F_p) \subset \P^2$.

    There is an evident injection $C(\F_p) \rightarrowtail \widetilde{C}(\F_p)$ sending $(x,y)$ to
    $[x:y:1]$. However, elements of $\widetilde{C}(\F_p)$ of the form $[X:Y:0]$ do not 
    correspond to elements of $C(\F_p)$. 
    These are solutions ``at infinity'' in the projective plane.
    Thankfully, these extraneous solutions are easily characterized: they are the
    solutions to the equation
    \begin{equation}\label{eq:projective-conic-at-infinity} 
        \widetilde{C}_0 \colon aX^2 + bXY + cY^2 = 0
    \end{equation} 
    on the projective line $\P^1$. Call this solution set $\widetilde{C}_0(\F_p)\subset \P^1$.  

    \begin{lem} 
        $|\widetilde{C}(\F_p)| = |C(\F_p)| + |\widetilde{C}_0(\F_p)|$.
    \end{lem} 
    \begin{proof}
        A solution $[X:Y:Z]$ of $\widetilde{C}$ corresponds 
        to a solution $(X/Z,Y/Z)$ of $C$ when $Z \neq 0$,
        and to a solution $[X:Y]$ of $\widetilde{C}_0$ 
        when $Z = 0$.  
    \end{proof}

    \section{Diagonalization of quadratic forms}
    \label{sec:diagonalizing-quadratic-forms} 

    In this section, we aim to simplify the equation defining the homogenized conic $\widetilde{C}$ via coordinate-changes of $\P^2$. Regard this equation as a quadratic form in three variables over $\F_p$ (where $p\ne 2$), which may be represented by the symmetric matrix
\begin{equation*}
A=\begin{bmatrix}
a&b/2&d/2\\
b/2&c&e/2\\
d/2&e/2&f
\end{bmatrix}.
\end{equation*}
Indeed, letting $v=[\begin{matrix}X&Y&Z\end{matrix}]^T$, the equation defining $\widetilde{C}$ is given by $v^TAv$. Now, any $P\in\GL_3(\F_p)$ gives rise to a change of coordinates $v\mapsto Pv$ in $\P^2$. After performing this change of coordinates, the quadratic form defining $\widetilde{C}$ is given by
\begin{equation*}
(Pv)^TA(Pv)=v^T(P^TAP)v,
\end{equation*}
hence is represented by the matrix $P^TAP$. Since $P$ is invertible, the projective conic corresponding to the quadratic form given by $P^TAP$ has the same number of solutions over $\F_p$ as $\widetilde{C}$. The following result is well-known (see for instance \cite[Prop.~42:1]{omeara}):
\begin{thm}
\label{thm:diag}
There exists some $P\in\GL_3(\F_p)$ such that $P^TAP$ is diagonal.
\end{thm}
This significantly simplifies the set of conics which we must consider: a diagonal matrix
\begin{equation*}
\begin{bmatrix}
\lambda_1&0&0\\
0&\lambda_2&0\\
0&0&\lambda_3
\end{bmatrix}
\end{equation*}
corresponds to the conic defined by $\lambda_1x^2+\lambda_2y^2+\lambda_3z^2=0$ (note that we cannot have $\lambda_1=\lambda_2=\lambda_3=0$ by our initial assumptions). By further scaling and permuting coordinates as appropriate, we immediately obtain the following:
\begin{cor}
\label{cor:sixcases}
Fix a quadratic nonresidue $r\in\F_p$. There exists some $P\in\GL_3(\F_p)$ such that $P^TAP$ is equal to one of the following:
$$
\begin{tabu}{cccccc}
(1)&(2)&(3)&(4)&(5)&(6)\\
\begin{bmatrix}
1&0&0\\
0&0&0\\
0&0&0
\end{bmatrix}&
\begin{bmatrix}
r&0&0\\
0&0&0\\
0&0&0
\end{bmatrix}&
\begin{bmatrix}
1&0&0\\
0&1&0\\
0&0&0
\end{bmatrix}&
\begin{bmatrix}
1&0&0\\
0&r&0\\
0&0&0
\end{bmatrix}&
\begin{bmatrix}
1&0&0\\
0&1&0\\
0&0&1
\end{bmatrix}&
\begin{bmatrix}
1&0&0\\
0&1&0\\
0&0&r
\end{bmatrix}
\end{tabu}
$$
\end{cor}
In the next section, we determine how many solutions each of the corresponding conics has in $\P^2$. For now, we determine to which of these six matrices \(A\) corresponds; however, we do not distinguish between the matrices (1) and (2) as the corresponding conics $X^2=0$ and $rX^2=0$ have equivalent vanishing sets.

\begin{thm}
Let
\begin{align*}
\alpha&=4acf-ae^2-b^2f+bde-cd^2,\\
\beta&=4ac+4af+4cf-b^2-d^2-e^2,
\end{align*}
and let $\gamma$ be defined by permuting the coordinates of $A$ so that $4ac-b^2\ne 0$, and setting $\gamma=4ac-b^2$. Then $A$ corresponds via Corollary~\ref{cor:sixcases} to
\begin{enumerate}
\item[(1) or (2)] if  $\alpha=\beta=0$;
\item[(3)] if $\alpha=0$, $\beta\ne 0$, and $\gamma$ is a nonzero quadratic residue;
\item[(4)] if $\alpha=0$, $\beta\ne 0$, and $\gamma$ is a nonzero quadratic nonresidue;
\item[(5)] if $\alpha$ is a nonzero quadratic residue;
\item[(6)] if $\alpha$ is a nonzero quadratic nonresidue.
\end{enumerate}
\end{thm}
\begin{proof}
Let $P\in\GL_3(\F_p)$ be such that $P^TAP$ is one of the six matrices in Corollary~\ref{cor:sixcases}; these six matrices are distinguished by their ranks and the products of their nonzero eigenvalues. Since $P$ is invertible, the rank of $P^TAP$ is equal to that of $A$. By the rank-nullity theory, the rank of $A$ is given by subtracting the multiplicity of the eigenvalue $0$ from $3$; this multiplicity is given by the number of times the characteristic polynomial of $A$ is divisible by $t$. A \texttt{Magma} \cite{magma} computation shows that the characteristic polynomial of $A$ is given by
\begin{equation*}
t^3+(-a-c-f)t^2+\tfrac{1}{4}(4ac+4af-b^2+4cf-d^2-e^2)t+\tfrac{1}{4}(-4acf+ae^2+b^2f-bde+cd^2).
\end{equation*}

Thus, $A$ has rank $3$ when $\alpha=4\det A$ is nonzero; since
\begin{equation*}
\det P^TAP=\det P^T\cdot\det A\cdot\det P=\det A\cdot(\det P)^2,
\end{equation*}
the determinant of $A$ is a quadratic residue if and only if that of $P^TAP$ is. Cases (5) and (6) of the theorem are now immediate.

Likewise, $A$ has rank $2$ when $\alpha=0$ and $\beta\ne 0$. Since $\beta$ is the sum of $4ac-b^2$, $4af-d^2$, and $4cf-e^2$, these cannot all be zero; the latter two are obtained from $4ac-b^2$ by permuting the coordinates of $A$, which justifies step (1) in computing $\gamma$. Thus, assume that step (1) has been performed. If $a\ne 0$, then the \texttt{Magma} method \texttt{DiagonalForm()} shows that $A$ diagonalizes to
\begin{equation*}
\begin{bmatrix}
a&0&0\\
0&\frac{4ac-b^2}{4a}&0\\
0&0&0
\end{bmatrix}
\end{equation*}
via some $P$. Thus, the product of the nonzero eigenvalues of this matrix is $\gamma=4ac-b^2$, up to a square; the case $c\ne 0$ is similar. Next, if $a=c=0$ and $f\ne 0$, then a similar \texttt{Magma} computation shows that $A$ diagonalizes to
\begin{equation*}
\begin{bmatrix}
f&0&0\\
0&-\frac{e^2}{4f}&0\\
0&0&0
\end{bmatrix}.
\end{equation*}
The product of the nonzero eigenvalues of this matrix is $\gamma=-1$, up to a square. Note that in this case the diagonalizing matrix $P$ is only well-defined if $e$ is nonzero as well, but this is clear as the condition $\alpha=0$ is equivalent to $bf=de$, and $b\ne 0$ as $4ac-b^2\ne 0$ (thus, $4ac-b^2=-b^2$ is also $-1$ up to a square). Finally, if $a=c=f=0$, then either $d=0$ or $e=0$, and the same \texttt{Magma} computation shows that in either case $A$ diagonalizes to 
\begin{equation*}
\begin{bmatrix}
b&0&0\\
0&-b/4&0\\
0&0&0
\end{bmatrix}.
\end{equation*}
The product of the nonzero eigenvalues of this matrix is again $\gamma=-1$, up to a square. Since the matrices (3) and (4) are obtained from these diagonalized forms of $A$ by scaling coordinates, it is clear that $A$ corresponds to (3) if $\gamma$ is a quadratic residue and to (4) if it is not.

Finally, $A$ has rank $1$ when $\alpha=\beta=0$, which establishes the first case.
\end{proof}

We will also need an analogous result for quadratic forms $ax^2+bxy+cy^2$ in two variables, which follows immediately from the previous theorem by setting $d=e=f=0$:
\begin{cor}
The matrix $\left[\begin{smallmatrix}a&b/2\\b/2&c\end{smallmatrix}\right]$ diagonalizes (as in Corollary~\ref{cor:sixcases}) to
\begin{itemize}
\item $\left[\begin{smallmatrix}1&0\\0&0\end{smallmatrix}\right]$ or $\left[\begin{smallmatrix}r&0\\0&0\end{smallmatrix}\right]$ if $4ac-b^2=0$;
\item $\left[\begin{smallmatrix}1&0\\0&1\end{smallmatrix}\right]$ if $4ac-b^2$ is a nonzero quadratic residue;
\item $\left[\begin{smallmatrix}1&0\\0&r\end{smallmatrix}\right]$ if $4ac-b^2$ is a nonzero quadratic nonresidue.
\end{itemize}
\end{cor}

    \section{Counting solutions of a conic in $\P^2$} 

    We precede this undertaking with a brief review of relevant number theory.
    The nonzero elements $\F_p^\times \subset \F_p$ form an abelian
    group under multiplication, and the squaring map 
    \[ \F^\times_p \xrightarrow{x \mapsto x^2} \F^\times_p \] 
    is a homomorphism. If $p$ divides $x^2 - 1 = (x+1)(x-1)$, we must have 
    $x = \pm 1$ mod $p$, so the kernel is $\{\pm 1\}$. 
    The cokernel $\F_p^\times / (\F_p^\times)^2$ is also group of order two: 
    \[ \F_p^\times / (\F_p^\times)^2 \cong \{\pm 1\}.  \]
    \begin{defn}
        The \emph{Legendere symbol} is the quotient map    
        \[ \legendre{\cdot}{p}: 
            \F_p^\times \to \F_p^\times / (\F_p^\times)^2 \cong \{\pm 1\}, \]
        where $\legendre{a}{p}$ is $-1$ if $a$ is a nonresidue,
        and $1$ if $a$ is a nonzero residue. This map extends to a map of magmas
        \[ (\F_p,\times) \to (\{-1,0,1\}, \times) \]
        with $\legendre{0}{p}$ defined as $0$.
    \end{defn}
    
    Note that each nonzero residue has exactly two square-roots (as we are assuming
    $p > 2$ throughout).

    In \S\ref{sec:diagonalizing-quadratic-forms}, we showed that, for an
    appropriate choice of coordinates, $aX^2 + bXY + cY^2$ can be rewritten as
    \begin{equation}\label{eq:general-form-P1} 
        a_1 X_1^2 + c_1 Y_1^2 = 0
    \end{equation} 
    and similarly, $aX^2 + bXY + cY^2 + dXZ + eYZ + fZ^2 = 0$ can be
    rewritten as 
    \begin{equation}\label{eq:general-form-P2} 
        a_2 X_2^2 + c_2 Y_2^2 + f_2 Z_2^2 = 0.
    \end{equation}
    The different subscripts express that the base
    changes involved for \eqref{eq:general-form-P1} and \eqref{eq:general-form-P2}
    need not be the same!

    Recall that the assumptions on the original coefficients 
    $a,b,c,d,e,f \in \Z$ were that 
    \begin{itemize} 
        \item not all of $a,b,c$ are zero, and
        \item $\gcd(a,b,c,d,e,f) = 1$.  
    \end{itemize} 
    If $p$ divides $a,b$, and $c$, then $a_1 = c_1 = 0$ above. However, 
    it will never be the case that $a_2 = c_2 = f_2 = 0$.

    \subsection{Extraneous solutions}
    \label{subsec:counting-P1-solutions}

    Fix a quadratic nonresidue $r \in \F_p$. By scaling the equation
    \eqref{eq:general-form-P1} and by scaling and permuting the coordinates as
    appropriate, we can obtain one of the following equivalent forms:
    \begin{enumerate} 
        \item $0 = 0$ if both $a_1, c_1$ are zero, 
        \item $X^2 = 0$ if exactly one of $a_1, c_1$ is zero, 
        \item $X^2 + Y^2 = 0$ if $a_1 c_1$ is a nonzero quadratic residue, and
        \label{case:x^2+y^2=0} \item $X^2 + rY^2 = 0$ if $a_1 c_1$ is a nonzero 
        quadratic nonresidue.  
    \end{enumerate}

    The first of these has $|\P^1| = p+1$ solutions. 

    The second has one solution, namely $[0:1]$.

    Evidently $Y$ must be nonzero for the last two cases, so we can rewrite them as
    $(X/Y)^2 = -1$ and $(X/Y)^2 = -r$ respectively. If $-1$ is a quadratic residue,
    then $-r$ is not. It follows that $X^2 + Y^2 = 0$ has two solutions and $X^2 +
    rY^2 = 0$ has none. If $-1$ is a nonresidue, the situation is reversed: $X^2 +
    Y^2 = 0$ has no solutions and $X^2 + rY^2 = 0$ has two solutions.

    \subsection{Total solutions}
    \label{subsec:counting-P2-solutions}

    Let $r$ be a quadratic nonresidue in $\F_p$ as before. Just as in
    \S\ref{subsec:counting-P1-solutions}, we can rewrite \eqref{eq:general-form-P2}
    as one of the following: 
    \begin{enumerate} 
        \item $X^2 = 0$ if two of $a_2, c_2, f_2$ are zero, 
        \item $X^2 + Y^2 = 0$ if exactly one of $a_2, c_2, f_2$ is zero,
        and the product of the other two is a quadratic residue, 
        \item $X^2 + rY^2 = 0$ if exactly one of $a_2, c_2, f_2$ is zero, 
        and the product of the other two is a quadratic nonresidue, 
        \item $X^2 + Y^2 + Z^2 = 0$ if $a_2 c_2 f_2$ is a nonzero quadratic residue, and 
        \item $X^2 + Y^2 + rZ^2 = 0$ if $a_2 c_2 f_2$ is
        a nonzero quadratic nonresidue.  
    \end{enumerate}

    Now we count the number of solutions in each of these cases.

    \subsubsection*{Rank 1 (first case)} 
    The equation $X^2 = 0$ has $p + 1$ solutions.

    \subsubsection*{Rank 2 (second and third cases)} 
    The only solution with $Y=0$ is $[0:0:1]$. So assume $Y \neq 0$, and rewrite the 
    equations as $(X/Y)^2 = -1$ and $(X/Y)^2 = -r$ respectively.

    Suppose $-1$ is a quadratic residue mod $p$. The equation $(X/Y)^2 = -1$ gives
    two possibilities for $X/Y$, and thus $2p$ solutions in $\mathbb{P}^2$. Hence
    $X^2 + Y^2 = 0$ has $2p + 1$ solutions in total. On the other hand, $(X/Y)^2 =
    -r$ has no solutions, meaning $[0:0:1]$ is the only solution to $X^2 + rY^2 =
    0$.

    If $-1$ is a nonresidue, the situation is reversed, and $X^2 + Y^2 = 0$ has one
    solution while $X^2 + rY^2 = 0$ has $2p + 1$ solutions.

    \subsubsection*{Rank 3 (fourth and fifth cases)} 
    Suppose $-1$ is a quadratic residue mod $p$. If $Z = 0$, then we are in the setting of
    \S\ref{subsec:counting-P1-solutions} case (\ref{case:x^2+y^2=0}), which has two
    solutions. Otherwise, we can divide by $Z$ and obtain 
    \begin{align*} 
        (X/Z)^2 + (Y/Z)^2 &= -1 \\ 
        (X/Z)^2 + (Y/Z)^2 &= -r 
    \end{align*} 
    respectively. By Proposition \ref{prop:circle-solutions} below, these each have $p - 1$
    solutions. Therefore $X^2 + Y^2 + Z^2 = 0$ and $X^2 + Y^2 + rZ^2 = 0$ each have
    $p+1$ solutions.

    Suppose $-1$ is a nonresidue. There are no solutions with $Z=0$, so rewrite the
    equations as above. By Proposition \ref{prop:circle-solutions} again, they each
    have $p + 1$ solutions. \footnote{\textcolor{red}{add comment about isomorphism
    to $\P^1$}}

    \begin{prop}\label{prop:circle-solutions} 
        For $a \neq 0$, the equation $x^2 + y^2 = a$ has $p - \legendre{-1}{p}$ 
        solutions over $\F_p$.
    \end{prop} 
    \begin{proof} 
        Let $S_a \subset \mathbb{A}^2$ denote the solution set of $x^2 + y^2 = a$. 
        First we will show the statement for $a=1$.
    	
    	Let $T$ denote the square-roots of $-1$ in $\F_p$. In the case
        that $-1$ is a nonresidue, $T$ is empty. Consider the maps $f\colon S_1
        \setminus (1,0) \to \F_p \setminus T$ and $g\colon \F_p
        \setminus T \to S_1 \setminus (1,0)$ defined by 
        \begin{align*} 
            f(x,y) &= \frac{y}{x-1} \\ 
            g(m) &= \left(\frac{m^2 - 1}{m^2 + 1}, \frac{-2m}{m^2 + 1}\right).
        \end{align*} 
        It is easy to check that they are inverse to one
        another, from which it follows that $|S_1| = |\F_p| - |T| + 1 = p -
        \legendre{-1}{p}$ as desired.
    	
    	Note that if $a,a'$ are both nonzero residues (or both nonresidues)
        then $|S_a| = |S_{a'}|$. From the preceding, we have the claim for all residues
        $a$. Observe that $S_0 \cup \cdots \cup S_{p-1}$ gives a partition of
        $\mathbb{A}^2$. Since 
        \[ |S_0| = \begin{cases}
            1 & \text{if } \legendre{-1}{p} = -1 \\
            -1 & \text{if} \legendre{-1}{p} = 1 \\
        \end{cases} \] 
        for a nonresidue $a$ we have 
        \[ |S_a| = \frac{2}{p-1} 
                    \left(p^2 - |S_0| - \frac{p-1}{2}|S_1|\right) = 
                    p - \legendre{-1}{p}. \qedhere \] 
    \end{proof}

    \begin{example}[A zoo of possibilities]
    \label{exa:all-possibilities}
    	
    In the table below, the first two columns reflect the discussion
    immediately preceding. The remaining columns consider the number of extraneous
    solutions. The entries marked ``N/A'' are impossible and explanation is given
    afterwards. For all other entries, examples are given for the specific case
    $p=3$.  
    \begin{center} % TODO: convert to booktabs
        \begin{tabular}{|c|c||c|c|c|c|} \hline \multicolumn{2}{|c||}{} &
    \multicolumn{4}{c|}{\# soln. in $\{Z=0\}$}  \\ \hline rank of proj. & \# soln.
    in $\mathbb{P}^2$ & 0 & 1 & 2 & $p + 1$\\ \hline \hline 1 & $p+1$ & N/A & $x^2$
    & N/A & 1 \\ \hline 2 & 1 & $x^2 + y^2$ & $x^2 + 1$ & N/A & N/A \\ \hline 2 &
    $2p+1$ & N/A & $x^2 + x$ & $xy$ & x \\ \hline 3 & $p+1$ & $x^2 + y^2 + 1$ &
    $x+y^2$ & $x^2 + 2y^2 + 1$ & N/A \\ \hline \end{tabular}
    \end{center} 
    \begin{itemize}
        \item When the projectivization has rank 1, it is a double line. If this
        line is distinct from $\{Z = 0\}$, they meet at exactly one point. 
        Otherwise they are the same line, giving $p+1$ solutions in $\{Z=0\}$. 
        \item When the projectivization has rank 2 and has $2p + 1$ solutions, it
        is the union of two intersecting lines. These lines cannot be disjoint 
        from $\{Z=0\}$.
        \item There are $p+1$ solutions in $\{Z = 0\}$ only when the 
        coefficients $a,b,c$ are all zero. In particular, such a conic cannot 
        have a projectivization of rank 3.
    \end{itemize} 
    \end{example}

    \section{Counting solutions of a conic in $\A^2$}
    \label{sec:affine-solutions}

    Given a conic, we now have almost all of the necessary pieces to determine how
    many solutions it has in $\A^2_{\F_p}$. The remaining case is that of a
    conic over $\F_2$.

    Fortunately there are very few such conics, and the solutions to 
    \[ ax^2 + bxy + cy^2 + dx + ey + f = 0 \] 
    are the same as the solutions to 
    \[ bxy + (a+d)x + (c+e)y + f = 0.  \] 
    At this point, one can simply list out all the
    possibilities. It can be checked that the number of solutions in each case can
    be concisely written as $2 + b(-1)^{(a+d)(c+e) + f}$, where $b$ is either 0 or
    1. \textcolor{red}{I have temporarily put this here, but this might flow better
    if moved into the proof of the theorem below. }

    (remainder to be written)

    \begin{thm} 
        main theorem determining $F(a,b,c,d,e,f,p)$ 
    \end{thm}

    % TODO: get a reference for the following fact (which we'll probably need)?
    \begin{prop} 
        For an odd prime $p$, the Legendre symbol $\legendre{-1}{p}$ is
        given by 
        \[ \legendre{-1}{p} = \begin{cases} 
            1 & \text{if } p \cong 1 \pmod 4 \\ 
            -1 & \text{if } p \cong 3 \pmod 4.  
        \end{cases} \]
    \end{prop}

    \section{To the $p$-adics via Hensel's lemma} (to be written)

    \section{To $\mathbb{Q}$ via Hasse-Minkowski} (to be written) '

\begin{filecontents}{references.bib}
@book {omeara,
    AUTHOR = {O'Meara, O. Timothy},
     TITLE = {\href{https://link.springer.com/book/10.1007\%2F978-3-642-62031-7}{Introduction to quadratic forms}},
    SERIES = {Classics in Mathematics},
      NOTE = {Reprint of the 1973 edition},
 PUBLISHER = {Springer-Verlag, Berlin},
      YEAR = {2000},
     PAGES = {xiv+342},
      ISBN = {3-540-66564-1},
   MRCLASS = {11Exx},
  MRNUMBER = {1754311},
}
@article{magma,
author={Wieb Bosma and John Cannon and Catherine Playoust},
title={\href{http://www.sciencedirect.com/science/article/pii/S074771719690125X}{The Magma algebra system {I}: The user language}},
journal={Journal of Symbolic Computation},
volume={24},
year={1997},
pages={235--265}
}
\end{filecontents}

\bibliographystyle{plain}
\bibliography{references}

\end{document}
