%Document setup
\documentclass[10pt,a4paper]{amsart} 

\usepackage[latin1]{inputenc}
\usepackage{mathtools} 
\usepackage{amssymb} 
\usepackage{mathrsfs} % gives mathscr font 
\usepackage{graphicx}
%\usepackage[headsep=0.15in, left=0.5in, right=0.5in, top=0.6in, bottom=0.5in]{geometry} 
%\usepackage[textlf,mathlf]{MinionPro} 
\usepackage{fancyhdr}
%\usepackage{xypic}
\usepackage{tikz-cd} 
\usepackage{todonotes}
\usepackage{verbatim}
\usepackage{filecontents}
\usepackage[colorlinks=true]{hyperref}
\usepackage{tabu}
\usepackage{multicol}
\usepackage{booktabs}
\usepackage{centernot}

%Environments
\numberwithin{equation}{section} 
\numberwithin{figure}{section}
\numberwithin{table}{section}
\theoremstyle{definition} 
\theoremstyle{plain} 
\newtheorem{thm}{\protect\theoremname}[section]
\newtheorem{example}[thm]{\protect\examplename}
\theoremstyle{remark} 
\newtheorem*{rem*}{\protect\remarkname}
\theoremstyle{plain}
\newtheorem{cor}[thm]{\protect\corollaryname}
\theoremstyle{definition} 
\newtheorem{defn}[thm]{\protect\definitionname}
\theoremstyle{plain} 
\newtheorem{prop}[thm]{\protect\propositionname}
\theoremstyle{plain} 
\newtheorem{lem}[thm]{\protect\lemmaname}

\providecommand{\definitionname}{Definition}
\providecommand{\examplename}{Example} 
\providecommand{\lemmaname}{Lemma}
\providecommand{\propositionname}{Proposition}
\providecommand{\remarkname}{Remark} 
\providecommand{\corollaryname}{Corollary}
\providecommand{\theoremname}{Theorem}

%macros, etc.
\newcommand{\legendre}[2]{\genfrac{(}{)}{}{}{#1}{#2}}
\renewcommand{\arraystretch}{1.2} 

\newcommand{\A}{\mathbb{A}}
\renewcommand{\P}{\mathbb{P}} 
\newcommand{\F}{\mathbb{F}}
\newcommand{\Z}{\mathbb{Z}}
\newcommand{\Q}{\mathbb{Q}}
\newcommand{\R}{\mathbb{R}}
\newcommand{\GL}{\operatorname{GL}}
\newcommand{\Span}{\operatorname{Span}}
\newcommand{\trsp}[1]{{{#1}^{\top}\!\!}}
\newcommand{\ndiv}{\centernot|}

\begin{document} 
	
	\title{Mod $p$ solutions to conics, and applications} 
	\author{Xianglong Ni, Oron Propp, and Miguel Young}
	\maketitle
	
	\begin{abstract} 
		We count the number of solutions of any affine plane conic modulo a prime.
		We use this result to determine whether such a conic has rational solutions. 
	\end{abstract}
	
	\tableofcontents
	
	\section{Introduction}\label{sec:intro} 
	
	% Get a reference for general nonsense over the reals
	A \emph{conic section} on the real plane $\R^2$ is the solution set of 
	a polynomial
	\[ C: ax^2 + bxy + cy^2 + dx + ey + f = 0, \]
	where the degree-2 coefficients $a, b, c$ are not all zero. This conic can be \emph{projectivized} by adding an additional
	unknown to make the polynomial homogeneous:
	\[ \widetilde{C}: aX^2 + bXY + cY^2 + dXZ + eXZ + fZ^2 = 0. \]
	This gives a quadratic form on $\R^3$; when this form is degenerate, we
	say that the original conic, $C$, is also degenerate.
	
	The homogeneous part of $C$, $ax^2 + bxy + cy^2$, gives a quadratic form
	on $\R^2$, with matrix
	\[ \begin{bmatrix}
	a & b/2 \\
	b/2 & c
	\end{bmatrix}. \]
	The determinant of this matrix, called the \emph{discriminant} and
	denoted $-4\Delta = b^2 - 4ac$, gives a characterization of the shape of $C$
	when it is nondegenerate. Positive, zero, and negative discriminant
	correspond to $C$ being an ellipse, a parabola, or a hyperbola. The
	projectivized conic also gives rise to a quadratic form; we say $C$ is
	degenerate when the matrix of $\widetilde{C}$ is singular. The curve
	will instead look like a pair of parallel or intersecting lines, a double
	line (where each point of the line is a solution of the conic and its 
	derivative), a single point, or no points at all. 
	
	The above discussion of the \emph{real} points of $C$ is fairly classical,
	and a full classification is well-understood. We can instead ask that solutions
	be in some other field $K$.
	The situation when $K=\Q$ turns out to be quite complicated. It is
	characterized by the Hasse--Minkowski theorem, of which we quote a special
	case:
	\begin{thm}[Hasse--Minkowski]
		A quadratic form $C$ over $\Q$ has a solution if and only if it has solutions
		over all the completions of $\Q$, namely, the reals, $\R$, and the $p$-adic
		numbers, $\Q_p$, for all primes $p$.
	\end{thm}
	The case when $K = \R$ is, as mentioned, classical. To solve over $\Q_p$,
	we can simply ask for solutions in $\Z_p$, the $p$-adic integers. We
	can in turn get these solutions through Hensel's lemma, which allows us
	to lift solutions from the finite field $\F_p$ to $\Z_p$. Our ultimate
	goal is to determine when there exist rational points on a given conic $C$, by lifting
	a characterization of its $\F_p$-points.
	
	This paper is roughly divided into two parts. The first part
	is bootstrapping this process by counting solutions over $\F_p$. To do this,
	we study the projectivization of our chosen conic, discussed in
	\S\ref{sec:projectivization-of-the-problem}. We then diagonalize 
	the projective curve through a procedure described in 
	\S\ref{sec:diagonalizing-quadratic-forms}, reducing to a finite number of cases. We work out these cases
	of diagonalized curves in \S\ref{sec:proj-solutions}, and finally
	drop back down to the affine case in \S\ref{sec:affine-solutions}. Our result
	is a complete characterization of the number of solutions, given by 
	Theorem \ref{thm:main-thm}.
	In the second part, we lift solutions to $\Z_p$ by way of
	Hensel's lemma, and use the Hasse--Minkowski theorem
	to completely characterize existence of rational points. This is the content of \S\ref{sec:rational-soln}; our main result is Theorem~\ref{thm:rational-solutions}.
	
	\subsection*{Acknowledgments} Oron wrote sections \S\ref{sec:diagonalizing-quadratic-forms} and \S\ref{sec:affine-solutions}. Together, Xianglong and Oron worked out much of \S\ref{sec:proj-solutions} and \S\ref{sec:rational-soln}, which were later written up by Xianglong (who also wrote \S\ref{sec:projectivization-of-the-problem}). Miguel ran experimental tests to both motivate and verify our computations, wrote \S\ref{sec:intro}, and helped format, edit, and proofread.
	
	We would like to thank the 18.821 course staff for organizing this project, and Rasmus Johansen in particular for his guidance and suggestions. We are also grateful
	to Ari Nieh and Stuart Thomson for their feedback on a draft of this paper.
	
	
	\section{Preliminaries}\label{sec:projectivization-of-the-problem}
	
	Let us begin by fixing a conic
	\begin{equation}\label{eq:original-conic} 
	C \colon ax^2 + bxy + cy^2 + dx + ey + f = 0 
	\end{equation} 
	with integer coefficients. We require that at least one of 
	$a$, $b$, and $c$ is nonzero, and that $\gcd(a,b,c,d,e,f)=1$.
	Fix a base field $\F_p$ where $p$ is an odd prime. (We defer treatment of the case $p = 2$ to \S\ref{sec:affine-solutions}.)
	\begin{defn}
		Let $\A^n_{\F_p}$ denote $n$-dimensional affine space over $\F_p$. Define 
		$n$-dimensional projective space over $\F_p$ as the quotient
		\[ \P^n_{\F_p} = (\A^{n+1}_{\F_p}\setminus \{0\})/{\sim}, \]
		where we declare that $(x_1,\ldots,x_{n+1}) \sim 
		(\lambda x_1,\ldots,\lambda x_{n+1})$ for $\lambda$ a nonzero scalar. We
		will primarily work with $n = 2$, and suppress the ground field from the 
		notation. We will write $[X:Y:Z]\in \P^2$ to denote the equivalence 
		class of $(X,Y,Z)\in \A^3 \setminus \{0\}$.
	\end{defn}
	
	With this construction in mind, we can \emph{projectivize} our conic
	\eqref{eq:original-conic} from $\A^2$ to $\P^2$, by writing it as a 
	homogenous equation: 
	\begin{equation}\label{eq:projective-conic} 
	\widetilde{C} \colon aX^2 + bXY + cY^2 + dXZ + eYZ + fZ^2 = 0. 
	\end{equation}
	Note that homogeneity ensures it is well-defined to ask when this
	equation has a solution in $\P^2$.
	
	Let $C(\F_p)$ denote the solution set of \eqref{eq:original-conic} in $\A^2$. Our aim
	is to characterize $|C(\F_p)|$ as a function of the coefficients $a,b,c,d,e,f$ and 
	the prime $p$. The solutions of \eqref{eq:projective-conic} define a set
	$\widetilde{C}(\F_p) \subset \P^2$.
	
	There is an evident injection $C(\F_p) \rightarrowtail \widetilde{C}(\F_p)$ sending $(x,y)$ to
	$[x:y:1]$. However, elements of $\widetilde{C}(\F_p)$ of the form $[X:Y:0]$ do not 
	correspond to elements of $C(\F_p)$. 
	These are solutions ``at infinity'' in the projective plane.
	Thankfully, these extraneous solutions are easily characterized: they are the
	solutions to the equation
	\begin{equation}\label{eq:projective-conic-at-infinity} 
	\widetilde{C}_0 \colon aX^2 + bXY + cY^2 = 0
	\end{equation} 
	on the projective line $\P^1$. Call this solution set $\widetilde{C}_0(\F_p)\subset \P^1$.  
	
	\begin{lem}\label{lem:relate-solutions-affine-proj}
		$|\widetilde{C}(\F_p)| = |C(\F_p)| + |\widetilde{C}_0(\F_p)|$.
	\end{lem} 
	\begin{proof}
		A solution $[X:Y:Z]$ of $\widetilde{C}$ corresponds 
		to a solution $(X/Z,Y/Z)$ of $C$ when $Z \neq 0$,
		and to a solution $[X:Y]$ of $\widetilde{C}_0$ 
		when $Z = 0$.  
	\end{proof}
	
	We will also need a bit of number theory, since our characterization
	involves the notion of quadratic residues:
	\begin{defn}
		The nonzero elements $\F_p^\times \subset \F_p$ form an abelian
		group under multiplication, and the squaring map 
		\[ \F^\times_p \xrightarrow{x \mapsto x^2} \F^\times_p \]
		is a homomorphism. We will call the image of this map the
		\emph{quadratic residues mod $p$}, or just the \emph{residues} (where the
		prime $p$ is inferred from context). We will also regard $0\in\F_p$ as a residue.
	\end{defn}
	
	If $p$ divides $x^2 - 1 = (x+1)(x-1)$, we must have $x \equiv \pm 1 \bmod{p}$, so 
	the kernel of the map is $\{\pm 1\}$. Thus each nonzero residue has exactly two 
	square roots. The cokernel of this map is also a group of order two: 
	\[ \F_p^\times / (\F_p^\times)^2 \cong \{\pm 1\}.  \]
	This gives a convenient characterization of quadratic residues:
	\begin{defn}
		The \emph{Legendre symbol} is the cokernel map    
		\[ \legendre{\cdot}{p}: 
		\F_p^\times \to \F_p^\times / (\F_p^\times)^2 \cong \{\pm 1\}, \]
		where $\legendre{a}{p}$ is $-1$ if $a$ is a nonresidue,
		and $1$ if $a$ is a nonzero residue. This map extends to a multiplication-respecting map
		\[ \F_p \to \{-1,0,1\} \]
		with $\legendre{0}{p}$ defined as $0$.
	\end{defn}
	
	\section{Diagonalization of quadratic forms}
	\label{sec:diagonalizing-quadratic-forms} 
	
	In this section, we aim to simplify the equation defining the projectivized conic $\widetilde{C}$ via coordinate-changes of $\P^2$. In doing so,
	we will make extensive use of \texttt{Magma} \cite{magma}, a computer 
	algebra system. Let
	\begin{equation*}
	A=\begin{bmatrix}
	a&b/2&d/2\\
	b/2&c&e/2\\
	d/2&e/2&f
	\end{bmatrix}, \hspace{1cm}
	v = \begin{bmatrix}
	X\\
	Y\\
	Z
	\end{bmatrix}.
	\end{equation*}
	The quadratic form $\trsp{v} A v$ is precisely the equation for $\widetilde{C}$. In this sense, the matrix $A$ represents $\widetilde{C}$ in the coordinates $[X,Y,Z]$.
	
	Now, any $P\in\GL_3(\F_p)$ gives rise to a change of coordinates $v\mapsto Pv$ in $\P^2$. Since
	\begin{equation*}
	\trsp{(Pv)} A(Pv)=\trsp{v}(\trsp{P} AP)v,
	\end{equation*}
	the matrix representing $\widetilde{C}$ is $\trsp{P}AP$ in the new coordinates. This transformation $A \mapsto \trsp{P}A P$ does not change the number of solutions of the corresponding conic, as $P$ is invertible. The following result is well-known (see for instance \cite[Prop.~42:1]{omeara}):
	\begin{thm}
		\label{thm:diag}
		There exists some $P\in\GL_3(\F_p)$ such that $\trsp{P}AP$ is diagonal.
	\end{thm}
	This significantly simplifies the set of conics which we must consider: a diagonal matrix
	\begin{equation*}
	\begin{bmatrix}
	\lambda_1&0&0\\
	0&\lambda_2&0\\
	0&0&\lambda_3
	\end{bmatrix}
	\end{equation*}
	corresponds to the conic defined by $\lambda_1x^2+\lambda_2y^2+\lambda_3z^2=0$ (note that we cannot have $\lambda_1=\lambda_2=\lambda_3=0$ by our initial assumptions). By further scaling and permuting coordinates as appropriate, we immediately obtain the following:
	\begin{cor}
		\label{cor:sixcases}
		Fix an element $r\in\F_p$ that is not a square. There exists some $P\in\GL_3(\F_p)$ such that $\trsp{P}AP$ is equal to a (nonzero) scalar multiple of one of the following:
		$$
		\begin{tabu}{cccccc}
		(1)&(2)&(3)&(4)&(5)\\
		\begin{bmatrix}
		1&0&0\\
		0&0&0\\
		0&0&0
		\end{bmatrix}&
		\begin{bmatrix}
		1&0&0\\
		0&1&0\\
		0&0&0
		\end{bmatrix}&
		\begin{bmatrix}
		1&0&0\\
		0&r&0\\
		0&0&0
		\end{bmatrix}&
		\begin{bmatrix}
		1&0&0\\
		0&1&0\\
		0&0&1
		\end{bmatrix}&
		\begin{bmatrix}
		1&0&0\\
		0&1&0\\
		0&0&r
		\end{bmatrix}
		\end{tabu}
		$$
	\end{cor}
	In the next section, we determine how many solutions each of the corresponding conics has in $\P^2$. For now, we determine to which of these five matrices $A$ corresponds. Note that we do not need to distinguish between scalar multiples of these matrices as the corresponding conics have equivalent solution sets.
	
	\begin{lem}
		\label{thm:diag-P2}
		In the context of Corollary \ref{cor:sixcases}, let
		\begin{align*}
		\alpha &= 4acf - ae^2 - b^2f + bde - cd^2,\\
		\beta &= 4ac + 4af + 4cf - b^2 - d^2 - e^2,
		\end{align*}
        and let $\gamma$ be the first nonzero element of the tuple
        \begin{equation*}
        (4ac-b^2,4af-d^2,4cf-e^2).
        \end{equation*}
		Then, the matrix $A$ corresponds to one of the cases in 
		Corollary \ref{cor:sixcases} when
		\begin{enumerate}
			\item\label{case:1} $\alpha=\beta=0$;
			\item\label{case:11} $\alpha=0$, $\beta\ne 0$, 
			and $\gamma$ is a nonzero residue;
			\item\label{case:1r} $\alpha=0$, $\beta\ne 0$, 
			and $\gamma$ is a nonresidue;
			\item\label{case:111} $\alpha$ is a nonzero residue;
			\item\label{case:11r} $\alpha$ is a nonresidue.
		\end{enumerate}
	\end{lem}
	\begin{proof}
		Let $P\in\GL_3(\F_p)$ be such that $\trsp{P}AP$ is a (nonzero) scalar multiple of one of the six matrices in Corollary~\ref{cor:sixcases}; these six matrices are distinguished by their ranks and the products of their nonzero eigenvalues. Since $P$ is invertible, the rank of $\trsp{P}AP$ is equal to that of $A$. By the rank-nullity theorem, the rank of $A$ is given by 3 minus the multiplicity of the eigenvalue $0$: this multiplicity is the highest power of $t$ dividing the characteristic polynomial of $A$. A \texttt{Magma} computation shows that the characteristic polynomial of $A$ is given by
		\begin{equation*}
		t^3+(-a-c-f)t^2+\tfrac{1}{4}(4ac+4af-b^2+4cf-d^2-e^2)t+\tfrac{1}{4}(-4acf+ae^2+b^2f-bde+cd^2).
		\end{equation*}
		Thus $A$ has rank $1$ when $\alpha=\beta=0$, which gives \eqref{case:1}.
		
		Likewise, $A$ has rank $2$ when $\alpha=0$ and $\beta\ne 0$. Since $\beta$ is the sum of $4ac-b^2$, $4af-d^2$, and $4cf-e^2$, these cannot all be zero; the latter two are obtained from $4ac-b^2$ by permuting the coordinates of $A$, which justifies the computation of $\gamma$. Thus, assume that $\gamma$ has been determined in this manner. If $a\ne 0$, then the \texttt{Magma} method \texttt{DiagonalForm()} shows that $A$ diagonalizes via
		\begin{equation*}
		\begin{bmatrix}
		1 & -\dfrac{b}{2a} & \dfrac{be - 2cd}{4ac - b^2}\\
		0 & 1 & \dfrac{-2ae + bd}{4ac - b^2} \\
		0 & 0 & 1
		\end{bmatrix}^\top\!
		A
		\begin{bmatrix}
		1 & -\dfrac{b}{2a} & \dfrac{be - 2cd}{4ac - b^2}\\
		0 & 1 & \dfrac{-2ae + bd}{4ac - b^2} \\
		0 & 0 & 1
		\end{bmatrix}
		=
		\begin{bmatrix}
		a&0&0\\
		0&\dfrac{4ac-b^2}{4a}&0\\
		0&0&0
		\end{bmatrix}.
		\end{equation*}
		Thus, the product of the nonzero eigenvalues of this matrix is $\gamma=4ac-b^2$, up to a square. The case $c\ne 0$ is similar.
		
		Next, if $a=c=0$ and $f\ne 0$, then a similar \texttt{Magma} computation shows that $A$ diagonalizes via
		\begin{equation*}
		\begin{bmatrix}
		0 & 0 & 1\\
		0 & 1 & \dfrac{2bf - de}{e^2}\\
		1 & -\dfrac{e}{2f} & -\dfrac{b}{e}
		\end{bmatrix}^\top\!
		A
		\begin{bmatrix}
		0 & 0 & 1\\
		0 & 1 & \dfrac{2bf - de}{e^2}\\
		1 & -\dfrac{e}{2f} & -\dfrac{b}{e}
		\end{bmatrix}
		=
		\begin{bmatrix}
		f&0&0\\
		0&-\dfrac{e^2}{4f}&0\\
		0&0&0
		\end{bmatrix}.
		\end{equation*}
		The product of the nonzero eigenvalues of this matrix is $\gamma=-1$, up to a square. Note that in this case the diagonalizing matrix $P$ is only well-defined if $e$ is nonzero as well, but this is clear as the condition $\alpha=0$ is equivalent to $bf=de$, and $b\ne 0$ as $4ac-b^2\ne 0$ (thus, $4ac-b^2=-b^2$ is also $-1$ up to a square). Finally, if $a=c=f=0$, then either $d=0$ or $e=0$, and the same \texttt{Magma} computation shows
		\begin{align*}
		\begin{bmatrix}
		1 & -1/2 & -e/b\\
		1 & 1/2 & 0\\
		0 & 0 & 1
		\end{bmatrix}^\top\!
		A
		\begin{bmatrix}
		1 & -1/2 & -e/b\\
		1 & 1/2 & 0\\
		0 & 0 & 1
		\end{bmatrix}
		&=
		\begin{bmatrix}
		b&0&0\\
		0&-b/4&0\\
		0&0&0
		\end{bmatrix}\\
		\begin{bmatrix}
		1 & -1/2 & 0\\
		1 & 1/2 & -d/b\\
		0 & 0 & 1
		\end{bmatrix}^\top\!
		A
		\begin{bmatrix}
		1 & -1/2 & 0\\
		1 & 1/2 & -d/b\\
		0 & 0 & 1
		\end{bmatrix}
		&=
		\begin{bmatrix}
		b&0&0\\
		0&-b/4&0\\
		0&0&0
		\end{bmatrix}.
		\end{align*}
		The product of the nonzero eigenvalues of this matrix is again $\gamma=-1$, up to a square. Since the matrices \eqref{case:11} and \eqref{case:1r} are obtained from these diagonalized forms of $A$ by scaling coordinates, it is clear that $A$ corresponds to \eqref{case:11} if $\gamma$ is a residue and to \eqref{case:1r} if it is not.
		
		Lastly, $A$ has rank $3$ when $\alpha$ is nonzero. Since
		\begin{equation*}
		\det \trsp{P}AP=\det \trsp{P} \cdot\det A\cdot\det P=\det A\cdot(\det P)^2,
		\end{equation*}
		the determinant of $A$ is a quadratic residue if and only if that of $\trsp{P}AP$ is. Cases \eqref{case:111} and \eqref{case:11r} of the theorem follow.
	\end{proof}
	
	We will also need an analogous result for quadratic forms $ax^2+bxy+cy^2$ in two variables, which follows immediately from the lemma by setting $d=e=f=0$:
	\begin{cor}\label{cor:diag-P1}
		The matrix $\left[\begin{smallmatrix}a&b/2\\b/2&c\end{smallmatrix}\right]$ diagonalizes (as in Corollary~\ref{cor:sixcases}, though with $P \in \GL_2(\F_p)$) to a (nonzero) scalar multiple of
		\begin{itemize}
			\item $\left[\begin{smallmatrix}0&0\\0&0\end{smallmatrix}\right]$ if $a=b=c=0$;
			\item $\left[\begin{smallmatrix}1&0\\0&0\end{smallmatrix}\right]$ if $4ac-b^2=0$;
			\item $\left[\begin{smallmatrix}1&0\\0&1\end{smallmatrix}\right]$ if $4ac-b^2$ is a nonzero quadratic residue;
			\item $\left[\begin{smallmatrix}1&0\\0&r\end{smallmatrix}\right]$ if $4ac-b^2$ is a nonzero quadratic nonresidue.
		\end{itemize}
	\end{cor}
	
	\begin{rem*}
		The discussion in this section suggests a definition for a \emph{degenerate} conic over $\F_p$: they should be the conics whose corresponding quadratic form is degenerate, or equivalently, whose corresponding matrix is singular.
	\end{rem*}
	
	\section{Counting solutions of a conic in $\P^2$}
	\label{sec:proj-solutions}
	
	Now that we have diagonalized our conics to a more easily-analyzed form, we can
	work on solving the projectivized conics, $\widetilde{C}$ and 
	$\widetilde{C}_0$. We will treat the easier $\P^1$ case first.
	\begin{prop}[Counting Solutions in $\P^1$]
		\label{prop:counting-P1-solutions}
		Fix a prime $p$ and let $r$ be a nonresidue. When diagonalized,
		$\widetilde{C}_0$ has one of the following forms:
		\begin{enumerate}
			\item \label{case:0} $0 = 0$, with $p+1$ solutions.
			\item $X^2 = 0$, with $1$ solution, namely $[0:1]$.
			\item \label{case:x^2+y^2=0} $X^2 + Y^2 = 0$, with 
			$1 + \legendre{-1}{p}$ solutions.
			\item $X^2 + rY^2 = 0$, with 
			$1 - \legendre{-1}{p}$ solutions.
		\end{enumerate}    
	\end{prop}
	\begin{proof}
		The first and second cases are clear; note that $|\P^1| = p+1$.
		
		In the other two cases, $Y$ must be nonzero. We can write them as
		$(X/Y)^2 = -1$ and $(X/Y)^2 = -r$ respectively. If $-1$ is a residue,
		then $-r$ is not. It follows that $X^2 + Y^2 = 0$ has two solutions and $X^2 +
		rY^2 = 0$ has none. If $-1$ is a nonresidue, the situation is reversed: $X^2 +
		Y^2 = 0$ has no solutions and $X^2 + rY^2 = 0$ has two solutions.
	\end{proof}
	
	Now, let us treat the $\P^2$ case.
	
	\begin{prop}[Counting Solutions in $\P^2$]
		\label{prop:counting-P2-solutions}
		Fix a prime and let $r$ be a nonresidue. When diagonalized $\widetilde{C}$ 
		has one of the following forms:
		\begin{enumerate} 
			\item\label{case:rank1_x^2=0} $X^2 = 0$, with $p+1$ solutions.
			\item\label{case:rank2_x^2+y^2=0} $X^2 + Y^2 = 0$, with
			$p(1+\legendre{-1}{p}) + 1$ solutions.
			\item\label{case:rank2_x^2+ry^2=0} $X^2 + rY^2 = 0$, with
			$p(1+\legendre{-1}{p}) - 1$ solutions.
			\item\label{case:rank3_x^2+y^2+z^2=0} $X^2 + Y^2 + Z^2 = 0$ has $p+1$ solutions.
			\item\label{case:rank3_x^2+y^2+rz^2=0} $X^2 + Y^2 + rZ^2 = 0$ has $p+1$ solutions.
		\end{enumerate}
		Note that the first three equations have $p$ times plus $1$ the number
		of solutions of the corresponding equaton in Proposition~\ref{prop:counting-P1-solutions}.
	\end{prop}
	\begin{proof} We treat the cases by rank.
		
		\noindent
		\textit{Rank 1: \eqref{case:rank1_x^2=0}.} 
		The equations $X^2 = 0$ and $rX^2 = 0$ each have $|\P^1|=p + 1$ solutions.
		
		\noindent
		\textit{Rank 2: \eqref{case:rank2_x^2+y^2=0} and \eqref{case:rank2_x^2+ry^2=0}.} 
		The only solution with $Y=0$ is $[0:0:1]$. So assume $Y \neq 0$, and rewrite the 
		equations as $(X/Y)^2 = -1$ and $(X/Y)^2 = -r$ respectively.
		
		Suppose $-1$ is a residue. The equation $(X/Y)^2 = -1$ gives
		two possibilities for $X/Y$, and thus $2p$ solutions in $\P^2$. Hence
		$X^2 + Y^2 = 0$ has $2p + 1$ solutions in total. On the other hand, $(X/Y)^2 =
		-r$ has no solutions, meaning $[0:0:1]$ is the only solution to $X^2 + rY^2 =
		0$.
		
		If $-1$ is a nonresidue, the situation is reversed, and $X^2 + Y^2 = 0$ has one
		solution while $X^2 + rY^2 = 0$ has $2p + 1$ solutions.
		
		\noindent
		\textit{Rank 3: \eqref{case:rank3_x^2+y^2+z^2=0} and \eqref{case:rank3_x^2+y^2+rz^2=0}.} 
		Suppose $-1$ is a residue. If $Z = 0$, then we are in the setting of case \eqref{case:x^2+y^2=0} of
		Proposition \ref{prop:counting-P1-solutions}, which has two
		solutions. Otherwise, we can divide by $Z^2$ and obtain 
		\begin{align*} 
		(X/Z)^2 + (Y/Z)^2 &= -1 \\ 
		(X/Z)^2 + (Y/Z)^2 &= -r 
		\end{align*} 
		respectively. By Proposition \ref{prop:circle-solutions} below, these each have $p - 1$
		solutions. Therefore $X^2 + Y^2 + Z^2 = 0$ and $X^2 + Y^2 + rZ^2 = 0$ each have
		$p+1$ solutions.
		
		Suppose $-1$ is a nonresidue. There are no solutions with $Z=0$, so rewrite the
		equations as above. By Proposition \ref{prop:circle-solutions} again, they each
		have $p + 1$ solutions.
	\end{proof}
	
	\begin{rem*}
		In fact, it is well-known that any nondegenerate conic is isomorphic (as an algebraic variety) to $\P^1$ (see for instance \cite[Prop.~19.3.1]{vakil}). This explains why cases \eqref{case:rank3_x^2+y^2+z^2=0} and \eqref{case:rank3_x^2+y^2+rz^2=0} both have $|\P^1|=p+1$ solutions.
	\end{rem*}
	
	\begin{prop}\label{prop:circle-solutions} 
		For $a \neq 0$, the equation $x^2 + y^2 = a$ has $p - \legendre{-1}{p}$ 
		solutions over $\F_p$.
	\end{prop} 
	\begin{proof} 
		Let $C_a$ denote the (affine) conic $x^2 + y^2 = a$. 
		First we will show the statement for $a=1$.
		
		Let $T$ denote the set of square-roots of $-1$ in $\F_p$. In the case
		that $-1$ is a nonresidue, $T$ is empty. Consider the maps $f\colon C_1(\F_p)
		\setminus (1,0) \to \F_p \setminus T$ and $g\colon \F_p
		\setminus T \to C_1(\F_p) \setminus (1,0)$ defined by 
		\begin{align*} 
		f(x,y) &= \frac{y}{x-1} \\ 
		g(m) &= \left(\frac{m^2 - 1}{m^2 + 1}, \frac{-2m}{m^2 + 1}\right).
		\end{align*} 
		It is easy to check that they are inverse to one
		another, from which it follows that $|C_1(\F_p)| = |\F_p| - |T| + 1 = p -
		\legendre{-1}{p}$ as desired.
		
		Note that if $a,a'$ are both nonzero residues (or both nonresidues)
		then $|C_a(\F_p)| = |C_{a'}(\F_p)|$. From the preceding, we have the claim for all residues
		$a$. Observe that $C_0(\F_p) \cup \cdots \cup C_{p-1}(\F_p)$ gives a partition of
		$\mathbb{A}^2$. Since 
		\[ |C_0(\F_p)| = \begin{cases}
		1 & \text{if } \legendre{-1}{p} = -1 \\
		-1 & \text{if } \legendre{-1}{p} = 1 \\
		\end{cases} \] 
		for a nonresidue $a$ we have 
		\[ |C_a(\F_p)| = \frac{2}{p-1} 
		\left(p^2 - |C_0(\F_p)| - \frac{p-1}{2}|C_1(\F_p)|\right) = 
		p - \legendre{-1}{p}. \qedhere \] 
	\end{proof}
	
	\begin{example}[A zoo of possibilities]
		\label{exa:all-possibilities}
		
		In the table below, the first two columns reflect the discussion
		immediately preceding. The remaining columns consider the number of extraneous
		solutions. The entries marked ``N/A'' are impossible and explanation is given
		afterwards. For all other entries, examples are given for the specific case
		$p=3$.  
		\begin{center}
			\begin{tabular}{c|c|c|c|c|c}  \multicolumn{2}{c|}{} &
				\multicolumn{4}{c}{sol. in $\{Z=0\}$}  \\ \hline
				rank of proj. & sol. in $\mathbb{P}^2$ & $0$ & $1$ & $2$ & $p + 1$\\
				\hline \hline
				$1$ & $p+1$ & \emph{N/A}\textsuperscript{\ref{case:NA(p+1)-0,2}} & $X^2$ & \emph{N/A}\textsuperscript{\ref{case:NA(p+1)-0,2}} & $Z^2$ \\
				\hline
				$2$ & $1$ & $X^2 + Y^2$ & $X^2 + Z^2$ & \emph{N/A} & \emph{N/A} \\
				\hline
				$2$ &	$2p+1$ & \emph{N/A}\textsuperscript{\ref{case:NA(2p+1)-0}} & $X^2 + XZ$ & $XY$ & $XZ$ \\
				\hline
				$3$ & $p+1$ & $X^2 + Y^2 + Z^2$ & $XZ+Y^2$ & $X^2 + 2Y^2 + Z^2$ & \emph{N/A}\textsuperscript{\ref{case:NA(p+1)-(p+1)}} \\
				\hline
			\end{tabular}
		\end{center} 
		\begin{enumerate}
			\item\label{case:NA(p+1)-0,2} When the projectivization has rank 1, it is a double line. If this
			line is distinct from $\{Z = 0\}$, they meet at exactly one point. 
			Otherwise they are the same line, giving $p+1$ solutions in $\{Z=0\}$. 
			\item\label{case:NA(2p+1)-0} When the projectivization has rank 2 and has $2p + 1$ solutions, it
			is the union of two intersecting lines. These lines cannot be disjoint 
			from $\{Z=0\}$.
			\item\label{case:NA(p+1)-(p+1)} There are $p+1$ solutions in $\{Z = 0\}$ only when the 
			coefficients $a,b,c$ are all zero. In particular, such a conic cannot 
			have a projectivization of rank 3.
		\end{enumerate} 
	\end{example}
	
	\section{Counting solutions of a conic in $\A^2$}
	\label{sec:affine-solutions}
	
	We can now use Lemma \ref{lem:relate-solutions-affine-proj} to combine the
	results from the previous two sections into our main theorem.
	
	\begin{thm}[Counting Solutions in $\A^2$]
		\label{thm:main-thm}
		Let $C$ be a conic as in (\ref{eq:original-conic}), and fix a prime $p$.
		
		If $p$ is even, then
		\begin{equation*}
		|C(\F_p)| = 2 + b(-1)^{(a + d)(c + e) + f},
		\end{equation*}
		where $b$ is either $0$ or $1$.
		
		If $p$ is odd, let $\alpha$, $\beta$, and $\gamma$ be as defined in 
		Lemma \ref{thm:diag-P2}, and let $\Delta = b^2-4ac$ be the discriminant. 
		If $\alpha$ and $\beta$ are both zero or
		$\alpha$ is nonzero, we have one of the following cases, in order
		of precedence:
		\begin{enumerate}
			\item All of $a$, $b$ and $c$ vanish, and $C(\F_p)$ is empty.
			\item $\Delta = 0$, and $|C(\F_p)| = p$.
			\item $\Delta \neq 0$, and,
			\[ |C(\F_p)| = p - \legendre{-\Delta}{p}\legendre{-1}{p}. \]
		\end{enumerate}
		If instead only $\alpha$ vanishes, then one of the following holds, again
		in order of precedence:
		\begin{enumerate}
			\item All of $a$, $b$ and $c$ vanish, and $|C(\F_p)|=p$.
			\item If $\Delta = 0$, then
			\[ |C(\F_p)| = p\left(1 + \legendre{\gamma}{p}\legendre{-1}{p}\right). \]
			\item $\Delta \neq 0$, and,
			\[ |C(\F_p)| = p\left(1 + \legendre{\gamma}{p}\legendre{-1}{p}\right)
			- \legendre{\gamma}{p}\legendre{-1}{p}. \]
		\end{enumerate}
	\end{thm}
	\begin{proof} 
		The case $p=2$ is very simple and can be checked by hand. There are very few conics over $\F_2$ ($56$, to be exact), and the solutions to 
		\[ ax^2 + bxy + cy^2 + dx + ey + f = 0 \] 
		are the same as the solutions to 
		\[ bxy + (a+d)x + (c+e)y + f = 0.  \] 
		At this point, one can simply list all of the
		possibilities. It can be checked that the number of solutions in each case agrees with the expression in the theorem statement.
		
		The remainder is a synthesis of \S\ref{sec:diagonalizing-quadratic-forms}, 
		\S\ref{sec:proj-solutions}, and Lemma~\ref{lem:relate-solutions-affine-proj}.
	\end{proof}
	
	\begin{comment}
	Consolidate with Example~4.4!
	\textcolor{red}{Note that if we are in case \eqref{case:0} of Proposition~\ref{prop:counting-P1-solutions}, i.e., if $a=b=c=0$, then $\gamma$ is a residue if and only if $-1$ is. Thus, by Theorem~\ref{thm:diag-P2}, we cannot be in cases \eqref{case:rank2_x^2+y^2=0} or \eqref{case:rank2_x^2+ry^2=0} of Proposition~\ref{prop:counting-P2-solutions} where the projectivized conic has only $1$ solution. Were this the case, Lemma~\ref{lem:relate-solutions-affine-proj} would imply that our affine conic has a negative number (to be exact, $-p$) of solutions!}
	\end{comment}
	
	We remark that while the expression for $|C(\F_p)|$ given above appears complicated, it is easily computable. The only parts that may be non-obvious are the Legendre symbols, but with tools such as the law of quadratic reciprocity, these too are readily determined.
	
	\section{Rational Solutions in $\P^2$}\label{sec:rational-soln}
	Until now, we have been concerned with the number of solutions our conic $C$ has over $\F_p$. Next we will study solutions over the rationals $\Q$. 
	We will only ask about existence, but even so, the classification and analysis of the preceding sections will still be helpful.
	
	Our goal is to lay out a procedure that can determine whether an arbitrary conic has a solution in $\P^2$ over $\Q$. First, by the methods of \S\ref{sec:diagonalizing-quadratic-forms}, we may perform a change of basis so that our conic is given by an equation
	\[
		aX^2 + bY^2 + cZ^2 = 0.
	\]
	If one of these coefficients is zero, say $c$, then we are done: $[0:0:1]$ is a rational solution. So hereafter we assume that $abc\neq 0$. We are able to scale the entire equation by any nonzero constant without altering its solution set, and we are also able to scale individual coefficients by nonzero squares without affecting the existence of solutions. Using these observations, we can assume that $abc$ is square-free, in the sense that $p^2\ndiv abc$ for all $p$.
	
	We make this a bit more precise. Given a prime $p$, a nonzero rational number $q$ can be uniquely expressed as $p^{\nu_p(q)}\frac{m}{n}$ where $\nu_p(q),m,n\in\Z$ and $p\ndiv m,n$. The number $\nu_p(q)$ is the \emph{$p$-adic valuation} of $q$.
	
	Define
	\[
		t_p(a) = \begin{cases}
			1 & \text{if } \nu_p(a)\not\equiv\nu_p(b)\equiv\nu_p(c)\bmod 2,\\
			0 & \text{otherwise,}
		\end{cases}
	\]
	with $t_p(b)$ and $t_p(c)$ defined analogously. Then if we set
	\begin{align*}
		a' = \frac{|a|}{a}\prod_{p \text{ prime}} p^{t_p(a)}
	\end{align*}
	and likewise for $b'$ and $c'$, the conic
	\[
		a'X^2 + b'Y^2 + c'Z^2 = 0
	\]
	is equivalent to our original conic. This new conic has the property that $a',b',c'\in \Z$ and $a'b'c'$ is a nonzero, square-free integer.
	
	Having massaged our conic into this form, the remaining work is taken care of by the following theorem, which is the main result of this section.
	\begin{thm}[Existence of Rational Solutions]\label{thm:rational-solutions}
		Let $a,b,c$ be nonzero integers such that $p^2 \ndiv abc$ for all primes $p$. Then the equation
		\begin{equation}\label{eq:diag-squarefree-conic}
			aX^2 + bY^2 + cZ^2 = 0
		\end{equation}
		has a solution in $\P^2_\Q$ exactly when all of the following conditions are met.
		\begin{enumerate}
			\item\label{cond:real-soln} At least one of $a,b,c$ is positive, and at least one is negative.
			\item\label{cond:Qp-soln} For all primes $p>2$ dividing $c$, $\legendre{-ab}{p} = 1$. The analogous conditions for primes $p>2$ dividing $a$ or $b$ are required as well.
			\item\label{cond:Q2-soln} We can perform a $2$-adic lift of this conic; that
			is, the coresponding conic in Table \ref{table:Q2-diag-conics} has a solution 
			in $\P^2_{\Q_2}$.\footnote{This condition will be explained in Lemma 
			\ref{lem:squares-in-Q2}. Unfortunately, the case $p=2$ is pathological,
			and requires special treatment.}
		\end{enumerate}
	\end{thm}
	We prove this theorem through a sequence of lemmas. In particular, we will verify
	existence of solutions in $\R$, the $p$-adic numbers $\Q_p$ for $p$ odd, and $\Q_2$.
	We can then apply the Hasse--Minkowski theorem, which states that a homogeneous quadratic
	has a solution in $\P^n_\Q$ if and only if it has a solution in $\P^n_K$ for 
	every completion $K$ of $\Q$, namely, $\R$ and $\Q_p$ for $p$ prime.
	We will use the case $n=2$. As noted above, the case $K = \Q_2$ is treated separately, 
	for technical reasons.
	
	The case $K=\R$ is easy. A necessary and sufficient condition for a solution 
	to \eqref{eq:diag-squarefree-conic} in $\P^2_\R$ is given by 
	\eqref{cond:real-soln} above.
	
	The remaining completions are the fields of $p$-adic numbers $\Q_p$, whose definition 
	we review now.
	\begin{defn}
		The ring of \emph{$p$-adic integers} $\Z_p$ is the inverse limit of
		\[
		\cdots \to \Z/p^3 \to \Z/p^2 \to \Z/p.
		\]
	\end{defn}
	In other words, elements of $\Z_p$ are sequences $(a_1,a_2,a_3,\ldots)$ where $a_i \in \Z/p^i$ and $a_{i+1} \equiv a_i\bmod {p^i}$.
	
	Such an element can also be interpreted as the series
	\[
	b_0 + b_1 p + b_2 p^2 + b_3 p^3 + \cdots
	\]
	where $b_0 = a_1$ and $b_i = (a_{i+1} - a_i)/p^i$ for $i \geq 1$. Addition and multiplication are done as expected with ``rightwards carrying.''
	
	Since $\Z_p$ is an integral domain, we are justified in taking its fraction field:
	\begin{defn}
		The field of \emph{$p$-adic numbers} $\Q_p$ is the fraction field of $\Z_p$.
	\end{defn}
	Since the units in $\Z_p$ are the elements with $b_0 \neq 0$, constructing $\Q_p$ amounts to inverting $p$ in $\Z_p$. Thus elements of $\Q_p$ can also be interpreted as series, though they are allowed to start from negative powers of $p$.
	
	We refer the reader to \cite{henselMO} for a proof of the following formulation of Hensel's lemma. \textcolor{blue}{[still need to do something about this reference maybe]}
	\begin{lem}[Multivariate Hensel's lemma]\label{thm:hensels-one-eq}
		Let $k$ be a nonnegative integer. Let $f \in \Z_p[x_1,\ldots,x_n]$ be a polynomial in $n$ variables and suppose $a = (a_1,\ldots,a_n)\in \Z_p^n$ satisfies $f(a) = 0 \bmod {p^{2k+1}}$ and $\frac{\partial f}{\partial x_i}(\gamma) \not\equiv 0 \bmod {p^{k+1}}$ for some $i\in \{1,\ldots,n\}$. Then there exists $\widetilde{a} \in \Z_p^n$ such that, for all $i$, $\widetilde{a}_i\equiv a_i \bmod {p^{2k+1}}$ and $f(\widetilde{a}) = 0$.
	\end{lem}
	To determine whether \eqref{eq:diag-squarefree-conic} has a solution in $\P^2_{\Q_p}$, we will make extensive use of this result. Note that, because the equation is homogeneous, existence of a solution over $\Q_p$ is the same as existence of a solution over $\Z_p$.
	
	The following two lemmas address existence of solutions in $\Q_p$ for odd $p$.
	\begin{lem}
		With the setup of Theorem \ref{thm:rational-solutions}, if $p > 2$ is a prime that does not divide $abc$, then \eqref{eq:diag-squarefree-conic} has a solution in $\P^2_{\Q_p}$.
	\end{lem}
	\begin{proof}
		We will apply Hensel's lemma with $k=0$. By Proposition \ref{prop:counting-P2-solutions}, \eqref{eq:diag-squarefree-conic} has $p+1$ solutions in $\P^2_{\F_p}$. The condition on partial derivatives is automatically satisfied: for any $[X:Y:Z]\in \P^2$ one of $X,Y,Z$ is nonzero, and thus one of $2aX, 2bY, 2cZ$ is as well. Hence any of the solutions over $\F_p$ can be lifted via Hensel's lemma to a solution over $\Q_p$.
	\end{proof}

    Otherwise, if $p\mid abc$, then $p$ divides either $a$, $b$, or $c$. We treat the last case; the other two cases are analogous.

	\begin{lem}
		With the setup of Theorem \ref{thm:rational-solutions}, suppose $p > 2$ is a prime dividing $c$. Then \eqref{eq:diag-squarefree-conic} has a solution in $\P^2_{\Q_p}$ if and only if $\legendre{-ab}{p}=1$ (condition \eqref{cond:Qp-soln} of Theorem \ref{thm:rational-solutions}).
	\end{lem}
	\begin{proof}
		When we reduce mod $p$, \eqref{eq:diag-squarefree-conic} becomes
		\begin{equation}\label{eq:p|abc-modp}
			aX^2 + bY^2 \equiv 0 \bmod p
		\end{equation}
		If $\legendre{-ab}{p}=1$, then take $s\in \F_p$ such that $s^2 = -ab$, and $[s:a:0]$ is a solution satisfying the hypotheses of Hensel's lemma with $k=0$. Thus \eqref{eq:diag-squarefree-conic} has a solution over $\Z_p$ in this case.
		
		Suppose instead that $\legendre{-ab}{p}=-1$. For the sake of contradiction, assume a solution $[X:Y:Z]$ existed over $\Q_p$. By scaling as needed, we can assume that $X,Y,Z\in \Z_p$ and not all of them are divisible by $p$. Since $-ab$ is a nonresidue, \eqref{eq:p|abc-modp} entails $X,Y \equiv 0 \bmod p$ (c.f. \S\ref{sec:proj-solutions}). Then we can write
		\[
			ap^2 (X/p)^2 + bp^2 (Y/p)^2 + cZ^2 = 0
		\]
		but we assumed that $p^2\ndiv c$, so this implies $p|Z$ and we have a contradiction.
	\end{proof}
	This leaves only the case $K = \Q_2$ to consider. The situation is more contrived here,
	 and it is responsible for the contrived final condition of Theorem 
	 \ref{thm:rational-solutions}. We'll need the following characterization of squares
	 in $\Q_2$.
	
	\begin{lem}\label{lem:squares-in-Q2}
		Let
		\[
			a = 2^{2m + i}(1 + j\cdot 2 + k \cdot 4+ \cdots)
		\]
		where $m$ is an integer and each of $i,j,k$ is either 0 or 1. Define
		\[
			\bar{a} = 2^i 3^j 5^k.
		\]
		Then $a/\bar{a}$ is a square in $\Q_2$.
	\end{lem}
	\begin{proof}
		We claim that if $s\in \Z_2$ is congruent to $1\bmod 8$, then it is a square. Consider the equation $x^2 - s=0$. Applying Hensel's lemma with $k=1$ to the solution $x = 1$ shows that it can be lifted to a solution in $\Z_2$ as desired.
		By checking each case for $j$ and $k$, one sees that $2^{-2m} a/\bar{a} \equiv 1 \bmod 8$; it follows that $a/\bar{a}$ is a square.
	\end{proof}
	Using this lemma and the reasoning at the start of this section, we can scale \eqref{eq:diag-squarefree-conic} over $\Q_2$ so that $abc$ is square free. The result will be one of the equations listed in Table \ref{table:Q2-diag-conics}. Many of these have solutions mod 8 which can be lifted using Hensel's lemma (where we use $k=1$). The other ones do not, and the claim is that they have no solutions over $\Q_2$.
	
	We prove this claim for the equation $X^2 + Y^2 + Z^2 = 0$ and omit the rest (which are similar). Suppose a solution $[X:Y:Z]$ existed over $\Q_2$. We can assume that $X,Y,Z\in \Z_2$ and that not all of them are divisible by 2. However, 1 is the only odd residue mod 8, from which it is easy to check that $X^2 + Y^2 + Z^2 = 0$ has no such solutions mod 8 (let alone such a solution over $\Z_2$).
	\begin{table}[h]
		\begin{tabular}{r@{\hskip 18pt}l}
            \toprule
			Equation & Solution in $\P^2_{\Q_2}$\\
            \midrule
			$X^2 + Y^2 + Z^2 = 0$ & No \\ 
			$2X^2 + Y^2 + Z^2 = 0$ & No \\ 
			$3X^2 + Y^2 + Z^2 = 0$ & Yes, lift $[1:1:2]$ \\ 
			$5X^2 + Y^2 + Z^2 = 0$ & No \\ 
			$6X^2 + Y^2 + Z^2 = 0$ & Yes, lift $[1:1:1]$ \\ 
			$10X^2 + Y^2 + Z^2 = 0$ & No \\ 
			$15X^2 + Y^2 + Z^2 = 0$ & Yes, lift $[1:1:0]$ \\ 
			$30X^2 + Y^2 + Z^2 = 0$ & Yes, lift $[1:1:1]$ \\ 
			$3X^2 + 2Y^2 + Z^2 = 0$ & No \\
			$5X^2 + 2Y^2 + Z^2 = 0$ & Yes, lift $[1:1:1]$ \\ 
			$5X^2 + 3Y^2 + Z^2 = 0$ & Yes, lift $[1:1:0]$ \\ 
			$6X^2 + 5Y^2 + Z^2 = 0$ & No \\ 
			$10X^2 + 3Y^2 + Z^2 = 0$ & No \\ 
			$15X^2 + 2Y^2 + Z^2 = 0$ & Yes, lift $[1:0:1]$ \\ 
			$5X^2 + 3Y^2 + 2Z^2 = 0$ & Yes, lift $[1:1:0]$ \\ 
			\bottomrule
			\multicolumn{2}{c}{}
		\end{tabular}
		
		\caption{An exhaustive list of representative conics over $\Q_2$. Here, ``lift'' means to apply Hensel's lemma with $k=1$ to the specified mod 8 solution.}
		 \label{table:Q2-diag-conics}
	\end{table}

	With the question of $p=2$ settled, this concludes the proof of Theorem \ref{thm:rational-solutions}. We comment that the powerful machinery of Hilbert symbols can be used to deduce the same result, and we leave this to the curious reader to verify.
	\begin{filecontents}{references.bib}
		@book {omeara,
			AUTHOR = {O'Meara, O. Timothy},
			TITLE = {\href{https://link.springer.com/book/10.1007\%2F978-3-642-62031-7}{Introduction to quadratic forms}},
			SERIES = {Classics in Mathematics},
			NOTE = {Reprint of the 1973 edition},
			PUBLISHER = {Springer-Verlag, Berlin},
			YEAR = {2000},
			PAGES = {xiv+342},
			ISBN = {3-540-66564-1},
			MRCLASS = {11Exx},
			MRNUMBER = {1754311},
		}
		@article{magma,
			author={Wieb Bosma and John Cannon and Catherine Playoust},
			title={\href{http://www.sciencedirect.com/science/article/pii/S074771719690125X}{The Magma algebra system {I}: The user language}},
			journal={Journal of Symbolic Computation},
			volume={24},
			year={1997},
			pages={235--265}
		}
		@misc{henselMO,
			author={Will Sawin},
			title={Answer to \href{https://mathoverflow.net/questions/108687}{\texttt{https://mathoverflow.net/questions/108687}}}
		}
		
		@book{vakil,
			title={\href{http://math.stanford.edu/~vakil/216blog/index.html}{Foundations of Algebraic Geometry}},
			author={Ravi Vakil},
			year={December 29, 2015}
		}
	\end{filecontents}
	
	\bibliographystyle{plain}
	\bibliography{references}
	
\end{document}

%%% Local Variables:
%%% mode: latex
%%% TeX-master: t
%%% End:
