%Document setup
\documentclass[10pt,a4paper]{amsart} 

\usepackage[latin1]{inputenc}
\usepackage{mathtools} 
\usepackage{amssymb} 
\usepackage{mathrsfs} % gives mathscr font 
\usepackage{graphicx}
%\usepackage[headsep=0.15in, left=0.5in, right=0.5in, top=0.6in, bottom=0.5in]{geometry} 
%\usepackage[textlf,mathlf]{MinionPro} 
\usepackage{fancyhdr}
%\usepackage{xypic}
\usepackage{tikz-cd} 
\usepackage{todonotes}
\usepackage{verbatim}
\usepackage{filecontents}
\usepackage[colorlinks=true]{hyperref}
\usepackage{tabu}

%Environments
\numberwithin{equation}{section} 
\numberwithin{figure}{section}
\theoremstyle{definition} 
\theoremstyle{plain} 
\newtheorem{thm}{\protect\theoremname}[section]
\newtheorem{example}[thm]{\protect\examplename}
\theoremstyle{remark} 
\newtheorem*{rem*}{\protect\remarkname}
\theoremstyle{plain}
\newtheorem{cor}[thm]{\protect\corollaryname}
\theoremstyle{definition} 
\newtheorem{defn}[thm]{\protect\definitionname}
\theoremstyle{plain} 
\newtheorem{prop}[thm]{\protect\propositionname}
\theoremstyle{plain} 
\newtheorem{lem}[thm]{\protect\lemmaname}

\providecommand{\definitionname}{Definition}
\providecommand{\examplename}{Example} 
\providecommand{\lemmaname}{Lemma}
\providecommand{\propositionname}{Proposition}
\providecommand{\remarkname}{Remark} 
\providecommand{\corollaryname}{Corollary}
\providecommand{\theoremname}{Theorem}

%macros, etc.
\newcommand{\legendre}[2]{\genfrac{(}{)}{}{}{#1}{#2}}
\renewcommand{\arraystretch}{1.2} 

\newcommand{\A}{\mathbb{A}}
\renewcommand{\P}{\mathbb{P}} 
\newcommand{\F}{\mathbb{F}}
\newcommand{\Z}{\mathbb{Z}}
\newcommand{\Q}{\mathbb{Q}}
\newcommand{\R}{\mathbb{R}}
\newcommand{\GL}{\operatorname{GL}}
\newcommand{\Span}{\operatorname{Span}}
\newcommand{\trsp}[1]{{{#1}^{\top}\!\!}}

\begin{document} 

\title{Mod $p$ solutions to conics, and applications} 
\author{Xianglong Ni, Oron Propp, and Miguel Young}
\maketitle

\begin{abstract} 
	We count the number of solutions of any affine plane conic modulo a prime.
	We use this result to determine whether such a conic has rational solutions. 
\end{abstract}

\tableofcontents

\section{Introduction}\label{sec:intro} 

	% Get a reference for general nonsense over the reals
	A \emph{conic section} on the real plane $\R^2$ is the solution set of 
	a polynomial
	\[ C: ax^2 + bxy + cy^2 + dx + ey + f = 0, \]
	where the degree-2 coefficients $a, b, c$ are not all zero. This conic can be \emph{projectivized} by adding an additional
	unknown to make the polynomial homogeneous:
	\[ \widetilde{C}: aX^2 + bXY + cY^2 + dXZ + eXZ + fZ^2 = 0. \]
	This gives a quadratic form on $\R^3$; when this form is degenerate, we
	say that the original conic, $C$, is also degenerate.
	
	The homogeneous part of $C$, $ax^2 + bxy + cy^2$, gives a quadratic form
	on $\R^2$, with matrix
	\[ \begin{bmatrix}
	a & b/2 \\
	b/2 & c
	\end{bmatrix}. \]
	The determinant of this matrix, called the \emph{discriminant} and
	denoted $-4\Delta = b^2 - 4ac$, gives a characterization of the shape of $C$
	when it is nondegenerate. Positive, zero, and negative discriminant
	correspond to $C$ being an ellipse, a parabola, or a hyperbola. The
	projectivized conic also gives rise to a quadratic form; we say $C$ is
	degenerate when the matrix of $\widetilde{C}$ is singular. The curve
	will instead look like a pair of parallel or intersecting line, a double
	line (where each point of the line is a solution of the conic and its 
	derivative), a single point, or no points at all.
	
	The above discussion of the \emph{real} points of $C$ is fairly classical,
	and a full classification is well-understood. We can instead ask that solutions
	be in some other field $k$.
	The situation when $k=\Q$ turns out to be quite complicated. It is
	characterized by the Hasse-Minkowski theorem, which we quote a special
	case of.
	\begin{thm}[Hasse-Minkowski]
		A quadratic form $C$ over $\Q$ has a solution if and only if it h
		as solutions
		over all the completions of $\Q$, namely the reals, $\R$, and the $p$-adic
		numbers, $\Q_p$, for all primes $p$.
	\end{thm}
	The case when $k = \R$ is, as mentioned, classical. To solve over $\Q_p$,
	we can simply ask for solutions in $\Z_p$, the $p$-adic integers. We
	can in turn get these solutions through Hensel's lemma, which allows us
	to lift solutions over the finite field $\F_p$ to $\Z_p$. Our ultimate
	goal is to give a condition for rational points of a conic $C$, by lifting
	a condition on $\F_p$-points.
	
	This paper is roughly divided into two parts. The first part
	is bootstrapping this process by counting solutions over $\F_p$. To do this,
	we study the projectivization of our chosen conic, discussed in
	\S\ref{sec:projectivization-of-the-problem}. We then diagonalize 
	the projective curve through a procedure described in 
	\S\ref{sec:diagonalizing-quadratic-forms}. We then work out several cases
	of diagonalized curves in \S\ref{sec:proj-solutions}, and finally
	drop back down to the affine case in \S\ref{sec:affine-solutions}. Our result
	is a complete characterization of the number of solutions, given by 
	Theorem \ref{thm:main-thm}.
	In the second part, we can lift solutions to $\Z_p$ by way of
	Hensel's lemma in \S\ref{sec:hensels-lemma}. Finally, using Hasse-Minkowski 
	we give a criterion for  existence of rational points in  
	\S\ref{sec:hasse-minkowski}.

	\subsection*{Acknowledgements} Xianglong wrote sections \S\ref{sec:projectivization-of-the-problem}, \ref{sec:proj-solutions}, \ref{sec:hensels-lemma}. Oron wrote sections \S\ref{sec:diagonalizing-quadratic-forms}, \ref{sec:affine-solutions} and helped with \S\ref{sec:proj-solutions} (Proposition \ref{prop:circle-solutions} especially). Miguel ran experimental tests to both motivate and verify our computations, wrote \S\ref{sec:intro}, and helped format, edit, and proofread.
	
	We would like to thank the 18.821 course staff for organizing this project, and Rasmus Johansen in particular for his guidance and suggestions.    
    \section{Preliminaries}\label{sec:projectivization-of-the-problem}

    Let us begin by fixing a conic
    \begin{equation}\label{eq:original-conic} 
        C \colon ax^2 + bxy + cy^2 + dx + ey + f = 0 
    \end{equation} 
    with integer coefficients. We require that at least one of 
    $a$, $b$, and $c$ is nonzero, and that $\gcd(a,b,c,d,e,f)=1$.
    Fix a base field $\F_p$ where $p$ is an odd prime. (We defer treatment of the case $p = 2$ to \S\ref{sec:affine-solutions}.)
    \begin{defn}
        Let $\A^n_{\F_p}$ denote $n$-dimensional affine space over $\F_p$. Define 
        $n$-dimensional projective space over $\F_p$ as the quotient
        \[ \P^n_{\F_p} = (\A^{n+1}_{\F_p}\setminus \{0\})/{\sim}, \]
        where we declare that $(x_1,\ldots,x_{n+1}) \sim 
        (\lambda x_1,\ldots,\lambda x_{n+1})$ for $\lambda$ a nonzero scalar. We
        will primarialy work with $n = 2$, and suppress the ground field from the 
        notation. We will write $[X:Y:Z]\in \P^2$ to denote the equivalence 
        class of $(X,Y,Z)\in \A^3 \setminus \{0\}$.
    \end{defn}

    With this construction in mind, we can \emph{projectivize} our conic
    \eqref{eq:original-conic} from $\A^2$ to $\P^2$, by writing it as a 
    homogenous equation: 
    \begin{equation}\label{eq:projective-conic} 
        \widetilde{C} \colon aX^2 + bXY + cY^2 + dXZ + eYZ + fZ^2 = 0. 
    \end{equation}
    Note that homogeneity ensures it is well-defined to ask when this
    equation has a solution in $\P^2$.

    Let $C(\F_p)$ denote the solution set of \eqref{eq:original-conic} in $\A^2$. Our aim
    is to characterize $|C(\F_p)|$ as a function of the coefficients $a,b,c,d,e,f$ and 
    the prime $p$. The solutions of \eqref{eq:projective-conic} define a set
    $\widetilde{C}(\F_p) \subset \P^2$.

    There is an evident injection $C(\F_p) \rightarrowtail \widetilde{C}(\F_p)$ sending $(x,y)$ to
    $[x:y:1]$. However, elements of $\widetilde{C}(\F_p)$ of the form $[X:Y:0]$ do not 
    correspond to elements of $C(\F_p)$. 
    These are solutions ``at infinity'' in the projective plane.
    Thankfully, these extraneous solutions are easily characterized: they are the
    solutions to the equation
    \begin{equation}\label{eq:projective-conic-at-infinity} 
        \widetilde{C}_0 \colon aX^2 + bXY + cY^2 = 0
    \end{equation} 
    on the projective line $\P^1$. Call this solution set $\widetilde{C}_0(\F_p)\subset \P^1$.  

    \begin{lem}\label{lem:relate-solutions-affine-proj}
        $|\widetilde{C}(\F_p)| = |C(\F_p)| + |\widetilde{C}_0(\F_p)|$.
    \end{lem} 
    \begin{proof}
        A solution $[X:Y:Z]$ of $\widetilde{C}$ corresponds 
        to a solution $(X/Z,Y/Z)$ of $C$ when $Z \neq 0$,
        and to a solution $[X:Y]$ of $\widetilde{C}_0$ 
        when $Z = 0$.  
    \end{proof}
    
    We will also need a bit of number theory, since our characterization
    involves the notion of quadratic residues:
    \begin{defn}
        The nonzero elements $\F_p^\times \subset \F_p$ form an abelian
        group under multiplication, and the squaring map 
	    \[ \F^\times_p \xrightarrow{x \mapsto x^2} \F^\times_p \]
	    is a homomorphism. We will call the image of this map the
	    \emph{quadratic residues mod $p$}, or just the \emph{residues} (where the
	    prime $p$ is inferred from context). We will also regard $0\in\F_p$ as a residue.
    \end{defn}
    
    If $p$ divides $x^2 - 1 = (x+1)(x-1)$, we must have $x \equiv \pm 1 \pmod{p}$, so 
    the kernel of the map is $\{\pm 1\}$. Thus each nonzero residue has exactly two 
    square roots. The cokernel of this map is also a group of order two: 
	\[ \F_p^\times / (\F_p^\times)^2 \cong \{\pm 1\}.  \]
	This gives a convenient characterization of quadratic residues:
	\begin{defn}
		The \emph{Legendre symbol} is the cokernel map    
		\[ \legendre{\cdot}{p}: 
		\F_p^\times \to \F_p^\times / (\F_p^\times)^2 \cong \{\pm 1\}, \]
		where $\legendre{a}{p}$ is $-1$ if $a$ is a nonresidue,
		and $1$ if $a$ is a nonzero residue. This map extends to a multiplication-respecting map
		\[ \F_p \to \{-1,0,1\} \]
		with $\legendre{0}{p}$ defined as $0$.
	\end{defn}
	
    \section{Diagonalization of quadratic forms}
    \label{sec:diagonalizing-quadratic-forms} 

    In this section, we aim to simplify the equation defining the projectivized conic $\widetilde{C}$ via coordinate-changes of $\P^2$. In this section,
    we will make extensive use of \texttt{Magma} \cite{magma}, a computer 
    algebra system. Let
	\begin{equation*}
	A=\begin{bmatrix}
	a&b/2&d/2\\
	b/2&c&e/2\\
	d/2&e/2&f
	\end{bmatrix}, \hspace{1cm}
	v = \begin{bmatrix}
	X\\
	Y\\
	Z
	\end{bmatrix}.
	\end{equation*}
	The quadratic form $\trsp{v} A v$ is precisely the equation for $\widetilde{C}$. In this sense, the matrix $A$ represents $\widetilde{C}$ in the coordinates $[X,Y,Z]$.
	
	Now, any $P\in\GL_3(\F_p)$ gives rise to a change of coordinates $v\mapsto Pv$ in $\P^2$. Since
	\begin{equation*}
	\trsp{(Pv)} A(Pv)=\trsp{v}(\trsp{P} AP)v,
	\end{equation*}
	the matrix representing $\widetilde{C}$ is $\trsp{P}AP$ in the new coordinates. This transformation $A \mapsto \trsp{P}A P$ does not change the number of solutions of the corresponding conic, as $P$ is invertible. The following result is well-known (see for instance \cite[Prop.~42:1]{omeara}):
	\begin{thm}
	\label{thm:diag}
	There exists some $P\in\GL_3(\F_p)$ such that $\trsp{P}AP$ is diagonal.
	\end{thm}
	This significantly simplifies the set of conics which we must consider: a diagonal matrix
	\begin{equation*}
	\begin{bmatrix}
	\lambda_1&0&0\\
	0&\lambda_2&0\\
	0&0&\lambda_3
	\end{bmatrix}
	\end{equation*}
	corresponds to the conic defined by $\lambda_1x^2+\lambda_2y^2+\lambda_3z^2=0$ (note that we cannot have $\lambda_1=\lambda_2=\lambda_3=0$ by our initial assumptions). By further scaling and permuting coordinates as appropriate, we immediately obtain the following:
	\begin{cor}
	\label{cor:sixcases}
	Fix an element $r\in\F_p$ that is not a square. There exists some $P\in\GL_3(\F_p)$ such that $\trsp{P}AP$ is equal to a (nonzero) scalar multiple of one of the following:
	$$
	\begin{tabu}{cccccc}
	(1)&(2)&(3)&(4)&(5)\\
	\begin{bmatrix}
	1&0&0\\
	0&0&0\\
	0&0&0
	\end{bmatrix}&
	\begin{bmatrix}
	1&0&0\\
	0&1&0\\
	0&0&0
	\end{bmatrix}&
	\begin{bmatrix}
	1&0&0\\
	0&r&0\\
	0&0&0
	\end{bmatrix}&
	\begin{bmatrix}
	1&0&0\\
	0&1&0\\
	0&0&1
	\end{bmatrix}&
	\begin{bmatrix}
	1&0&0\\
	0&1&0\\
	0&0&r
	\end{bmatrix}
	\end{tabu}
	$$
	\end{cor}
	In the next section, we determine how many solutions each of the corresponding conics has in $\P^2$. For now, we determine to which of these five matrices $A$ corresponds. Note that we do not need to distinguish between scalar multiples of these matrices as the corresponding conics have equivalent solution sets.
	
	\begin{lem}
	\label{thm:diag-P2}
	   In the context of Corollary \ref{cor:sixcases}, let
	   \begin{align*}
	       \alpha &= 4acf - ae^2 - b^2f + bde - cd^2,\\
	       \beta &= 4ac + 4af + 4cf - b^2 - d^2 - e^2, \\
	       \gamma &= 4ac - b^2,
	   \end{align*} 
	   where the first two expressions are coefficients in the 
	   characteristic polynomial of $A$. 
	   Then, the matrix $A$ corresponds to one of the cases in 
	   Corollary \ref{cor:sixcases} when
	   \begin{enumerate}
	       \item\label{case:1} $\alpha=\beta=0$;
	       \item\label{case:11} $\alpha=0$, $\beta\ne 0$, 
	           and $\gamma$ is a nonzero residue;
	       \item\label{case:1r} $\alpha=0$, $\beta\ne 0$, 
	           and $\gamma$ is a nonresidue;
	       \item\label{case:111} $\alpha$ is a nonzero residue;
	       \item\label{case:11r} $\alpha$ is a nonresidue.
	   \end{enumerate}
	\end{lem}
	\begin{proof}
	Let $P\in\GL_3(\F_p)$ be such that $P\trsp{P}AP$ is one of the six matrices in Corollary~\ref{cor:sixcases}; these six matrices are distinguished by their ranks and the products of their nonzero eigenvalues. Since $P$ is invertible, the rank of $\trsp{P}AP$ is equal to that of $A$. By the rank-nullity theorem, the rank of $A$ is given by 3 minus the multiplicity of the eigenvalue $0$: this multiplicity is the highest power of $t$ dividing the characteristic polynomial of $A$. A \texttt{Magma} computation shows that the characteristic polynomial of $A$ is given by
	\begin{equation*}
	t^3+(-a-c-f)t^2+\tfrac{1}{4}(4ac+4af-b^2+4cf-d^2-e^2)t+\tfrac{1}{4}(-4acf+ae^2+b^2f-bde+cd^2).
	\end{equation*}
	Thus $A$ has rank $1$ when $\alpha=\beta=0$, which gives \eqref{case:1}.
	
	Likewise, $A$ has rank $2$ when $\alpha=0$ and $\beta\ne 0$. Since $\beta$ is the sum of $4ac-b^2$, $4af-d^2$, and $4cf-e^2$, these cannot all be zero; the latter two are obtained from $4ac-b^2$ by permuting the coordinates of $A$, which justifies the computation of $\gamma$. Thus, assume that $\gamma$ has been determined in this manner. If $a\ne 0$, then the \texttt{Magma} method \texttt{DiagonalForm()} shows that $A$ diagonalizes via
	\begin{equation*}
		\begin{bmatrix}
			1 & -\dfrac{b}{2a} & \dfrac{be - 2cd}{4ac - b^2}\\
			0 & 1 & \dfrac{-2ae + bd}{4ac - b^2} \\
			0 & 0 & 1
		\end{bmatrix}^\top\!
		A
		\begin{bmatrix}
			1 & -\dfrac{b}{2a} & \dfrac{be - 2cd}{4ac - b^2}\\
			0 & 1 & \dfrac{-2ae + bd}{4ac - b^2} \\
			0 & 0 & 1
		\end{bmatrix}
		=
		\begin{bmatrix}
		a&0&0\\
		0&\dfrac{4ac-b^2}{4a}&0\\
		0&0&0
		\end{bmatrix}.
	\end{equation*}
	Thus, the product of the nonzero eigenvalues of this matrix is $\gamma=4ac-b^2$, up to a square. The case $c\ne 0$ is similar.
	
	Next, if $a=c=0$ and $f\ne 0$, then a similar \texttt{Magma} computation shows that $A$ diagonalizes via
	\begin{equation*}
		\begin{bmatrix}
			0 & 0 & 1\\
			0 & 1 & \dfrac{2bf - de}{e^2}\\
			1 & -\dfrac{e}{2f} & -\dfrac{b}{e}
		\end{bmatrix}^\top\!
		A
		\begin{bmatrix}
			0 & 0 & 1\\
			0 & 1 & \dfrac{2bf - de}{e^2}\\
			1 & -\dfrac{e}{2f} & -\dfrac{b}{e}
		\end{bmatrix}
		=
		\begin{bmatrix}
		f&0&0\\
		0&-\dfrac{e^2}{4f}&0\\
		0&0&0
		\end{bmatrix}.
	\end{equation*}
	The product of the nonzero eigenvalues of this matrix is $\gamma=-1$, up to a square. Note that in this case the diagonalizing matrix $P$ is only well-defined if $e$ is nonzero as well, but this is clear as the condition $\alpha=0$ is equivalent to $bf=de$, and $b\ne 0$ as $4ac-b^2\ne 0$ (thus, $4ac-b^2=-b^2$ is also $-1$ up to a square). Finally, if $a=c=f=0$, then either $d=0$ or $e=0$, and the same \texttt{Magma} computation shows
	\begin{align*}
		\begin{bmatrix}
			1 & -1/2 & -e/b\\
			1 & 1/2 & 0\\
			0 & 0 & 1
		\end{bmatrix}^\top\!
		A
		\begin{bmatrix}
			1 & -1/2 & -e/b\\
			1 & 1/2 & 0\\
			0 & 0 & 1
		\end{bmatrix}
		&=
		\begin{bmatrix}
			b&0&0\\
			0&-b/4&0\\
			0&0&0
		\end{bmatrix}\\
		\begin{bmatrix}
			1 & -1/2 & 0\\
			1 & 1/2 & -d/b\\
			0 & 0 & 1
		\end{bmatrix}^\top\!
		A
		\begin{bmatrix}
			1 & -1/2 & 0\\
			1 & 1/2 & -d/b\\
			0 & 0 & 1
		\end{bmatrix}
		&=
		\begin{bmatrix}
			b&0&0\\
			0&-b/4&0\\
			0&0&0
		\end{bmatrix}.
	\end{align*}
	The product of the nonzero eigenvalues of this matrix is again $\gamma=-1$, up to a square. Since the matrices \eqref{case:11} and \eqref{case:1r} are obtained from these diagonalized forms of $A$ by scaling coordinates, it is clear that $A$ corresponds to \eqref{case:11} if $\gamma$ is a residue and to \eqref{case:1r} if it is not.
	
	Lastly, $A$ has rank $3$ when $\alpha$ is nonzero. Since
	\begin{equation*}
	\det \trsp{P}AP=\det \trsp{P} \cdot\det A\cdot\det P=\det A\cdot(\det P)^2,
	\end{equation*}
	the determinant of $A$ is a quadratic residue if and only if that of $\trsp{P}AP$ is. Cases \eqref{case:111} and \eqref{case:11r} of the theorem follow.
	\end{proof}
	
	We will also need an analogous result for quadratic forms $ax^2+bxy+cy^2$ in two variables, which follows immediately from the lemma by setting $d=e=f=0$:
	\begin{cor}\label{cor:diag-P1}
	The matrix $\left[\begin{smallmatrix}a&b/2\\b/2&c\end{smallmatrix}\right]$ diagonalizes (as in Corollary~\ref{cor:sixcases}, though with $P \in \GL_2(\F_p)$) to
	\begin{itemize}
	\item $\left[\begin{smallmatrix}0&0\\0&0\end{smallmatrix}\right]$ if $a=b=c=0$;
	\item $\left[\begin{smallmatrix}1&0\\0&0\end{smallmatrix}\right]$ or $\left[\begin{smallmatrix}r&0\\0&0\end{smallmatrix}\right]$ if $4ac-b^2=0$;
	\item $\left[\begin{smallmatrix}1&0\\0&1\end{smallmatrix}\right]$ if $4ac-b^2$ is a nonzero quadratic residue;
	\item $\left[\begin{smallmatrix}1&0\\0&r\end{smallmatrix}\right]$ if $4ac-b^2$ is a nonzero quadratic nonresidue.
	\end{itemize}
	\end{cor}
	
	\begin{rem*}
		The discussion in this section suggests a definition for a \emph{degenerate} conic over $\F_p$: they should be the conics whose corresponding quadratic form is degenerate, or equivalently, whose corresponding matrix is singular.
	\end{rem*}

    \section{Counting solutions of a conic in $\P^2$}
    \label{sec:proj-solutions}
    
    Now that we have diagonalized our conics to a more easily-analyzed form, we can
    work on solving the projectivized conics, $\widetilde{C}$ and 
    $\widetilde{C}_0$. We will treat the easier $\P^1$ case first.
    \begin{prop}[Counting Solutions in $\P^1$]
        \label{prop:counting-P1-solutions}
        Fix a prime $p$ and let $r$ be a nonresidue. When diagonalized,
        $\widetilde{C}_0$ has one of the following forms:
        \begin{enumerate}
            \item \label{case:0} $0 = 0$, with $p+1$ solutions.
            \item $X^2 = 0$, with $1$ solution, namely $[0:1]$.
            \item \label{case:x^2+y^2=0} $X^2 + Y^2 = 0$, with 
            $1 + \legendre{-1}{p}$ solutions.
            \item $X^2 + rY^2 = 0$, with 
            $1 - \legendre{-1}{p}$ solutions.
        \end{enumerate}    
    \end{prop}
	\begin{proof}
        The first and second cases are clear; note that $|\P^1| = p+1$.
        
        In the other two cases, $Y$ must be nonzero. We can write them as
		$(X/Y)^2 = -1$ and $(X/Y)^2 = -r$ respectively. If $-1$ is a residue,
		then $-r$ is not. It follows that $X^2 + Y^2 = 0$ has two solutions and $X^2 +
		rY^2 = 0$ has none. If $-1$ is a nonresidue, the situation is reversed: $X^2 +
		Y^2 = 0$ has no solutions and $X^2 + rY^2 = 0$ has two solutions.
	\end{proof}
	
	Now, let us treat the $\P^2$ case.
	
	\begin{prop}[Counting Solutions in $\P^2$]
	   \label{prop:counting-P2-solutions}
	   Fix a prime and let $r$ be a nonresidue. When diagonalized $\widetilde{C}$ 
	   has one of the following forms:
		\begin{enumerate} 
			\item\label{case:rank1_x^2=0} $X^2 = 0$, with $p+1$ solutions.
			\item\label{case:rank2_x^2+y^2=0} $X^2 + Y^2 = 0$, with
			$p(1+\legendre{-1}{p}) + 1$ solutions.
			\item\label{case:rank2_x^2+ry^2=0} $X^2 + rY^2 = 0$, with
			$p(1+\legendre{-1}{p}) - 1$ solutions.
			\item\label{case:rank3_x^2+y^2+z^2=0} $X^2 + Y^2 + Z^2 = 0$ has $p+1$ solutions.
			\item\label{case:rank3_x^2+y^2+rz^2=0} $X^2 + Y^2 + rZ^2 = 0$ has $p+1$ solutions.
		\end{enumerate}
		Note that the first three equations have $p$ times plus $1$ the number
		of solutions of the corresponding equaton in Proposition 
		\label{prop:counting-P1-solutions}
	\end{prop}
	\begin{proof} We treat the cases by rank.
		
		\noindent
		\textit{Rank 1: \eqref{case:rank1_x^2=0}.} 
		The equations $X^2 = 0$ and $rX^2 = 0$ each have $|\P^1|=p + 1$ solutions.
		
		\noindent
		\textit{Rank 2: \eqref{case:rank2_x^2+y^2=0} and \eqref{case:rank2_x^2+ry^2=0}.} 
		The only solution with $Y=0$ is $[0:0:1]$. So assume $Y \neq 0$, and rewrite the 
		equations as $(X/Y)^2 = -1$ and $(X/Y)^2 = -r$ respectively.
		
		Suppose $-1$ is a residue. The equation $(X/Y)^2 = -1$ gives
		two possibilities for $X/Y$, and thus $2p$ solutions in $\P^2$. Hence
		$X^2 + Y^2 = 0$ has $2p + 1$ solutions in total. On the other hand, $(X/Y)^2 =
		-r$ has no solutions, meaning $[0:0:1]$ is the only solution to $X^2 + rY^2 =
		0$.
		
		If $-1$ is a nonresidue, the situation is reversed, and $X^2 + Y^2 = 0$ has one
		solution while $X^2 + rY^2 = 0$ has $2p + 1$ solutions.
		
		\noindent
		\textit{Rank 3: \eqref{case:rank3_x^2+y^2+z^2=0} and \eqref{case:rank3_x^2+y^2+rz^2=0}.} 
		Suppose $-1$ is a residue. If $Z = 0$, then we are in the setting of
		Proposition \ref{prop:counting-P1-solutions} case \eqref{case:x^2+y^2=0}, which has two
		solutions. Otherwise, we can divide by $Z$ and obtain 
		\begin{align*} 
		(X/Z)^2 + (Y/Z)^2 &= -1 \\ 
		(X/Z)^2 + (Y/Z)^2 &= -r 
		\end{align*} 
		respectively. By Proposition \ref{prop:circle-solutions} below, these each have $p - 1$
		solutions. Therefore $X^2 + Y^2 + Z^2 = 0$ and $X^2 + Y^2 + rZ^2 = 0$ each have
		$p+1$ solutions.
		
		Suppose $-1$ is a nonresidue. There are no solutions with $Z=0$, so rewrite the
		equations as above. By Proposition \ref{prop:circle-solutions} again, they each
		have $p + 1$ solutions.
	\end{proof}

    \begin{rem*}
      In fact, it is well-known that any nondegenerate conic is isomorphic (as an algebraic variety) to $\P^1$ (see for instance \cite[Prop.~19.3.1]{vakil}). This explains why cases \eqref{case:rank3_x^2+y^2+z^2=0} and \eqref{case:rank3_x^2+y^2+rz^2=0} both have $|\P^1|=p+1$ solutions.
    \end{rem*}
	
	\begin{prop}\label{prop:circle-solutions} 
		For $a \neq 0$, the equation $x^2 + y^2 = a$ has $p - \legendre{-1}{p}$ 
		solutions over $\F_p$.
	\end{prop} 
	\begin{proof} 
		Let $C_a$ denote the conic $x^2 + y^2 = a$. 
		First we will show the statement for $a=1$.
		
		Let $T$ denote the square-roots of $-1$ in $\F_p$. In the case
		that $-1$ is a nonresidue, $T$ is empty. Consider the maps $f\colon C_1(\F_p)
		\setminus (1,0) \to \F_p \setminus T$ and $g\colon \F_p
		\setminus T \to C_1(\F_p) \setminus (1,0)$ defined by 
		\begin{align*} 
		f(x,y) &= \frac{y}{x-1} \\ 
		g(m) &= \left(\frac{m^2 - 1}{m^2 + 1}, \frac{-2m}{m^2 + 1}\right).
		\end{align*} 
		It is easy to check that they are inverse to one
		another, from which it follows that $|C_1(\F_p)| = |\F_p| - |T| + 1 = p -
		\legendre{-1}{p}$ as desired.
		
		Note that if $a,a'$ are both nonzero residues (or both nonresidues)
		then $|C_a(\F_p)| = |C_{a'}(\F_p)|$. From the preceding, we have the claim for all residues
		$a$. Observe that $C_0(\F_p) \cup \cdots \cup C_{p-1}(\F_p)$ gives a partition of
		$\mathbb{A}^2$. Since 
		\[ |C_0(\F_p)| = \begin{cases}
		1 & \text{if } \legendre{-1}{p} = -1 \\
		-1 & \text{if } \legendre{-1}{p} = 1 \\
		\end{cases} \] 
		for a nonresidue $a$ we have 
		\[ |C_a(\F_p)| = \frac{2}{p-1} 
		\left(p^2 - |C_0(\F_p)| - \frac{p-1}{2}|C_1(\F_p)|\right) = 
		p - \legendre{-1}{p}. \qedhere \] 
	\end{proof}
	
	\begin{example}[A zoo of possibilities]
		\label{exa:all-possibilities}
		
		In the table below, the first two columns reflect the discussion
		immediately preceding. The remaining columns consider the number of extraneous
		solutions. The entries marked ``N/A'' are impossible and explanation is given
		afterwards. For all other entries, examples are given for the specific case
		$p=3$.  
		\begin{center} % TODO: convert to booktabs
			\begin{tabular}{c|c|c|c|c|c}  \multicolumn{2}{c|}{} &
				\multicolumn{4}{c}{\# soln. in $\{Z=0\}$}  \\ \hline
				rank of proj. & \# soln. in $\mathbb{P}^2$ & $0$ & $1$ & $2$ & $p + 1$\\
				\hline \hline
				$1$ & $p+1$ & N/A\textsuperscript{\ref{case:NA(p+1)-0,2}} & $x^2$ & N/A\textsuperscript{\ref{case:NA(p+1)-0,2}} & 1 \\
				\hline
				$2$ & $1$ & $x^2 + y^2$ & $x^2 + 1$ & N/A & N/A \\
				\hline
				$2$ &	$2p+1$ & N/A\textsuperscript{\ref{case:NA(2p+1)-0}} & $x^2 + x$ & $xy$ & $x$ \\
				\hline
				$3$ & $p+1$ & $x^2 + y^2 + 1$ & $x+y^2$ & $x^2 + 2y^2 + 1$ & N/A\textsuperscript{\ref{case:NA(p+1)-(p+1)}} \\
				\hline
			\end{tabular}
		\end{center} 
		\begin{enumerate}
			\item\label{case:NA(p+1)-0,2} When the projectivization has rank 1, it is a double line. If this
			line is distinct from $\{Z = 0\}$, they meet at exactly one point. 
			Otherwise they are the same line, giving $p+1$ solutions in $\{Z=0\}$. 
			\item\label{case:NA(2p+1)-0} When the projectivization has rank 2 and has $2p + 1$ solutions, it
			is the union of two intersecting lines. These lines cannot be disjoint 
			from $\{Z=0\}$.
			\item\label{case:NA(p+1)-(p+1)} There are $p+1$ solutions in $\{Z = 0\}$ only when the 
			coefficients $a,b,c$ are all zero. In particular, such a conic cannot 
			have a projectivization of rank 3.
		\end{enumerate} 
	\end{example}

    \section{Counting solutions of a conic in $\A^2$}
    \label{sec:affine-solutions}
	
	We can now use Lemma \label{lem:relate-solutions-affine-proj} to combine the
	results from the previous section into our main theorem.
	
    \begin{thm}[Counting Solutions in $\A^2$]
        \label{thm:main-thm}
	   Let $C$ be a conic as in (\ref{eq:original-conic}), and fix a prime $p$.
	   
	   If $p$ is even, then
	   \begin{equation*}
	       |C(\F_p)| = 2 + b(-1)^{(a + d)(c + e) + f},
	   \end{equation*}
	   where $b$ is either $0$ or $1$.
	
	   If $p$ is odd, let $\alpha$, $\beta$, and $\gamma$ be as defined in 
	   Lemma \ref{thm:diag-P2}, and $\Delta = b^2-4ac$ is the discriminant. 
	   If $\alpha$ and $\beta$ are both zero or
	   $\alpha$ is nonzero, we have one of the following cases, in order
	   of precedence:
	   \begin{enumerate}
	       \item All of $a$, $b$ and $c$ vanish, and $C(\F_p)$ is empty.
	       \item $\Delta = 0$, and $|C(\F_p)| = p$.
	       \item $\Delta \neq 0$, and,
	       \[ |C(\F_p)| = p - \legendre{-\Delta}{p}\legendre{-1}{p}. \]
	   \end{enumerate}
	   If instead only $\alpha$ vanishes, then one of the following holds, again
	   in order of precedence:
	   \begin{enumerate}
	       \item All of $a$, $b$ and $c$ vanish, and $|C(\F_p)|=p$.
	       \item If $\Delta = 0$, then
	       \[ |C(\F_p) = p\left(1 + \legendre{\gamma}{p}\legendre{-1}{p}\right). \]
	       \item $\Delta \neq 0$, and,
	       \[ |C(\F_p) = p\left(1 + \legendre{\gamma}{p}\legendre{-1}{p}\right)
	               - \legendre{\gamma}{p}\legendre{-1}{p}. \]
	   \end{enumerate}
	\end{thm}
	\begin{proof} 
	The case $p=2$ is very simple and can be checked by hand. There are very few conics over $\F_2$, and the solutions to 
		\[ ax^2 + bxy + cy^2 + dx + ey + f = 0 \] 
		are the same as the solutions to 
		\[ bxy + (a+d)x + (c+e)y + f = 0.  \] 
		At this point, one can simply list all of the
		possibilities. It can be checked that the number of solutions in each case agrees with the expression in the theorem statement.
		
		The remainder is a synthesis of \S\ref{sec:diagonalizing-quadratic-forms}, 
		\S\ref{sec:proj-solutions}, and Lemma~\ref{lem:relate-solutions-affine-proj}.
	\end{proof}
	
	\begin{comment}
        Consolidate with Example~4.4!
        \textcolor{red}{Note that if we are in case \eqref{case:0} of Proposition~\ref{prop:counting-P1-solutions}, i.e., if $a=b=c=0$, then $\gamma$ is a residue if and only if $-1$ is. Thus, by Theorem~\ref{thm:diag-P2}, we cannot be in cases \eqref{case:rank2_x^2+y^2=0} or \eqref{case:rank2_x^2+ry^2=0} of Proposition~\ref{prop:counting-P2-solutions} where the projectivized conic has only $1$ solution. Were this the case, Lemma~\ref{lem:relate-solutions-affine-proj} would imply that our affine conic has a negative number (to be exact, $-p$) of solutions!}
        \end{comment}

	We remark that while the expression for $|C(\F_p)|$ given above appears complicated, it is easily computable. The only parts that may be non-obvious are the Legendre symbols, but with tools such as the law of quadratic reciprocity, these too are readily determined.

    \section{To the $p$-adics via Hensel's lemma}\label{sec:hensels-lemma}
	Until now, we have been concerned with the number of solutions our conic $C$ has over $\F_p$. Now we will study solutions over the $p$-adic integers. We will give a sufficient condition for their existence in Theorem \ref{thm:existence-over-Zp}. Despite only being concerned with existence, the classification and analysis of the preceding sections will still be helpful.
	
	We begin by recalling the definition of the $p$-adic integers and then stating Hensel's lemma, which will be our primary tool for this section.
	\begin{defn}
		The ring of \emph{$p$-adic integers} $\Z_p$ is the inverse limit of
		\[
		\cdots \to \Z/p^3 \to \Z/p^2 \to \Z/p.
		\]
	\end{defn}
	In other words, elements of $\Z_p$ can be thought of as sequences $(a_1,a_2,a_3,\ldots)$ where $a_i \in \Z/p^i$ and $a_{i+1} \equiv a_i \pmod {p^i}$.
	
	Such an element can alternatively be interpreted as the series
	\[
	b_0 + b_1 p + b_2 p^2 + b_3 p^3 + \cdots
	\]
	where $b_0 = a_1$ and $b_i = (a_{i+1} - a_i)/p^i$ for $i \geq 1$. Addition and multiplication are done as expected with ``rightwards carrying.''
	
	%We will be particularly interested in the ``mod $p$'' map $\kappa\colon \Z_p \to \F_p$, whose kernel is the unique maximal ideal $(p) \subset \Z_p$. This implies that $\kappa^{-1}(\F_p^\times)$ is the set of units in $\Z_p$.
		
	% TODO: uncomment when needed (when used in section 7?)
	%Since $\Z_p$ is an integral domain, we are justified in taking its fraction field:
	%\begin{defn}
	%	The field of \emph{$p$-adic numbers} $\Q_p$ is the fraction field of $\Z_p$.
	%\end{defn}
	Hensel's lemma gives us a sufficient condition for the existence of roots over $\Z_p$. We refer the reader to \cite{henselMO} for a proof of the following formulation.
	\begin{thm}[Multivariate Hensel's lemma]\label{thm:hensels-one-eq}
		Let $f \in \Z_p[x_1,\ldots,x_n]$ be a polynomial in $n$ variables and let $\gamma = (\gamma_1,\ldots,\gamma_n)\in \Z_p^n$ be such that $f(\gamma) = 0 \pmod p$ and $\frac{\partial f}{\partial x_i}(\gamma) \not\equiv 0 \pmod p$ for some $i\in \{1,\ldots,n\}$. Then there exists $\alpha \in \Z_p^n$ such that, for all $i$, $\alpha_i = \gamma_i \pmod p$ and $f(\alpha) = 0$.
	\end{thm}
	Using the results of Proposition \ref{prop:counting-P2-solutions} and Hensel's lemma, we deduce the following result.
	\begin{thm}\label{thm:existence-over-Zp}
		Let $p$ be an odd prime. Let $C$ denote the conic \eqref{eq:original-conic} and $\widetilde{C}$ its projectivization \eqref{eq:projective-conic}. Then if $\widetilde{C}$ is either full rank or the union of two distinct lines over $\F_p$, the original conic $C$ has a solution over $\Z_p$.
		
		Explicitly,
		\begin{itemize}
			\item If $-1$ is a residue mod $p$ and $\widetilde{C}$ is equivalent to \eqref{case:rank2_x^2+y^2=0}, \eqref{case:rank3_x^2+y^2+z^2=0}, or \eqref{case:rank3_x^2+y^2+rz^2=0} of Proposition \ref{prop:counting-P2-solutions}, then $C$ has a solution over $\Z_p$.
			\item If $-1$ is a nonresidue mod $p$ and $\widetilde{C}$ is equivalent to \eqref{case:rank2_x^2+ry^2=0}, \eqref{case:rank3_x^2+y^2+z^2=0}, or \eqref{case:rank3_x^2+y^2+rz^2=0} of Proposition \ref{prop:counting-P2-solutions}, then $C$ has a solution over $\Z_p$.
		\end{itemize}
	\end{thm}
	\begin{proof}
		We need a point of $C$ over $\F_p$ where not all partial derivatives of the conic vanish, i.e. a \emph{non-singular} point. A non-singular point of $C$ corresponds to a non-singular point of $\widetilde{C}$ not lying in $\{Z=0\}$.
		
		If $\widetilde{C}$ has full rank, it is equivalent to
		\[
		X^2 + Y^2 + Z^2 = 0 \text{ or } X^2 + Y^2 + rZ^2 = 0
		\]
		where $r$ is a nonresidue. In each case, the condition of having a non-vanishing partial derivative is precisely that one of the coordinates must be nonzero---but this is automatically satisfied in $\P^2$.
		
		If $\widetilde{C}$ consists of two distinct lines, its only singular point is where the two lines intersect. We recall from \S\ref{sec:proj-solutions} that $\widetilde{C}$ has $2p+1$ points in $\P^2$, at most $p+1$ of which are extraneous. It follows that $C$ has a non-singular point.
	\end{proof}
	If $\widetilde{C}$ is a double line or has only a single solution, then Hensel's lemma is inconclusive as we do not have a solution mod $p$ satisfying the requisite condition.

    \section{To $\mathbb{Q}$ via Hasse-Minkowski}\label{sec:hasse-minkowski}
    The following takes place in $\P^2$. We start with a conic over $\Q$, which we can diagonalize so that it is represented by the matrix
    \[
    	A= \begin{bmatrix}
    		a & 0 & 0\\
    		0 & c & 0\\
    		0 & 0 & f
    	\end{bmatrix}.
    \]
    First off, if $\det A = acf = 0$ then the equation has obvious solutions. So assume $\det A \neq 0$.
    
    We can assume that $a,c,f \in \Z$, and that their product is square-free. This can always be achieved by scaling the entire equation or by scaling individual coordinates. Once this is accomplished, $\det A = \pm p_1\cdots p_m$ for distinct primes $p_1,\ldots,p_m$. Call this set of primes $\mathcal{P}$.
    
    The conic evidently has a solution in $\R$ if and only if $a$, $c$, and $f$ are neither all positive nor all negative.

    For each prime $p>2$ \emph{not} in $\mathcal{P}$, reducing mod $p$ gives a full-rank conic, to which Hensel's applies immediately and we deduce that our original conic has $\Q_p$-points.
    
    $\mathcal{P}$ is finite, so we only have finitely many other things to check. For each of these primes $p>2$, reducing mod $p$ gives a rank 2 conic. Depending on the coefficients and whether $-1$ is a residue (see \ref{prop:counting-P2-solutions}) we may either have two lines or a single solution. The former can be lifted with Hensel's.
    
    So what if we have the latter? In this case...
    \begin{quotation}
    	Can $A$ be written differently over $\Q_p$ so that its reduction mod $p$ is ``nicer'' to apply Hensel's to? Also, maybe Hensel's mod a higher power of the prime can tell us something?
    \end{quotation}
	I \emph{think} the answer to the first question is ``no'' because we already made the square-free assumption on $\det A$. I still haven't thought about the latter.
	
	Then there's also the pesky question of $p=2$ still.
	
	But this looks hopeful overall, since we only have to check $2$ and the primes which divide $\det A$ to come up with sufficient criteria for existence of $\Q$-points.

\begin{filecontents}{references.bib}
@book {omeara,
    AUTHOR = {O'Meara, O. Timothy},
     TITLE = {\href{https://link.springer.com/book/10.1007\%2F978-3-642-62031-7}{Introduction to quadratic forms}},
    SERIES = {Classics in Mathematics},
      NOTE = {Reprint of the 1973 edition},
 PUBLISHER = {Springer-Verlag, Berlin},
      YEAR = {2000},
     PAGES = {xiv+342},
      ISBN = {3-540-66564-1},
   MRCLASS = {11Exx},
  MRNUMBER = {1754311},
}
@article{magma,
author={Wieb Bosma and John Cannon and Catherine Playoust},
title={\href{http://www.sciencedirect.com/science/article/pii/S074771719690125X}{The Magma algebra system {I}: The user language}},
journal={Journal of Symbolic Computation},
volume={24},
year={1997},
pages={235--265}
}
@misc{henselMO,
	author={Will Sawin},
	title={Answer to \href{https://mathoverflow.net/questions/108687}{\texttt{https://mathoverflow.net/questions/108687}}}
}

@book{vakil,
title={\href{http://math.stanford.edu/~vakil/216blog/index.html}{Foundations of Algebraic Geometry}},
author={Ravi Vakil},
year={December 29, 2015}
}
\end{filecontents}

\bibliographystyle{plain}
\bibliography{references}

\end{document}

%%% Local Variables:
%%% mode: latex
%%% TeX-master: t
%%% End:
