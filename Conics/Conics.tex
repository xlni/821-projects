%Document setup
\documentclass[10pt,a4paper]{amsart} 

\usepackage[latin1]{inputenc}
\usepackage{mathtools} 
\usepackage{amssymb} 
\usepackage{mathrsfs} % gives mathscr font 
\usepackage{graphicx}
%\usepackage[headsep=0.15in, left=0.5in, right=0.5in, top=0.6in, bottom=0.5in]{geometry} 
%\usepackage[textlf,mathlf]{MinionPro} 
\usepackage{fancyhdr}
%\usepackage{xypic}
\usepackage{tikz-cd} 
\usepackage{todonotes}
\usepackage{verbatim}

%Environments
\numberwithin{equation}{section} 
\numberwithin{figure}{section}
\theoremstyle{definition} 
\newtheorem{example}{\protect\examplename}[section]
\theoremstyle{remark} 
\newtheorem*{rem*}{\protect\remarkname}
\theoremstyle{plain} 
\newtheorem{thm}{\protect\theoremname}[section]
\theoremstyle{plain} 
\newtheorem{cor}{\protect\corollaryname}[section]
\theoremstyle{definition} 
\newtheorem{defn}{\protect\definitionname}[section]
\theoremstyle{plain} 
\newtheorem{prop}{\protect\propositionname}[section]
\theoremstyle{plain} 
\newtheorem{lem}{\protect\lemmaname}[section]

\providecommand{\definitionname}{Definition}
\providecommand{\examplename}{Example} 
\providecommand{\lemmaname}{Lemma}
\providecommand{\propositionname}{Proposition}
\providecommand{\remarkname}{Remark} 
\providecommand{\corollaryname}{Corollary}
\providecommand{\theoremname}{Theorem}

%macros, etc.
\newcommand{\legendre}[2]{\genfrac{(}{)}{}{}{#1}{#2}}
\renewcommand{\arraystretch}{1.2} 

\newcommand{\A}{\mathbb{A}}
\renewcommand{\P}{\mathbb{P}} 
\newcommand{\F}{\mathbb{F}}
\newcommand{\Z}{\mathbb{Z}}
\newcommand{\Q}{\mathbb{Q}}
\newcommand{\R}{\mathbb{R}}

\begin{document} 

    \title{Conics (title tbd)} 

    \maketitle

    \begin{abstract} 
        to be written 
    \end{abstract}

    \tableofcontents

    \section{Introduction} 
    
    
    
    \section{Projectivization of the problem}

    Let us  begin by fixing a conic
    \begin{equation}\label{eq:original-conic} 
        ax^2 + bxy + cy^2 + dx + ey + f = 0 
    \end{equation} 
    and a choice of ground field $\F_p$. We further require that not all of 
    $a$, $b$, and $c$ be zero, and that the coefficients have gcd $1$.
    Unless otherwise specified, all primes are odd.
    (We defer treatment of the case $p = 2$ to \S\ref{sec:affine-solutions}.)
    \begin{defn}
        Let $\A^n_{\F_p}$ denote $n$-dimensional affine space over $\F_p$. Define 
        $n$-dimensional projective space over $\F_p$ as the quotient
        \[ \P^n_{\F_p} = (\A^{n+1}_{\F_p}\setminus \{0\})/{\sim}, \]
        where we declare that $(x_1,\ldots,x_{n+1}) \sim 
        (\lambda x_1,\ldots,\lambda x_{n+1})$ for $\lambda$ a nonzero scalar. We
        will primarially work with $n = 2$, and suppress the ground field from the 
        notation. We will write $[X:Y:Z]\in \P^2$ to denote the equivalence 
        class of $(X,Y,Z)\in \A^3 \setminus \{0\}$.
    \end{defn}

    With this construction in mind, we can \emph{projectivize} our conic
    \eqref{eq:original-conic} from $\A^2$ to $\P^2$, by writing it as a 
    homogenous equation: 
    \begin{equation}\label{eq:projective-conic} 
        aX^2 + bXY + cY^2 + dXZ + eYZ + fZ^2 = 0 
    \end{equation}
    Note that homogeneity ensures it is well-defined to ask when this
    equation has a solution in $\P^2$

    Let $S$ denote the solution set of \eqref{eq:original-conic} in $\A^2$. Our aim
    is to characterize $|S|$ as a function of the coefficients $a,b,c,d,e,f$ and 
    the prime $p$. The solutions of \eqref{eq:projective-conic} define a set
    $S_\P \subset \P^2$.

    There is an evident injection $S \rightarrowtail S_\P$ sending $(x,y)$ to
    $[x:y:1]$. However, elements of $S_\P$ of the form $[X:Y:0]$ do not 
    correspond to elements of $S$. 
    These are solutions ``at infinity'' in the projective plane.
    Thankfully, these extraneous solutions are easily characterized: they are the
    solutions to the equation
    \begin{equation}\label{eq:projective-conic-at-infinity} 
        aX^2 + bXY + cY^2 = 0
    \end{equation} 
    on the projective line $\P^1$. Call this solution set $S_0\subset \P^1$.  

    \begin{lem} 
        $|S_\P| = |S| + |S_0|$.
    \end{lem} 
    \begin{proof}
        A solution $[X:Y:Z]$ of \eqref{eq:projective-conic} corresponds 
        to a solution $(X/Z,Y/Z)$ of \eqref{eq:original-conic} when $Z \neq 0$,
        and to a solution $[X:Y]$ of \eqref{eq:projective-conic-at-infinity} 
        when $Z = 0$.  
    \end{proof}

    \section{Diagonalization of quadratic forms}
    \label{sec:diagonalizing-quadratic-forms} 

    (to be written)

    \section{Counting solutions of a conic in $\P^2$} 

    We precede this undertaking with a brief review of relevant number theory.
    The nonzero elements $\F_p^\times \subset \F_p$ form an abelian
    group under multiplication, and the squaring map 
    \[ \F^\times_p \xrightarrow{x \mapsto x^2} \F^\times_p \] 
    is a homomorphism. If $p$ divides $x^2 - 1 = (x+1)(x-1)$, we must have 
    $x = \pm 1$ mod $p$, so the kernel is $\{\pm 1\}$. 
    The cokernel $\F_p^\times / (\F_p^\times)^2$ is also group of order two: 
    \[ \F_p^\times / (\F_p^\times)^2 \cong \{\pm 1\}.  \]
    \begin{defn}
        The \emph{Legendere symbol} is the quotient map    
        \[ \legendre{\cdot}{p}: 
            \F_p^\times \to \F_p^\times / (\F_p^\times)^2 \cong \{\pm 1\}, \]
        where $\legendre{a}{p}$ is $-1$ if $a$ is a nonresidue,
        and $1$ if $a$ is a nonzero residue. This map extends to a map of magmas
        \[ (\F_p,\times) \to (\{-1,0,1\}, \times) \]
        with $\legendre{0}{p}$ defined as $0$.
    \end{defn}
    
    Note that each nonzero residue has exactly two square-roots (as we are assuming
    $p > 2$ throughout).

    In \S\ref{sec:diagonalizing-quadratic-forms}, we showed that, for an
    appropriate choice of coordinates, $aX^2 + bXY + cY^2$ can be rewritten as
    \begin{equation}\label{eq:general-form-P1} 
        a_1 X_1^2 + c_1 Y_1^2 = 0
    \end{equation} 
    and similarly, $aX^2 + bXY + cY^2 + dXZ + eYZ + fZ^2 = 0$ can be
    rewritten as 
    \begin{equation}\label{eq:general-form-P2} 
        a_2 X_2^2 + c_2 Y_2^2 + f_2 Z_2^2 = 0.
    \end{equation}
    The different subscripts express that the base
    changes involved for \eqref{eq:general-form-P1} and \eqref{eq:general-form-P2}
    need not be the same!

    Recall that the assumptions on the original coefficients 
    $a,b,c,d,e,f \in \Z$ were that 
    \begin{itemize} 
        \item not all of $a,b,c$ are zero, and
        \item $\gcd(a,b,c,d,e,f) = 1$.  
    \end{itemize} 
    If $p$ divides $a,b$, and $c$, then $a_1 = c_1 = 0$ above. However, 
    it will never be the case that $a_2 = c_2 = f_2 = 0$.

    \subsection{Extraneous solutions}
    \label{subsec:counting-P1-solutions}

    Fix a quadratic nonresidue $r \in \F_p$. By scaling the equation
    \eqref{eq:general-form-P1} and by scaling and permuting the coordinates as
    appropriate, we can obtain one of the following equivalent forms:
    \begin{enumerate} 
        \item $0 = 0$ if both $a_1, c_1$ are zero, 
        \item $X^2 = 0$ if exactly one of $a_1, c_1$ is zero, 
        \item $X^2 + Y^2 = 0$ if $a_1 c_1$ is a nonzero quadratic residue, and
        \label{case:x^2+y^2=0} \item $X^2 + rY^2 = 0$ if $a_1 c_1$ is a nonzero 
        quadratic nonresidue.  
    \end{enumerate}

    The first of these has $|\P^1| = p+1$ solutions. 

    The second has one solution, namely $[0:1]$.

    Evidently $Y$ must be nonzero for the last two cases, so we can rewrite them as
    $(X/Y)^2 = -1$ and $(X/Y)^2 = -r$ respectively. If $-1$ is a quadratic residue,
    then $-r$ is not. It follows that $X^2 + Y^2 = 0$ has two solutions and $X^2 +
    rY^2 = 0$ has none. If $-1$ is a nonresidue, the situation is reversed: $X^2 +
    Y^2 = 0$ has no solutions and $X^2 + rY^2 = 0$ has two solutions.

    \subsection{Total solutions}
    \label{subsec:counting-P2-solutions}

    Let $r$ be a quadratic nonresidue in $\F_p$ as before. Just as in
    \S\ref{subsec:counting-P1-solutions}, we can rewrite \eqref{eq:general-form-P2}
    as one of the following: 
    \begin{enumerate} 
        \item $X^2 = 0$ if two of $a_2, c_2, f_2$ are zero, 
        \item $X^2 + Y^2 = 0$ if exactly one of $a_2, c_2, f_2$ is zero,
        and the product of the other two is a quadratic residue, 
        \item $X^2 + rY^2 = 0$ if exactly one of $a_2, c_2, f_2$ is zero, 
        and the product of the other two is a quadratic nonresidue, 
        \item $X^2 + Y^2 + Z^2 = 0$ if $a_2 c_2 f_2$ is a nonzero quadratic residue, and 
        \item $X^2 + Y^2 + rZ^2 = 0$ if $a_2 c_2 f_2$ is
        a nonzero quadratic nonresidue.  
    \end{enumerate}

    Now we count the number of solutions in each of these cases.

    \subsubsection*{Rank 1 (first case)} 
    The equation $X^2 = 0$ has $p + 1$ solutions.

    \subsubsection*{Rank 2 (second and third cases)} 
    The only solution with $Y=0$ is $[0:0:1]$. So assume $Y \neq 0$, and rewrite the 
    equations as $(X/Y)^2 = -1$ and $(X/Y)^2 = -r$ respectively.

    Suppose $-1$ is a quadratic residue mod $p$. The equation $(X/Y)^2 = -1$ gives
    two possibilities for $X/Y$, and thus $2p$ solutions in $\mathbb{P}^2$. Hence
    $X^2 + Y^2 = 0$ has $2p + 1$ solutions in total. On the other hand, $(X/Y)^2 =
    -r$ has no solutions, meaning $[0:0:1]$ is the only solution to $X^2 + rY^2 =
    0$.

    If $-1$ is a nonresidue, the situation is reversed, and $X^2 + Y^2 = 0$ has one
    solution while $X^2 + rY^2 = 0$ has $2p + 1$ solutions.

    \subsubsection*{Rank 3 (fourth and fifth cases)} 
    Suppose $-1$ is a quadratic residue mod $p$. If $Z = 0$, then we are in the setting of
    \S\ref{subsec:counting-P1-solutions} case (\ref{case:x^2+y^2=0}), which has two
    solutions. Otherwise, we can divide by $Z$ and obtain 
    \begin{align*} 
        (X/Z)^2 + (Y/Z)^2 &= -1 \\ 
        (X/Z)^2 + (Y/Z)^2 &= -r 
    \end{align*} 
    respectively. By Proposition \ref{prop:circle-solutions} below, these each have $p - 1$
    solutions. Therefore $X^2 + Y^2 + Z^2 = 0$ and $X^2 + Y^2 + rZ^2 = 0$ each have
    $p+1$ solutions.

    Suppose $-1$ is a nonresidue. There are no solutions with $Z=0$, so rewrite the
    equations as above. By Proposition \ref{prop:circle-solutions} again, they each
    have $p + 1$ solutions. \footnote{\textcolor{red}{add comment about isomorphism
    to $\P^1$}}

    \begin{prop}\label{prop:circle-solutions} 
        For $a \neq 0$, the equation $x^2 + y^2 = a$ has $p - \legendre{-1}{p}$ 
        solutions over $\F_p$.
        \textcolor{red}{usage of $S_0$ in this proof is conflicting notation; need to address} 
    \end{prop} 
    \begin{proof} 
        Let $S_a \subset \mathbb{A}^2$ denote the solution set of $x^2 + y^2 = a$. 
        First we will show the statement for $a=1$.
    	
    	Let $T$ denote the square-roots of $-1$ in $\F_p$. In the case
        that $-1$ is a nonresidue, $T$ is empty. Consider the maps $f\colon S_1
        \setminus (1,0) \to \F_p \setminus T$ and $g\colon \F_p
        \setminus T \to S_1 \setminus (1,0)$ defined by 
        \begin{align*} 
            f(x,y) &= \frac{y}{x-1} \\ 
            g(m) &= \left(\frac{m^2 - 1}{m^2 + 1}, \frac{-2m}{m^2 + 1}\right).
        \end{align*} 
        It is easy to check that they are inverse to one
        another, from which it follows that $|S_1| = |\F_p| - |T| + 1 = p -
        \legendre{-1}{p}$ as desired.
    	
    	Note that if $a,a'$ are both nonzero residues (or both nonresidues)
        then $|S_a| = |S_{a'}|$. From the preceding, we have the claim for all residues
        $a$. Observe that $S_0 \cup \cdots \cup S_{p-1}$ gives a partition of
        $\mathbb{A}^2$. Since 
        \[ |S_0| = \begin{cases}
            1 & \text{if } \legendre{-1}{p} = -1 \\
            -1 & \text{if} \legendre{-1}{p} = 1 \\
        \end{cases} \] 
        for a nonresidue $a$ we have 
        \[ |S_a| = \frac{2}{p-1} 
                    \left(p^2 - |S_0| - \frac{p-1}{2}|S_1|\right) = 
                    p - \legendre{-1}{p}. \qedhere \] 
    \end{proof}

    \begin{example}[A zoo of possibilities]
    \label{exa:all-possibilities}
    	
    In the table below, the first two columns reflect the discussion
    immediately preceding. The remaining columns consider the number of extraneous
    solutions. The entries marked ``N/A'' are impossible and explanation is given
    afterwards. For all other entries, examples are given for the specific case
    $p=3$.  
    \begin{center} % TODO: convert to booktabs
        \begin{tabular}{|c|c||c|c|c|c|} \hline \multicolumn{2}{|c||}{} &
    \multicolumn{4}{c|}{\# soln. in $\{Z=0\}$}  \\ \hline rank of proj. & \# soln.
    in $\mathbb{P}^2$ & 0 & 1 & 2 & $p + 1$\\ \hline \hline 1 & $p+1$ & N/A & $x^2$
    & N/A & 1 \\ \hline 2 & 1 & $x^2 + y^2$ & $x^2 + 1$ & N/A & N/A \\ \hline 2 &
    $2p+1$ & N/A & $x^2 + x$ & $xy$ & x \\ \hline 3 & $p+1$ & $x^2 + y^2 + 1$ &
    $x+y^2$ & $x^2 + 2y^2 + 1$ & N/A \\ \hline \end{tabular}
    \end{center} 
    \begin{itemize}
        \item When the projectivization has rank 1, it is a double line. If this
        line is distinct from $\{Z = 0\}$, they meet at exactly one point. 
        Otherwise they are the same line, giving $p+1$ solutions in $\{Z=0\}$. 
        \item When the projectivization has rank 2 and has $2p + 1$ solutions, it
        is the union of two intersecting lines. These lines cannot be disjoint 
        from $\{Z=0\}$.
        \item There are $p+1$ solutions in $\{Z = 0\}$ only when the 
        coefficients $a,b,c$ are all zero. In particular, such a conic cannot 
        have a projectivization of rank 3.
    \end{itemize} 
    \end{example}

    \section{Counting solutions of a conic in $\A^2$}
    \label{sec:affine-solutions}

    Given a conic, we now have almost all of the necessary pieces to determine how
    many solutions it has in $\A^2_{\F_p}$. The remaining case is that of a
    conic over $\F_2$.

    Fortunately there are very few such conics, and the solutions to 
    \[ ax^2 + bxy + cy^2 + dx + ey + f = 0 \] 
    are the same as the solutions to 
    \[ bxy + (a+d)x + (c+e)y + f = 0.  \] 
    At this point, one can simply list out all the
    possibilities. It can be checked that the number of solutions in each case can
    be concisely written as $2 + b(-1)^{(a+d)(c+e) + f}$, where $b$ is either 0 or
    1. \textcolor{red}{I have temporarily put this here, but this might flow better
    if moved into the proof of the theorem below. }

    (remainder to be written)

    \begin{thm} 
        main theorem determining $F(a,b,c,d,e,f,p)$ 
    \end{thm}

    % TODO: get a reference for the following fact (which we'll probably need)?
    \begin{prop} 
        For an odd prime $p$, the Legendre symbol $\legendre{-1}{p}$ is
        given by 
        \[ \legendre{-1}{p} = \begin{cases} 
            1 & \text{if } p \cong 1 \pmod 4 \\ 
            -1 & \text{if } p \cong 3 \pmod 4.  
        \end{cases} \]
    \end{prop}

    \section{To the $p$-adics via Hensel's lemma} (to be written)

    \section{To $\mathbb{Q}$ via Hasse-Minkowski} (to be written) '
\end{document}
