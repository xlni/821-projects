\section{$S$ is a linear subspace of $V$}\label{sec:part0}
\textcolor{red}{moved some stuff here from first section, will edit some later---need to put lemma in here somewhere}

In this situation, the Fourier transform is somewhat simpler, and is entirely
determined by what a functional $\lambda$ does on the subset $S$.
\[ \Phi(\IND_S)(\lambda) = \sum_{x\in S} \exp \left(\frac{2\pi i}{p} \lambda(x)\right). \]
Noting that $\IND_S = 1 - \IND_{V\setminus S}$, we can write the Fourier transform as
\[ \Phi(\IND_S)(\lambda) = 1 - \sum_{x\not\in S} \exp \left(\frac{2\pi i}{p} \lambda(x)\right), \]
which will be occasionally convenient. 

\begin{rem}
	Throughout, for each vector space $V$, we implicitly pick an isomorphism 
	$V\cong\F_p^n$, i.e., an explicit basis. This
	entitles us to an inner product on $V$, as well as an isomorphism between $V$ and
	$V^*$, where all functionals are of the form $x \mapsto \langle \lambda, x \rangle$
	for some $\lambda\in V$. By abuse of notation, $\lambda$ also denotes the functional.
\end{rem}

Let us being with a warmup. Consider the case of a subspace $W$ of our vector space $V$.
A classical theorem of linear algebra allows us to decompose $V$:
\begin{prop}
    The \emph{orthogonal complement} of $W$ is the subspace    
    \[ W^\perp = \{ v \in V \,|\, \langle v, w \rangle = 0 \text{ for all } w \in W \}. \]
    Any element $x\in V$ may be uniquely written as a sum $x = w + w'$, where $w\in W$
    and $w'\in W^\perp$.
\end{prop}

Before we consider the Fourier transform over the whole space $V$, consider what
it does to elements of $W$:
\begin{lem}
    \label{lem:subspace-transform-on-subspace}
    Suppose $\lambda\in W$ is nonzero. Then, $\Phi(\IND_W)(\lambda) = 0$.
    \begin{proof}
        We prove this by induction on the dimension of $W$. Fix any element $w\in W$,
        and let $U\subset W$ be the line spanned by it. Then,   
        \[ \Phi(\IND_W)(\lambda) = 
            \sum_{x\in W/U} 
                \exp \left(\frac{2\pi i}{p} \langle \lambda, x \rangle\right) +
            \sum_{x\in U} 
                \exp \left(\frac{2\pi i}{p} \langle \lambda, x \rangle\right). \]
        It is easy to see that the rightmost term is zero:
        \[ \sum_{x\in U} 
                \exp \left(\frac{2\pi i}{p} \langle \lambda, x \rangle\right) =
            \sum_{x\in \F_p} 
                \exp \left(\frac{2\pi i}{p} x \langle \lambda, w \rangle\right) = 0,
                \]
        since $\langle \lambda, w \rangle \neq 0$. Since the dimension of $W/U$ is less
        than that of $W$, we can continue repeating this process until there's nothing left.
    \end{proof}
\end{lem}
We can now put these facts together to prove our result:
\begin{thm}
    $\Phi(\IND_W) = |W| \IND_{W^\perp}$.
    \begin{proof}
        Given $\lambda\in V$, split it into orthogonal parts as $\lambda = w + w'$. Then,
        \[ \Phi(\IND_W)(w + w') = 
            \sum_{x\in W} 
                \exp \left(\frac{2\pi i}{p} \langle w + w', x \rangle\right) = 
            \sum_{x\in W} 
                \exp \left(\frac{2\pi i}{p} \langle w, x \rangle\right). \]
        If $w\neq 0$, by Lemma \ref{lem:subspace-transform-on-subspace}, the whole
        sum will vanish; otherwise, the exponent will be constantly $1$, and add up to
        $|W|$. In the case that $w = 0$, $\lambda\in W^\perp$, as desired.
    \end{proof}
\end{thm}
TODO: geometric intution?
