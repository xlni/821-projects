\section{Conical subsets of $V$}\label{sec:part0}

Let $V$ be an $\F_p$ vector space.
Throughout the rest of paper, for all such $V$, we implicitly pick an isomorphism 
	$V\cong\F_p^n$, i.e., an explicit basis. This
	entitles us to an inner product on $V$, as well as an isomorphism between $V$ and
	$V^*$, where all functionals are of the form $x \mapsto \langle \lambda, x \rangle$
	for some $\lambda\in V$. By abuse of notation, $\lambda$ also denotes the functional.

We are interested in indicator functions of conical subsets of $V$:
\begin{defn}
    A subset $S\subset V$ is called \emph{conical} if it is closed under scaling,
    i.e., for all $x\in \F_p$, $xS = S$.    
\end{defn}
Since we're studying indicator functions, we can simplify our expression for 
$\Phi(\IND_S)$:
\[ \Phi(\IND_S)(\lambda) = \sum_{x\in S} \exp \left(\frac{2\pi i}{p} \lambda(x)\right). \]
Noting that $\IND_S = 1 - \IND_{V\setminus S}$, we can further write the Fourier transform as
\[ \Phi(\IND_S)(\lambda) = 1 - \sum_{x\not\in S} \exp \left(\frac{2\pi i}{p} \lambda(x)\right), \]
which will be occasionally convenient. The following lemma further reduces this expression,
which will be useful throughout the paper.
\begin{lem}\label{lem:FT-conical-subset}
	For a conical subset $S \subset V$,
	\[ \Phi(\IND_S)(\lambda) = \frac{p|S \cap \ker \lambda| - |S|}{p-1}. \]
	\begin{proof}
	   The key insight is that a conical set can be viewed as a collection of lines.
	   Define $S\P = (S\setminus\{0\}) / {\sim}$, where we declare that 
	   $x \sim a x$, for all $a\in\F_p^\times$ and
	   $x\in S$ (as one might define projective space). Let $[x]$ denote the class
	   of $x$ in $S\P$. We can then write the Fourier transform as follows:
	   \[ \Phi(\IND_S)(\lambda) = 
	           1 + \sum_{[x]\in S\P} 
	               \sum_{a\in\F_p^\times} \exp \left(\frac{2\pi i a}{p} \lambda(x)\right) \]
	   If $\lambda(x)$ vanishes, then the inner sum is just $p-1$; if it doesn't, then it's
	   a sum of equal powers of all the roots of unity, except $1$, so the inner sum is $-1$.
	   Thus,
	   \begin{eqnarray*}
	       \Phi(\IND_S)(\lambda)
	       &=& 1 + 
	           (p-1)|S\P \cap \ker \lambda| - 
	           |S\P \setminus \ker\lambda| \\
	       &=& 1 + 
	           (p-1)|S\P \cap \ker \lambda| - 
	           (|S\P| - |S\P \cap \ker\lambda|) \\
	       &=& 1 + 
	           p|S\P \cap \ker \lambda| - |S\P|
	   \end{eqnarray*}
	   Note now that $|S\P| = \frac{|S| - 1}{p-1}$, by definition. If we substitute
	   this in, we find that 
	   \[ \Phi(\IND_S)(\lambda) = \frac{p|S \cap \ker \lambda| - |S|}{p-1}, \]
	   as desired.
	\end{proof}
\end{lem}

The simplest conical set is a linear subspace $W$ of $V$.
To show the power of Lemma \ref{lem:FT-conical-subset}, we use it to compute $\IND_W$
as a warmup. We'll need a classical theorem of linear algebra allows us to decompose $V$
along $W$:
\begin{prop}
    The \emph{orthogonal complement} of $W$ is the subspace    
    \[ W^\perp = \{ v \in V \,|\, \langle v, w \rangle = 0 \text{ for all } w \in W \}. \]
    Any element $x\in V$ may be uniquely written as a sum $x = w + w'$, where $w\in W$
    and $w'\in W^\perp$.
\end{prop}

\begin{thm}\label{thm:FT-subspace}
    $\Phi(\IND_W) = |W| \IND_{W^\perp}$.
    \begin{proof}
        By Lemma \ref{lem:FT-conical-subset}, we need only compute what 
        $|W\cap\ker\lambda|$. We can split $\lambda$ into orthogonal parts $w + w'$.
        Then, for $x\in W$,
        \[ \langle w + w', x \rangle = \langle w, x \rangle + \langle w', x \rangle
            = \langle w, x \rangle. \]
        Thus, $x\in W\cap\ker\lambda$ iff $w = 0$, i.e., $\lambda\in W^\perp$. Since
        this is independent of $x$, $W\cap\ker\lambda$ is either all of $W$ or zero,
        depending on if $\lambda$ is in $W^\perp$ or not. Plugging this into the
        formula of Lemma \ref{lem:FT-conical-subset} gives our result.
    \end{proof}
\end{thm}