\section{Introduction}\label{sec:intro}

The \emph{Fourier transform} takes a function $f:\R\to\C$ and produces another
function, $\widehat{f}:\R\to\C$, via the following integral transform:
\[ \widehat{f}(\xi) = \int_\R e^{-2\pi i x \xi} f(x)\, dx. \]
Conceptually, this takes a \emph{frequency} $\xi$, which determines the ``pure sinusoid''
$\chi_\xi (x) = e^{-2\pi i x \xi}$, and convolves it with $f$ to extract the \emph{amplitude} of $f$ 
at that frequency. This is especially clear when $f$ represents a wave, such as an alternating
current through time or a quantum mechanical wavefunction. 

The Fourier transform generalizes to higher dimensions, too. If we now have a function
$f:\R^n\to\C$, we can express the pure sinusoid associated to $\xi\in\R^n$ by 
$\chi_\xi (x) = e^{-2\pi i x\cdot\xi}$. The dot product suggests that, for a general real vector space $V$, it would be more natural to define the 
Fourier transform as a function from the dual, $V^* \to \C$, given by
\[ \widehat{f}(\xi) = \int_\R e^{-2\pi i \xi(x)} f(x)\, dx. \]
This is the same transform described for $\R^n$; every linear functional from $\R^n$ is
just given by a dot product. In all of the cases outlined above, a pure sinusoid, or
\emph{character}, sends addition to multiplication. In fact, it's a group homomorphism 
$\chi:V \to \C^\times$, where we require that $|\chi(x)| = 1$ for all $x\in V$.

We don't need to stick to real analysis to perform Fourier transforms; they're
valid wherever we have a notion of \emph{characters}
and \emph{integration}. A familiar object with both of these properties is the finite field
$\F_p$, as we will show. Our choice of integration is easy: given any $f:\F_p\to\C$, we define
\[ \int_{\F_p} f(x)\, dx = \sum_{x\in\F_p} f(x). \]
It's a linear map, and respects translation: $f(x + n)$, $n$ fixed, integrates to the 
same value as $f(x)$. Characters require just a bit more thinking: to give a character 
$\chi$, we need to give a complex number whose $p$th power is $1$, that is, we need
to chose a root of unity. So we have a formula for the set of characters:
\[ \chi_\xi(x) = \exp {\frac{2\pi i}{p} x \xi}, \]
where $\xi\in\F_p$. Applying a generalization like for $\R^n$, we can define a general
finite Fourier transform.
\begin{defn}\label{defn:intro-FpFT}
    Let $V$ be any $\F_p$-vector space, and $f:V\to \C$ be any function. The Fourier
    transform is a function $\widehat{f}:V^*\to\C$, defined on the dual space, given by
    \[ \widehat{f}(\xi) = 
        \sum_{x\in\F_p} \exp \left(\frac{2\pi i}{p} \xi(x)\right) f(x). \]
\end{defn}

It is evident from its definition that the Fourier transform is a \emph{linear} map from the space of $\C$-valued functions on $V$ to the space of $\C$-valued functions on $V^*$. In this paper, we will consider only finite-dimensional (and thus finite) $V$. Then, all functions on $V$ are complex linear combinations of indicator functions, where for a subset $S\subset V$ we define
\[ \IND_S(x) = \begin{cases}
1 & x \in S \\
0 & \text{otherwise.}     
\end{cases} \]
Thus it is reasonable to focus on these functions specifically. One could take this to an extreme and compute the Fourier transform $\widehat{\IND_{\{x\}}}$ for each $x\in V$, and then dismissively say that the Fourier transform of any function is just a linear combination of these:
\[
	\widehat{f} = \sum_{x\in V} f(x)\widehat{\IND_{\{x\}}}.
\]

While this is true, it is not an illuminating answer to the question ``what is $\widehat{f}$?'' As we will demonstrate in this paper, this question often admits a much more revealing answer than the above.

In what follows, we will exclusively consider $\widehat{\IND_S}$ for \emph{conical} subsets $S\subset V$, meaning that $S$ is preserved under scaling. We will see in \S\ref{sec:part0} (specifically, Lemma~\ref{lem:FT-conical-subset}) that $\widehat{\IND_S}$ admits a simpler description in this case. We will use this description throughout the rest of the paper to compute $\widehat{\IND_S}$ for specific examples of $S\subset V$, beginning in \S\ref{sec:part0} with the case where $S$ is a vector subspace of $V$. In section \S\ref{sec:part1}, we consider the situation in which $S$ is the vanishing set of a homogeneous quadratic polynomial in $n$-variables in an affine $n$-space $V$. Next, in \S\ref{sec:part2}, we consider the situation in which $V$ is dual to the space of homogeneous degree-$n$ bivariate polynomials and $S$ is the conical subset generated by the functionals that evaluate these polynomials at a specified point of $\F_p^2$. Finally, in \S\ref{sec:part3} we analyze the case when $S$ is the set of nilpotent $n\times n$ matrices in the space $V$ of all $n\times n$ matrices.

\subsection*{Acknowledgments} Miguel wrote \S\ref{sec:intro} and \S\ref{sec:part0}. Xianglong wrote \S\ref{sec:part1}, \S\ref{sec:part2}, and Appendix \ref{sec:recursion-for-quadratic-solns}. Oron wrote \S\ref{sec:part3}, and helped Xianglong with \S\ref{sec:part2}.

We would like to thank the 18.821 staff for coordinating the course; Stuart Thomson for his mentorship throughout this project; Ari Nieh, Haynes Miller, and Stuart Thomson for their comments on an earlier draft of this paper; and Roman Bezrukavnikov for suggesting this project.
%%% Local Variables:
%%% mode: latex
%%% TeX-master: t
%%% End:
