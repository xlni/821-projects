\section{Introduction}\label{sec:intro}

The \emph{Fourier transform} takes a function $f:\R\to\C$ and produces another
function, $\Phi(f):\R\to\C$, via the following integral transform:
\[ \Phi(f)(\lambda) = \int_\R e^{-2\pi i x \lambda} f(x)\, dx. \]
Conceptually, this takes a \emph{frequency} $\lambda$, which determines the ``pure sinusoid''
$e^{-2\pi i x \lambda}$, and convolves it with $f$ to extract the \emph{amplitude} of $f$ 
at that frequency. This is especially clear when $f$ represents a wave, such as an alternating
current through time or a quantum mechanical wavefunction. 

The Fourier transform generalizes to higher dimensions, too. If we now have a function
$f:\R^n\to\C$, we can express the pure sinusoid associated to $\lambda\in\R^n$ by 
$e^{-2\pi i x\cdot\lambda}$ (note the dot product). For a general real vector space $V$, where
we still have a notion of integration, it might be more natural to define the 
Fourier transform as a function from the dual, $V^* \to \C$, given by
\[ \Phi(f)(\lambda) = \int_\R e^{-2\pi i \lambda(x)} f(x)\, dx. \]
This is the same transform described for $\R^n$; every linear functional from $\R^n$ is
just given by a dot product. In all of the cases outlined above, a pure sinusoid, or
\emph{character}, sends addition to multiplication. In fact, it's a group homomorphism 
$\lambda:V \to \C^\times$, where we require that $|\lambda(x)| = 1$ for all $x\in V$.

In fact, we don't need to stick to real analysis to preform Fourier transforms; they're
valid wherever we have a notion of \emph{characters}
and \emph{integration}. A familiar object with both of these properties is the finite field
$\F_p$, as we will show. Our choice of integration is easy: given any $f:\F_p\to\C$, we define
\[ \int_{\F_p} f(x)\, dx = \sum_{x\in\F_p} f(x). \]
It's a linear map, and respects translation: $f(x + n)$, $n$ fixed, integrates to the 
same value as $f(x)$. Characters require just a bit more thinking: to give a character 
$\lambda$, we need to give a complex number whose $p$th power is $1$, that is, we need
to chose a root of unity. So we have a formula for the set of characters:
\[ \lambda(x) = \exp {\frac{2\pi i}{p} x \lambda}, \]
where we've identified the character $\lambda$ with the element $\lambda\in\F_p$, by
abuse of notation. Applying a generalization like for $\R^n$, we can define a general
finite Fourier transform.
\begin{defn}
    Let $V$ be any $\F_p$-vector space, and $f:V\to \C$ be any function. The Fourier
    transform is a function $\Phi(f):V^*\to\C$, defined on the dual space, given by
    \[ \Phi(f)(\lambda) = 
        \sum_{x\in\F_p} \exp \left(\frac{2\pi i}{p} \lambda(x)\right) f(x). \]
\end{defn}

It is evident from its definition that the Fourier transform is a \emph{linear} map from the space of $\C$-valued functions on $V$ to the space of $\C$-valued functions on $V^*$. In this paper, we will consider only finite-dimensional (and thus finite) $V$. Then, all functions on $V$ are linear combinations of indicator functions, where for a subset $S\subset V$ we define
\[ \IND_S(x) = \begin{cases}
1 & x \in V \\
0 & \text{otherwise.}     
\end{cases} \]
Thus it is reasonable to focus on these functions specifically. One could take this to an extreme and compute the Fourier transform $\Phi(\IND_{\{v\}})$ for each $v\in V$, and then dismissively say that the Fourier transform of any function is just a linear combination of these:
\[
	\Phi(f) = \sum_{v\in V} f(v)\Phi(\IND_{\{v\}}).
\]

While this is true, it alone is hardly an answer to the question ``what is $\Phi(f)$?'' As we will demonstrate in this paper, this question often admits a much more elegant answer than the above.

In what follows, we will exclusively consider $\Phi(\IND_S)$ for \emph{conical} subsets $S\subset V$, meaning that $S$ is preserved under scaling. We will see in \S\ref{sec:part0} that $\Phi(\IND_S)$ has a simpler description in this case, and also compute the Fourier transform for when $S$ is a subspace of $V$. In sections \S\ref{sec:part1} and \S\ref{sec:part2} we consider two situations in which $S$ is related to polynomial equations. Finally, in \S\ref{sec:part3} we analyze the case when $S$ is the set of nilpotent $n\times n$ matrices in the space $V$ of all $n\times n$ matrices.

\subsection*{Acknowledgments} Miguel wrote \S\ref{sec:intro} and \S\ref{sec:part0}. Xianglong wrote \S\ref{sec:part1}, \S\ref{sec:part2} (with Oron), and Appendix \ref{sec:recursion-for-quadratic-solns}. In addition to helping with \S\ref{sec:part2}, Oron also wrote \S\ref{sec:part3}.

We would like to thank the 18.821 staff for coordinating everything, and our mentor Stuart Thomson for his help throughout this project.