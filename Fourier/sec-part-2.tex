\section{Homogeneous polynomials in 2 variables}\label{sec:part2}
In the preceding section, we considered degree $2$ homogeneous polynomials in $n$ variables. Now we will instead consider degree $n$ homogeneous polynomials in $2$ variables. However, we will set things up quite differently.

Let $\F_p[x,y]_n$ denote the space of homogeneous polynomials of degree $n$ over $\F_p$. It is an $(n+1)$-dimensional vector space, spanned by the basis elements
\[
	\{x^n, x^{n-1}y, \ldots, xy^{n-1}, y^n\}.
\]
Now define $V$ to be the dual space of this vector space:
\[
	V \coloneqq \left(\F_p[x,y]_n\right)^*.
\]
Of particular algebraic interest are the elements of $V$ which are given by \emph{evaluation}---these are the functionals of the form
\[
	\F_p[x,y]_n \ni P \mapsto P(a,b) \in \F_p
\]
for some $a,b\in \F_p$. However, this set is not conical for $n>1$. So we will instead let $S$ be the set
\[
	S=\{\xi_{a,b,c} \in V \mid \xi_{a,b,c}(P) = cP(a,b)\}.
\]
The Fourier transform of $\IND_S$ will be a $\C$-valued function on $V^*$, where $V^*$ is the double dual of $\F_p[x,y]_n$. Since $\F_p[x,y]_n$ is finite-dimensional, it is canonically isomorphic to its double dual, so we will identify $V^* = \F_p[x,y]_n$.

Define $\P^1$ to be the set of lines through the origin in $\F_p^2$. For each line $\ell \in \P^1$, pick a nonzero point $(a_\ell,b_\ell)$ on $\ell$. Then we have
\begin{equation}\label{eq:part2S}
	S = \{0\} \cup \{\xi_{a_\ell,b_\ell,c} \mid c\in \F_p \text{ nonzero}, \ell \in \P^1 \}
\end{equation}
and hence
\[
	|S| = 1 + (p+1)(p-1) = p^2.
\]
We will now give a very simple description of $\Phi(\IND_S)$.
\begin{thm}\label{thm:part2thm}
	Let $\F_p[x,y]_n$ be the space of homogeneous polynomials of degree $n$ over $\F_p$, $V$ its dual space, and $S\subset V$ the set $\{\xi_{a,b,c} \in V \mid \xi_{a,b,c}(P) = cP(a,b)\}$.
	
	An element $\lambda \in V^*$ is a polynomial $P(x,y)$. Let $N_\lambda$ be the number of solutions to $P(x,y)=0$ in $\F_p^2$. Then
	\begin{equation}
		\Phi(\IND_S)(\lambda) = \frac{pN_\lambda - p^2}{p-1}.\label{eq:FTpart2}
	\end{equation}
\end{thm}
\begin{proof}
	An element $\xi_{a,b,c}\in S$ is in $\ker \lambda$ when $P(a,b) = 0$. From \eqref{eq:part2S}, we have
	\[
		|S \cap \ker \lambda| = N_\lambda
	\]
	and thus, by Lemma \ref{lem:FT-conical-subset},
	\begin{align*}
		\Phi (\IND_S) (\lambda) &= \frac{p|S\cap \ker \lambda| - |S|}{p-1}\\
		&= \frac{pN_\lambda - p^2}{p-1}.\qedhere
	\end{align*}
\end{proof}
There is unfortunately no explicit formula for the number $N_\lambda$ of solutions to $P(x,y) = 0$ in terms of the coefficients of $P$, so we will not attempt to expand further on \eqref{eq:FTpart2}. Instead, we offer an alternative geometric perspective which also leads to a proof of Theorem \ref{thm:part2thm}. Each polynomial in $V^* = \F_p[x,y]_n$ can be thought of a row vector $(a_0,\ldots,a_n)$ where $a_i$ is the coefficient of $x^{n-i}y^i$ in the polynomial. Then ``evaluation at $(a,b)$'' is given by pairing with the column vector
\[
	(a^n,a^{n-1}b,\ldots,ab^{n-1},b^n)^\top.
\]
With this identification, $S = \{\xi_{a,b,c}\}$ is the union of the following lines:
\begin{align*}
	\ell_\infty &= \langle (0,0,\ldots,0,1)^\top \rangle,\\
	\ell_0 &= \langle (1,0,\ldots,0,0)^\top \rangle,\\
	\ell_1 &= \langle (1,1,\ldots,1,1)^\top \rangle,\\
	\ell_2 &= \langle (1,2,\ldots,2^{n-1},2^n)^\top \rangle,\\
	\vdots &= \vdots\\
	\ell_{p-1} &= \langle (1,p-1,\ldots,(p-1)^{n-1},(p-1)^n)^\top \rangle
\end{align*}
As an aside for the reader familiar with algebraic geometry, if we think of $S$ as living inside projective space $\P^{n}$, it is the \emph{rational normal curve} of degree $n$ over $\F_p$.

Using the fact that $S$ is the union of the above lines, we can write
\begin{align*}
	\Phi(\IND_S) &= \Phi\left( \sum \IND_{\ell_i} - p\IND_{\{0\}} \right)\\
	&= p\left(\sum \IND_{\ell_i^\bot} - \IND_{V^*}\right).
\end{align*}
where we have used the linearity of $\Phi$ together with Theorem \ref{thm:FT-subspace}. A polynomial $P$ is in $\ell_i^\bot$ when it vanishes on $\ell_i$. If $P$ vanishes at $N$ points in total, then it vanishes on $(N - 1)/(p-1)$ of the lines $\ell_i$. This gives another proof of Theorem \ref{thm:part2thm}.

What is the ``geometry'' of these hyperplanes $\ell_i^\bot$? If $n < p$ then clearly not all $p+1$ of the lines $\ell_i$ can be linearly independent. We claim, however, that every subset of size at most $n+1$ is indeed linearly independent. A way of seeing this is by noting that the Vandermonde determinant is nonzero. Alternatively, one can explicitly produce a polynomial $P \in \F_p[x,y]_n$ which vanishes on $n$ given lines but not on the others. On the other hand, if an element of $\F_p[x,y]_n$ vanishes on $n+1$ of the lines, then in fact it vanishes on all of the lines. This is entirely what we expect: dividing the polynomial by $y^n$ gives a univariate polynomial in $(x/y)$ of degree at most $n$. If such a polynomial has $n+1$ roots, then it is identically zero.