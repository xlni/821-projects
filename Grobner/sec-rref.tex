\section{Existence and uniqueness of Gr\"obner bases}\label{sec:rref}
We commence this section with an example. Let $X = \{x,y\}$ with $x<y$ and consider the ideal in $T(V)$ generated by $x^2 - y$. As a vector space, it is spanned by the set $\{u(x^2 - y)v \mid u,v\text{ monomials}\}$. We can organize these polynomials as rows of a matrix in the following way, where empty entries are zero:
\[
	\begin{blockarray}{cccccccccc}
	\cdots & x^2 y & x^3 & y^2 & yx & xy & xx & y & x & 1\\
	\begin{block}{[cccccccccc]}
	\ddots&\vdots& \vdots & \vdots& \vdots& \vdots& \vdots& \vdots& \vdots\\
	\cdots&&1 & 0 & -1 & 0 & 0 & 0 & 0 & 0\\
	\cdots&&1 & 0 & 0 & -1 & 0 & 0 & 0 & 0\\
	\cdots&& &  &  &  & 1 & -1 & 0 & 0\\
	\end{block}
	\end{blockarray}
\]
To use language from linear algebra, we see that this matrix is \emph{not} in reduced row echelon form: the distinct rows corresponding to $x^3 - yx$ and $x^3 - xy$ both have pivots in the $x^3$ position, reflecting the fact that we have an inconsistent overlap. As we will demonstrate in this section, the endeavor of producing a Gr\"obner basis is essentially equivalent to putting the above matrix in reduced row echelon form. Let us first make precise what is meant by this phrase---it is usually used in the context of finite-dimensional vector spaces, but we are dealing with infinite-dimensional ones.

\begin{prop}[Reduced row echelon form]\label{prop:rref}
	Let $W$ be a linear subspace of $T(V)$. Then $W$ has a unique basis satisfying the following properties:
	\begin{enumerate}
		\item Each element of the basis has the form $u_i - \psi_i$, where $u_i$ is a monomial and $\psi_i$ is lower than $u_i$.
		\item $u_i \neq u_j$ for all distinct $i,j$.
		\item $\psi_i$ does not contain the monomial $u_j$ for all $i,j$.
	\end{enumerate}
\end{prop}
\begin{proof}
	This is completely analogous to the standard fact that reduced row echelon form of a matrix is unique, although the difference is that $W$ may be infinite-dimensional. The basis can be constructed as follows.
	
	We construct the basis inductively, starting from the lowest monomial. Suppose we have finished considering all monomials lower than $u$. If there exists a polynomial $\psi$ such that
	\begin{enumerate}
		\item $\psi$ is lower than $u$,
		\item $\psi$ does not contain any monomial $u_i$ for which $u_i - \psi_i$ has already been added to the basis, and
		\item $u - \psi \in W$,
	\end{enumerate}
	then we add $u - \psi$ to the basis. Otherwise we do not add anything to the basis, and move on to the next monomial. Note that if such a $\psi$ exists, then it is unique---for if $\psi,\psi'$ both satisfy the above conditions, we must have $\psi - \psi' \in W$. If this is nonzero, then (up to scaling) it can be rewritten in the form $\bar{u} - \bar{\psi}$ where $\bar{\psi}$ is lower than the monomial $\bar{u}$. Since $\psi, \psi'$ satisfy (2) above, we know that nothing was added to the basis when we considered the monomial $\bar{u}$, but this gives a contradiction---we should have added $\bar{u}-\bar{\psi}$.
	
	Note that although $T(V)$ is infinite-dimensional, this inductive process is valid because the monomials are well-ordered. It is easy to check that the resulting set $\{u_i - \psi_i\}$ is a basis for $W$. Moreover, it is evident from its construction that it is unique.
\end{proof}
Now we will use this to prove an existence and uniqueness result for Gr\"obner bases (though we remind the reader that this is for a fixed well-ordering of $X$).
\begin{thm}\label{thm:exist+unique}
	Every two-sided ideal $\mathfrak{a}\in T(V)$ has a unique (reduced) Gr\"obner basis.
\end{thm}
\begin{proof}
	To prove this, we will show that a Gr\"obner basis gives a vector space basis of $\mathfrak{a}$ in reduced row echelon form as defined in Proposition \ref{prop:rref}, and that conversely the Gr\"obner basis can be extracted from a basis of $\mathfrak{a}$ in reduced row echelon form.
	
	Given a Gr\"obner basis for $\mathfrak{a}$, reduction with respect to this basis is well-defined (c.f. \S\ref{sec:grobner-theorem}). Consider the set
	\[
		\{u_i - (u_i)_\red \mid u_i \text{ is a non-reduced monomial}\}.
	\]
	This set is linearly independent. Note that a polynomial $p$ lies in the span of this set precisely when $p_\red = 0$, and we know from Theorem \ref{thm:grobner-thm} that this is equivalent to $p \in \mathfrak{a}$. Hence $\{u_i - (u_i)_\red\}$ is a basis for $\mathfrak{a}$, and by design, it satisfies the conditions in Proposition \ref{prop:rref}.
	
	Conversely, suppose $\{u_i - \psi_i\}$ is a basis for $\mathfrak{a}$ in reduced row echelon form. Among these, consider the subset
	\[
		\{m_i - \phi_i\} \coloneqq \{u_i - \psi_i \mid u_i \text{ is not divisible by } u_j \text{ for any }j\neq i\}.
	\]
	We claim that these form a Gr\"obner basis for $\mathfrak{a}$. Let $\mathfrak{b}$ denote the ideal generated by $\{m_i -\phi_i\}$. Then $\mathfrak{b} \subset \mathfrak{a}$. Suppose there exists $p\in \mathfrak{a}$ which does not belong to $\mathfrak{b}$; we can assume that its highest monomial is minimal among all such $p$. But its leading monomial is equal to some $u_i$, which is equal to $rm_j s$ for some $m_j$ and some monomials $r,s$. Then if $a$ is the coefficient of $u_i$ in $p$, the polynomial
	\[
		p - ar(m_i -\phi_i)s \in \mathfrak{a}
	\]
	is lower than the leading monomial of $p$, and also doesn't belong to $\mathfrak{b}$---contradicting the minimality of $p$. Thus $\mathfrak{b} = \mathfrak{a}$.
	
	Suppose that $p\in \mathfrak{a}$ and we use congruences $m_i \equiv \phi_i$ to fully reduce it in some way, obtaining $p_\red \in \mathfrak{a}$. This means that none of the terms in $p_\red$ are divisible by any $m_i$, so none of the monomials $u_i$ appear in $p_\red$. But since $\{u_i - \psi_i\}$ is in reduced row echelon form and $p_\red \in \mathfrak{a}$, we must have $p_\red = 0$. Hence any element of $\mathfrak{a}$ reduces to zero. In particular, this implies that all overlaps are consistent. We conclude that $\{m_i - \phi_i\}$ is indeed a Gr\"obner basis for $\mathfrak{a}$ (and it is reduced by construction).
	
	We have given processes for converting between a Gr\"obner basis and reduced row echelon form for $\mathfrak{a}$. These constructions are inverse to one another, and hence the theorem follows from Proposition \ref{prop:rref}.
\end{proof}
We have shown that a unique Gr\"obner basis always exists for any ideal $\mathfrak{a}$, but the basis need not be finite. In the following section, we detail a computer algorithm for finding the basis in the event that it is finite.
