\section{Existence and uniqueness of Gr\"obner bases}\label{sec:rref}
We begin this section with an example. Let $X = \{x,y\}$ with $x<y$ and consider the ideal in $T(V)$ generated by $x^2 - y$. As a vector space, it is spanned by the set $\{u(x^2 - y)v \mid u,v\text{ monomials}\}$. We can organize these polynomials as rows of a matrix in the following way, where empty entries are zero:
\[
	\begin{blockarray}{cccccccccc}
	\cdots & x^2 y & x^3 & y^2 & yx & xy & x^2 & y & x & 1\\
	\begin{block}{[cccccccccc]}
	\ddots&\vdots& \vdots & \vdots& \vdots& \vdots& \vdots& \vdots& \vdots\\
	\cdots&&1 & 0 & -1 & 0 & 0 & 0 & 0 & 0\\
	\cdots&&1 & 0 & 0 & -1 & 0 & 0 & 0 & 0\\
	\cdots&& &  &  &  & 1 & -1 & 0 & 0\\
	\end{block}
	\end{blockarray}
\]
To use language from linear algebra, we see that this matrix is \emph{not} in reduced row echelon form: the distinct rows corresponding to $x^3 - yx$ and $x^3 - xy$ both have pivots in the $x^3$ position, reflecting the fact that we have an inconsistent overlap. As we will demonstrate in this section, the endeavor of producing a Gr\"obner basis is essentially equivalent to putting the above matrix in reduced row echelon form.

Let us first make precise what is meant by ``reduced row echelon form''---this phrase is usually used in the context of finite-dimensional vector spaces, but we are dealing with infinite-dimensional ones. We return to the general setting in which $X$ is any well-ordered set of variables, possibly infinite.

\begin{prop}[Reduced row echelon form]\label{prop:rref}
	Let $W$ be a linear subspace of $T(V)$. Then $W$ has a unique ordered basis of the form $\{u_i - \psi_i\}$ satisfying the following properties:
	\begin{enumerate}
		\item Each $u_i$ is a monomial and $\psi_i$ is lower than $u_i$.
		\item $u_i \neq u_j$ for all distinct $i,j$.
		\item For all $i$ and $j$, $\psi_i$ does not contain the monomial $u_j$.
	\end{enumerate}
\end{prop}
\begin{proof}
	For each monomial $u$ we shall define a set of monomials $B(u)$, which will be the set of \emph{pivots} not greater than $u$. We do this inductively. First, let $B(1) = \{1\}$ if $1 \in W$, and $B(1) = \emptyset$ otherwise.
	
	Let $u$ be a monomial, and suppose $B(v)$ has been defined for all $v < u$. Let $M_{<u}$ denote the set of monomials lower than $u$. If there exists a polynomial $\psi$ in the span of $M_{<u} \setminus \bigcup_{v<u} B(v)$ such that $u - \psi \in W$, then define
	\[
		B(u) = \{u\} \cup \bigcup_{v<u} B(v).
	\]
	Note that if there were two distinct choices of $\psi$, say $\psi_1$ and $\psi_2$, then $\psi_1 - \psi_2 \in W$ is also in the span of $M_{<u} \setminus \bigcup_{v<u} B(v)$. But then its leading monomial is lower than $u$ and should belong to $\bigcup_{v<u} B(v)$, a contradiction. Thus if $\psi$ exists, it is unique---call it $\psi_u$. If no such $\psi_u$ exists for $u$, define $B(u) = \bigcup_{v<u} B(v)$.
	
	This inductive process is valid because the monomials are well-ordered. Let $B = \bigcup_u B(u)$ be the set of all pivots. Our desired basis is then
	\[
		\{u - \psi_u \mid u \in B\},
	\]
	ordered by leading monomial. This set is linearly independent because each element has a different leading monomial. To see that it spans $W$, suppose $f\in W$. By subtracting some linear combination of $\{u - \psi_u\}$, we can obtain $f' \in W$ which contains only monomials not in $B$. But if $f' \neq 0$, this implies that its leading coefficient should be a monomial in $B$, contradiction. Thus $f' = 0$, meaning that $f$ is a linear combination of $\{u - \psi_u\}$ as desired. The uniqueness of this basis is clear from its construction.
\end{proof}
Now we will use this to prove an existence and uniqueness result for Gr\"obner bases, though we remind the reader that this is for a fixed well-ordering of $X$, the set of variables.
\begin{thm}\label{thm:exist+unique}
	Every two-sided ideal $\mathfrak{a}\in T(V)$ has a unique Gr\"obner basis.
\end{thm}
\begin{proof}
	To prove this, we will show that a Gr\"obner basis gives a vector space basis of $\mathfrak{a}$ in reduced row echelon form as defined in Proposition \ref{prop:rref}, and that conversely the Gr\"obner basis can be extracted from a basis of $\mathfrak{a}$ in reduced row echelon form.
	
	Given a Gr\"obner basis for $\mathfrak{a}$, reduction with respect to this basis is well-defined (c.f. \S\ref{sec:grobner-theorem}). Consider the set
	\[
		\{u_i - (u_i)_\red \mid u_i \text{ is a non-reduced monomial}\}.
	\]
	This set is linearly independent because each element has a different leading monomial. Note that a polynomial $p$ lies in the span of this set precisely when $p_\red = 0$, and we know from Theorem \ref{thm:grobner-thm} that this is equivalent to $p \in \mathfrak{a}$. Hence $\{u_i - (u_i)_\red\}$ is a basis for $\mathfrak{a}$, and by design, it satisfies the conditions in Proposition \ref{prop:rref}.
	
	Conversely, suppose $\{u_i - \psi_i\}$ is a basis for $\mathfrak{a}$ in reduced row echelon form. Consider the subset of this basis given by
	\[
		\{m_i - \phi_i\} \coloneqq \{u_i - \psi_i \mid u_i \text{ is not divisible by } u_j \text{ for any }j\neq i\}.
	\]
	We claim that this is a Gr\"obner basis for $\mathfrak{a}$. Let $\mathfrak{b}$ denote the ideal generated by $\{m_i -\phi_i\}$. Then $\mathfrak{b} \subseteq \mathfrak{a}$. Conversely, suppose there exists $p\in \mathfrak{a}$ which does not belong to $\mathfrak{b}$; we can assume that its highest monomial is minimal among all such $p$. As a linear combination of $\{u_i - \psi_i\}$, its leading monomial is equal to some $u_i$, as the leading monomials in the basis are all distinct. This $u_i$ is equal to $rm_j s$ for some $m_j$ and some monomials $r,s$. Then if $a$ is the coefficient of $u_i$ in $p$, the polynomial
	\[
		p - ar(m_i -\phi_i)s \in \mathfrak{a}
	\]
	is lower than the leading monomial of $p$, and also doesn't belong to $\mathfrak{b}$---contradicting the minimality of $p$. Thus $\mathfrak{b} = \mathfrak{a}$.
	
	Suppose that $p\in \mathfrak{a}$ and we use congruences $m_i \equiv \phi_i$ to fully reduce it in some way, obtaining $p_\red \in \mathfrak{a}$. This means that none of the terms in $p_\red$ are divisible by any $m_i$, so none of the monomials $u_i$ appear in $p_\red$. But since $\{u_i - \psi_i\}$ is in reduced row echelon form and $p_\red \in \mathfrak{a}$, we must have $p_\red = 0$. Hence any element of $\mathfrak{a}$ reduces to zero. In particular, this implies that all overlaps are consistent. Condition \eqref{item:reduced-1} of Definition \ref{def:grobner} is immediate from our construction, and condition \eqref{item:reduced-2} follows from the definition of reduced row echelon form. We conclude that $\{m_i - \phi_i\}$ is indeed a Gr\"obner basis for $\mathfrak{a}$.
	
	We have given processes for converting between a Gr\"obner basis and reduced row echelon form for $\mathfrak{a}$. These constructions are inverse to one another, and hence the theorem follows from Proposition \ref{prop:rref}.
\end{proof}
We have shown that a unique Gr\"obner basis always exists for any ideal $\mathfrak{a}$, but the basis need not be finite. In the following section, we detail a computer algorithm for finding the basis in the event that it is finite. Unfortunately, we do not have a criterion for finiteness---other than running the program and hoping that it terminates!