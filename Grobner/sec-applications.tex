\newcommand{\g}{\mathfrak{g}}
\renewcommand{\a}{\mathfrak{a}}
\newcommand{\gr}{\operatorname{gr}}
\newcommand{\Sym}{\operatorname{Sym}}

\section{Universal enveloping algebras}\label{sec:applications}

Let $\g$ be a Lie algebra over $k$, that is, a $k$-vector space equipped with a linear map
\begin{equation}
\label{eqn:bracket}
[-,-]\colon\g\otimes\g\to\g
\end{equation}
satisfying the axioms
\begin{equation*}
\begin{aligned}
[x,x]&=0,\\
[x,[y,z]]+[z,[x,y]]+[y,[z,x]]&=0,
\end{aligned}
\qquad
\begin{aligned}
&\text{(alternativity)}\\
&\text{(the Jacobi identity)}
\end{aligned}
\end{equation*}
for all $x,y,z\in\g$. Note that the alternativity axiom implies the identity
\begin{equation}
\hspace{8.5em}[x,y]+[y,x]=0,\qquad\text{(anti-commutativity)},
\end{equation}
for all $x,y\in\g$, and that the two are equivalent if the characteristic of $k$ is not $2$ (indeed, consider $[x+y,x+y]$).

Now, consider the tensor algebra $T(\g)$. We may express the ``bracket'' \eqref{eqn:bracket} of $\g$ in terms of commutators in $T(\g)$ by quotienting $T(\g)$ by the two-sided ideal $\a$ generated by the elements
\begin{equation}
\label{eqn:ueaideal}
\{xy-yx-[x,y]\mid x,y\in\g\}.
\end{equation}
This quotient $U(\g):=T(\g)/\a$ is known as the ``universal enveloping algebra'' of $\g$; we would like to use the methods of \S\ref{sec:grobner-theorem} to find an explicit basis for this algebra.

We begin by fixing a well-ordered basis $\{x_i:i\in I\}$ of $\g$, and introducing the following notational conveniences, which will be useful shortly:
\begin{notation}
\begin{enumerate}
\item Write $i>j$ if $x_i>x_j$ for some $i,j\in I$.
\item For any $x=\sum_{j\in I}a_jx_j\in\g$ and $i\in I$, put
\begin{equation*}
x_{<i}:=\sum_{\substack{j\in I\\j<i}}a_jx_j,
\end{equation*}
and similarly for $x_{\le i}$, $x_{=i}$, $x_{>i}$, and $x_{\ge i}$. We then have the identities
\begin{equation*}
x=x_{<i}+x_{=i}+x_{>i}=x_{\le i}+x_{>i}=x_{<i}+x_{\ge i}.
\end{equation*}

\end{enumerate}
\end{notation}
Now, by bilinearity of the bracket, $\a$ is equal to the two-sided ideal generated by the subset of \eqref{eqn:ueaideal} where $x$ and $y$ are taken to be basis elements, that is,
\begin{equation}
\label{eqn:ueabasis}
\{x_ix_j-x_jx_i-[x_i,x_j]\mid i,j\in I\}.
\end{equation}
In fact, more is true:
\begin{lem}
The set \eqref{eqn:ueabasis} is a (reduced) Gr\"obner basis for $\a$.
\end{lem}
\begin{proof}
For each $i,j\in I$ with $i>j$, set $m_{i,j}=x_ix_j$ and $\phi_{i,j}=x_jx_i+[x_i,x_j]$. The first and last two properties of Definition~\ref{def:grobner} are then immediate; it remains to verify property \eqref{item:grobnerconsistent}, that all overlaps in \eqref{eqn:ueabasis} are (weakly) consistent. The only overlaps in this family of congruences are those of the form $m_{i,j}x_\ell=x_im_{j,\ell}$, where $i>j>\ell$. Given any such overlap, we have sequences of reductions
\begin{align*}
m_{i,j}x_\ell&\equiv(x_jx_i+[x_i,x_j])x_\ell\\
&=x_j(x_ix_\ell)+([x_i,x_j]_{\le\ell}+[x_i,x_j]_{>\ell})x_\ell\\
&\equiv x_j(x_\ell x_i+[x_i,x_\ell])+[x_i,x_j]_{\le\ell}x_\ell+(x_\ell[x_i,x_j]_{>\ell}+[[x_i,x_j]_{>\ell},x_\ell])\\
&\equiv(x_jx_\ell)x_i+x_j([x_i,x_\ell]_{<j}+[x_i,x_\ell]_{\ge j})+[x_i,x_j]_{\le\ell}x_\ell\\
&\qquad+(x_\ell[x_i,x_j]_{>\ell}+[[x_i,x_j]_{>\ell},x_\ell])\\
&\equiv(x_\ell x_j+[x_j,x_\ell])x_i+([x_i,x_\ell]_{<j}x_j+[x_j,[x_i,x_\ell]_{<j}])+x_j[x_i,x_\ell]_{\ge j}\\
&\qquad+[x_i,x_j]_{\le\ell}x_\ell+(x_\ell[x_i,x_j]_{>\ell}+[[x_i,x_j]_{>\ell},x_\ell])\\
&\equiv x_\ell x_jx_i+[x_j,x_\ell]_{\le i}x_i+(x_i[x_j,x_\ell]_{>i}+[[x_j,x_\ell]_{>i},x_i])\\
&\qquad+([x_i,x_\ell]_{<j}x_j+[x_j,[x_i,x_\ell]_{<j}])+x_j[x_i,x_\ell]_{\ge j}\\
&\qquad+[x_i,x_j]_{\le\ell}x_\ell+(x_\ell[x_i,x_j]_{>\ell}+[[x_i,x_j]_{>\ell},x_\ell]),\\
x_im_{j,\ell}&\equiv x_i(x_\ell x_j+[x_j,x_\ell])\\
&=(x_ix_\ell)x_j+x_i([x_j,x_\ell]_{<i}+[x_j,x_\ell]_{\ge i})\\
&\equiv(x_\ell x_i+[x_i,x_\ell])x_j+([x_j,x_\ell]_{<i}x_i+[x_i,[x_j,x_\ell]_{<i}])+x_i[x_j,x_\ell]_{\ge i}\\
&\equiv x_\ell(x_ix_j)+[x_i,x_\ell]_{\le j}x_j+(x_j[x_i,x_\ell]_{>j}+[[x_i,x_\ell]_{>j},x_j])\\
&\qquad+([x_j,x_\ell]_{<i}x_i+[x_i,[x_j,x_\ell]_{<i}])+x_i[x_j,x_\ell]_{\ge i}\\
&\equiv x_\ell(x_jx_i+[x_i,x_j])+[x_i,x_\ell]_{\le j}x_j+(x_j[x_i,x_\ell]_{>j}+[[x_i,x_\ell]_{>j},x_j])\\
&\qquad+([x_j,x_\ell]_{<i}x_i+[x_i,[x_j,x_\ell]_{<i}])+x_i[x_j,x_\ell]_{\ge i}\\
&\equiv x_\ell x_jx_i+x_\ell[x_i,x_j]_{\ge \ell}+([x_i,x_j]_{<\ell}x_\ell+[x_\ell,[x_i,x_j]_{<\ell}])\\
&\qquad+[x_i,x_\ell]_{\le j}x_j+(x_j[x_i,x_\ell]_{>j}+[[x_i,x_\ell]_{>j},x_j])\\
&\qquad+([x_j,x_\ell]_{<i}x_i+[x_i,[x_j,x_\ell]_{<i}])+x_i[x_j,x_\ell]_{\ge i}.
\end{align*}
In both cases, we have made extensive use of the bilinearity of the bracket. Now, observe that these reduced polynomials are equivalent if and only if
\begin{align*}
[[x_j,x_\ell]_{>i},x_i]+&[x_j,[x_i,x_\ell]_{<j}]+[[x_i,x_j]_{>\ell},x_\ell]\\
&=[x_\ell,[x_i,x_j]_{<\ell}]+[[x_i,x_\ell]_{>j},x_j]+[x_i,[x_j,x_\ell]_{<i}],
\end{align*}
as all other terms of $(m_{i,j}x_\ell)_\red$ and $(x_im_{j,\ell})_\red$ cancel. Applying anti-commutativity and alternativity, we see that this is equivalent to the identity
\begin{equation*}
[x_i,[x_j,x_\ell]]+[x_j,[x_\ell,x_i]]+[x_\ell,[x_i,x_j]]=0.
\end{equation*}
This is just the Jacobi identity.
\end{proof}

Thus, by Corollary~\ref{cor:grobner}, a basis of $U(\g)$ is given by the monomials in $T(\g)$ that are not divisible by any $x_ix_j$ with $i>j$. Equivalently:
\begin{thm}
The set
\begin{equation}
\label{eqn:ugbasis}
\{1\}\cup\left\{x_{i_1}^{e_1}x_{i_2}^{e_2}\cdots x_{i_n}^{e_n}\mid n\ge 1,\text{ }i_1<i_2<\cdots<i_n,\text{ }e_1,\cdots,e_n\ge 1\right\}
\end{equation}
is a basis for $U(\g)$.
\end{thm}
Morally, this result says that a universal enveloping algebra looks very much like a \emph{commutative} polynomial ring. We make this precise as follows: define the \emph{symmetric algebra} on $\g$ to be the quotient
\begin{equation*}
\Sym(\g)=T(\g)/\langle x\otimes y-y\otimes x\mid x,y\in\g\rangle.
\end{equation*}
Then it is easy to see that the choice of basis $\{x_i\}$ of $\g$ furnishes an isomorphism
\begin{equation}
\label{eqn:symisopoly}
\Sym(\g)\cong k[x_i\mid i\in I],
\end{equation}
where the latter denotes the commutative polynomial algebra generated by the basis elements $\{x_i\}$. Now, endow $U(\g)$ with the structure of a filtered algebra by defining $F_m\subset U(\g)$ to be the image of $\bigoplus_{0\le n\le m}\g^{\otimes n}\subset T(\g)$ under the quotient map $T(\g)\to U(\g)$, for each $m\ge 0$. Equivalently, $F_m$ is the span of all monomials of degree at most $m$. This gives a filtration
\begin{equation*}
\{0\}\subset F_0\subset F_1\subset\cdots\subset F_m\subset\cdots\subset U(\g)
\end{equation*}
satisfying $F_nF_m\subset F_{n+m}$ for each $n,m\ge 0$. It follows that the vector space
\begin{equation*}
\gr U(\g)=\bigoplus_{n\ge 0}F_n/F_{n-1},
\end{equation*}
where we have put $F_{-1}=\{0\}$, has the structure of an associative algebra; we call it the \emph{associated graded algebra} of $U(\g)$.

Now, observe that for any relation
\begin{equation*}
x_ix_j-x_jx_i-[x_i,x_j]=0
\end{equation*}
in $U(\g)$, the bracket $[x_i,x_j]$ is an element of $F_1$, whereas the commutator $x_ix_j-x_jx_i$ is an element of $F_2$, but not of $F_1$. Thus, in $F_2/F_1\subset\gr U(\g)$, the relation $x_ix_j-x_jx_i=0$ holds. This gives rise to a factorization
\begin{equation}
\label{eqn:factorization}
\begin{tikzcd}[column sep=tiny,row sep=small]
T(\g)\arrow[rr,two heads]\arrow[rd,two heads]&&\gr U(\g)\\
&\Sym(\g).\arrow[ru,dashed]&
\end{tikzcd}
\end{equation}
Now, \eqref{eqn:symisopoly} shows that \eqref{eqn:ugbasis} is also a basis for $\Sym(\g)$, which is evidently mapped bijectively to the corresponding basis elements of $U(\g)$ by the dashed map. Thus, our results on Gr\"obner bases have given a slick proof of the following well-known result:

\begin{cor}[Poincar\'e--Birkhoff--Witt theorem \cite{cohn}]
The dashed map \eqref{eqn:factorization} is a (canonical) isomorphism of commutative algebras.
\end{cor}

%%% Local Variables:
%%% mode: latex
%%% TeX-master: t
%%% End:
