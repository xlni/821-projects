\section{Implementation of the simplification algorithm}
\label{sec:implementation}

The results from the previous section allow us to devise a process by which we can
turn a finitely-generated, two-sided ideal $\mathfrak{a}$ into a Gr\"obner basis.
Let us consider an abridged description first.

\begin{alg}
    Let $\mathfrak{a} = (p_1,\ldots,p_n)$ be a two-sided ideal. Let $m_i$ be the 
    leading (w.r.t. deglex) term of each $p_i$ (we assume our polynomials are monic).
    If $m_i \mid m_j$, we can reduce $m_j$ with respect to $p_i$, and replace $p_j$
    with the resulting polynomial; this removes extraneous relations.
    
    Now that we have a set of polynomials with relatively-prime leading terms, we need
    to correct overlaps. For each pair of polynomials, we say there is an overlap 
    if there are $u$ and $v$ such that $um_i = m_j v$, for $v, u$ having degree smaller 
    than $m_i$ and $m_j$, respectively. Let $q_i = u(m_i - p_i)$ and $q_j = (m_j - p_j)v$.
    We reduce these polynomials with respect to $\mathfrak{a}$; if $q_{i,\red} = q_{j,\red}$,
    this overlap produces consistent results, so we move on. Otherwise, we introduce
    $q_{i,red} - q_{j,red}$ as a new generator of $\mathfrak{a}$. 
    We then repeat this process with the enlarged set of generators 
    until we reach a fixed point.
    
    Finally, we reduce the $\phi_i$ component of each generator with respect to the 
    ideal; this gives us our Gr\"obner basis.
\end{alg}

We will now dive into a detailed description of the algorithm. We assume that the
reader of this section is familiar with basic low-level programing, algorithms, and
big-oh notation.

\begin{rem}
    We will give a primarily high-level overview of the implementation, though since some
    design choices involve avoiding memory allocation whenever possible, which
    affects, for example, complexity of polynomial arithmetic. 
    
    We will denote data structures and the names of instances thereof in typewriter font:
    for example, \verb|p| is a \verb|Poly|. All algorithms are described in
    \verb|C|-like pseudo-code.
\end{rem}

Let us begin by defining the necessary data structures.

\begin{defn}
    A \verb|Rational| represents a rational number as a pair of signed machine-sized 
    integers\footnote{A machine-sized integer is an integer large enough to hold a pointer on
    the host system. For most modern systems, this is eight bytes, or 64 bits.}. Arithmetic
    operations are a fixed number of arithmetic instructions, though testing for equality
    may require running Euclid's algorithm, which requires a variable but bounded number 
    of instructions.
    
    A \verb|Monomial| represents a generator of the algebra $T(V)$ as a heap-allocated
    list of unsigned bytes, where the byte value $n$ represents the variable $x_n$ 
    (for a total of 256 variables). \verb|Monomials| are ordered by deglex,
    and multiplication can be performed in linear time (though it may invoke memory
    allocation).

    A \verb|Poly| represents an element of the algebra $T(V)$ as a heap-allocated
    list of pairs of \verb|Rational|s and \verb|Monomial|s. This object maintains the
    invariant that pairs are sorted by the ordering on \verb|Monomial|s. Addition
    of a \verb|Monomial| with some \verb|Rational| coefficient takes $O(\log n)$ time,
    where $n$ is the number of nonzero terms of the polynomial, implemented with binary
    search. In-place addition of two polynomials $p + q$ takes $O(m\log (n+m))$ time, 
    where $n$ and $m$ are the number of nonzero terms of these polynomials, and
    multiplication takes $O(nm)$ time. In-place multiplication by a single monomial
    is $O(n)$, disregarding memory allocation, implemented by concatenating the monomial
    to each term in the polynomial; note that this preserves the order invariant.
    
    This approach, unlike a hashtable or a binary
    search tree, avoids having to perform hashing or indirection. It also has the
    advantage that accessing the maximal element takes $O(1)$ time. We could have
    also represented polynomials as a list of \verb|Rational|s, and assigned each
    \verb|Monomial| an index corresponding to the ordering, but we would still have to pay
    the cost of converting a \verb|Monomial| to index form and back later on, in addition
    to performing additional allocations.
    
    An \verb|Ideal| will consist of a heap-allocated list of \verb|Relation|s,
    each consisting of a \verb|Monomial| \verb|m| and a \verb|Poly| \verb|phi|, 
    such that \verb|m| is greater than all terms in \verb|phi|. A \verb|Relation|
    may be constructed from a \verb|Poly| by subtracting off the leading term (due
    to the ordering invariant, this may be done in $O(1)$ time).
\end{defn}

Our goal is to implement an algorithm which takes a list of \verb|Poly|s, representing the
generators of $\mathfrak{a}$, and produce a Gr\"obner basis. This is performed in three
steps: expansion of relations by divisibility, resolution of overlaps, and reduction
of the $\phi_i$. Before
we can describe this algorithm, we need to describe reduction of a \verb|Poly| with
respect to an \verb|Ideal|.

\begin{alg}
    Reduction with respect to an ideal is defined in three steps: first, reduce
    a monomial with respect to a relation; then, reduce a polynomial with respect
    to a relation; finally, reduce repeatedly to reduce against an ideal. We will
    describe the first algorithm in detail.
\begin{verbatim}
function reduce(mon: Monomial, rel: Relation) {
    let result = 1
    let index = 0
    while true {
        // run through mon looking for occurences of rel.m
        if mon[index ... index + degree(rel.m)] == rel.m {
            // if the range [index, index + degree(rel.m)) 
            // in the variables of mon is rel.m, replace it.
            result *= rel.phi
            index += degree(rel.m)
        } else {
            // otherwise, append the next variable
            result *= mon[index]
            index += 1
        }
        
        if index + degree(rel.m) > degree(mon) {
            // if rel.m can't fit into what's left, we're done
            result *= mon[index ...]
            break
        }
    }
    
    return result
}
\end{verbatim}
    This takes, at worst, $O(n^d)$ time, where $d$ is the
    degree of \verb|mon| and $n$ is the number of terms in \verb|rel.phi|,
    in the case where the degree of \verb|rel.m| is $1$ and \verb|mon| is all one
    variable. As the length of \verb|rel.m| increases, the probability of a long-enough
    substring of \verb|mon| being replaceable will decrease: if there are $v$ variables
    and \verb|rel.m| is of degree $k$ and there are $v$ variables to choose from,
    the probability of a substring matching \verb|rel.m| is $v^{-k}$. Ignoring
    the fact that these substrings may overlap, this comes out to having to perform
    $(d-k+1)v^{-k}$ substitutions on average, each of which costs $O(n)$. This gives
    an average bound of $O(n^{(d-k+1)v^{-k}})$. When $k=1$, this gives our original
    worst-case analysis.
    
    The other two steps can be performed by repeated application of this algorithm:
    we can reduce a \verb|Poly| with respect to a \verb|Relation| by by applying 
    this algorithm to each constituent monomial, and we can reduce a \verb|Poly| with
    respect to an \verb|Ideal| by repeatedly applying each relation until none of them
    have any further effect. This procedure will terminate, since in each step
    we reduce the the maximal monomial with respect to their ordering.
\end{alg}

Next, we describe how to simplify an ideal with respect to itself, the first
step in finding a Gr\"obner basis.

\begin{alg}
    The following algorithm modifies an \verb|Ideal| in-place, by reducing a \verb|Relation|
    with respect to another if the \verb|Monomial| of the first divides the \verb|Monomial|
    of the second.
\begin{verbatim}
function simplify(ideal: Ideal) {
    while true {
        // keep track of whether this iteration 
        // did anything
        let changed = false
        for (a, b) in ideal x ideal {
            // attempt to reduce b.m by a if a.m divides b.m
            if a == b or b.m == 0 or degree(a.m) >= degree(b.m) or 
                not a.m | b.m {
                continue
            }
            
            // we're going to change
            changed = true
            
            let reduced = reduce(b.m, a)
            b.phi -= reduced
            
            // if b.phi is zero, we're totally done with it,
            // because this just indicates that the ideal
            // kills b.m
            if b.phi == 0 {
                continue
            }
            
            // remove the highest term from b.phi
            // and make it into the new b.m
            let (c, m) = pop_highest_term(b.phi)
            b.phi /= c
            b.m = m
        }
        
        // we've found a fixed point
        if !changed {
            break
        }
    }
}
\end{verbatim}
\end{alg}

Finally, we produce a Gr\"obner basis by expanding overlapping relations as
necessary.

\begin{alg}
    The following algorithm modifies an \verb|Ideal| in-place, introducing a new
    \verb|Relation| any time the overlap of two \verb|Relation|s reduces to two
    distinct \verb|Poly|s, where the \verb|Relation| declares these \verb|Poly|s to
    be equal.
\begin{verbatim}
function expand_overlaps(ideal: Ideal, iterations: int) {
    // each iterations will produce new relations; we 
    // only add them to the ideal at the end of an iteration
    let new_relations = new List
    // since we're running through the ideal over and over,
    // we need to make sure we don't check the same pair twice.
    let checked_pairs = new HashTable
    while true {
        for (a, b) in ideal x ideal {
            if (a, b) in checked_pairs {
                continue
            } else {
                checked_pairs += (a, b)
            }
            
            for k in 0 ... degree(a.m) {
                if b.m.starts_with(a.m[k ...]) {
                    // we've found an overlap
                    
                    // compute m_i * u and v * m_j
                    let p1 = a.m[k ...] * b2.phi
                    let p2 = a.phi * b2.m[degree(a.m) - k ...]
                    
                    // reduce the polynomials
                    p1 = reduce(p1, ideal)
                    p2 = reduce(p2, ideal)
                    
                    if p1 != p2 {
                        // the reduced forms are inconsistent, 
                        // so we need to add a new relation
                        new_relations += relation(p1 - p2)
                    }
                }
            }
        }
        
        if empty(new_rels) {
            // no more relations were added, so everything
            // is consistent and we're done
            break
        } else {
            // otherwise, add the new relations
            // and check again
            ideal += new_rels
            new_rels = new List
        }
    }
}
\end{verbatim}
\end{alg}

Taking these three algorithms together, we can compute Gr\"obner bases. We have written
a program which implements these algorithms, which can be be run from any Athena
terminal with the command \verb|add mcyoung| followed by \verb|grobner|. Running
\verb|grobner help| will show available options. 

\begin{example}
    Use \verb|grobner| to compute a basis for $\mathfrak{a}=(x^2 - y^2)$:
\begin{verbatim}
$ grobner basis 'x^2 - y^2'
y^2 - x^2
yx^2 - x^2y
\end{verbatim}

    \noindent
    Use \verb|grobner| to to reduce $3x^4 + \frac{1}{3} y^4$ with respect to
    $\mathfrak{a}$:
\begin{verbatim}
$ grobner reduce 'x^2 - y^2' --poly '1/3y^4 + 3x^4'
a = (y^2 - x^2, yx^2 - x^2y)
1/3y^4 + 3x^4 = 10/3x^4 (mod a)
\end{verbatim}    
\end{example}




