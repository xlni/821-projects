\section{Computations}\label{sec:examples}
Now we will put the machinery of \S\ref{sec:implementation} to use by computing Gr\"obner bases for the examples provided in the project description. Afterwards, in \S\ref{sec:assoc-graded}, we will quantify the size of $T(V)/\mathfrak{a}$ in each case, using Hilbert functions and Poincar\'e series.

\subsection{Examples from the assignment}\label{sec:assignment-examples}
For all of the following examples, our set of variables $X$ will be $\{x,y\}$, with the order $x < y$.
\subsection*{Part I}

The first part of the assignment was to determine whether a given polynomial $p$ is contained in the two-sided ideal generated by a given relation $f$, in five different cases. For this we will use the command
\begin{verbatim}
	$ grobner reduce 'f' -p 'p'
\end{verbatim}
from \S\ref{sec:implementation}, which first computes a Gr\"obner basis for $\mathfrak{a} = (f)$, and then prints the reduction outcome of $p$ using that basis. By Theorem~\ref{thm:grobner-thm}, we have $p\in\mathfrak{a}$ if and only if $p_\red=0$.

\begin{enumerate}
	\item If $f = x^2y$ and $p = x^4 y x - x^3 y x^2 + xyx^4$, then $p_\red = xyx^4$.
	\item If $f = yx - xy - 1$ and $p = yx^2 - 2x$, then $p_\red = x^2 y$.
    \item If $f = yx - 2xy$ and $p = yx^2 - xyx$, then $p_\red = 2x^2y$.
    \item If $f = y^2 - x^2$ and $p = yx^2 - x^2y$, then $p_\red = 0$.
    \item If $f = y^2 - x + 1$ and $p=y^3 - xy$, then $p_\red = - y$.
\end{enumerate}

\subsection*{Parts II, III}

The second and third parts of the assignment were to use the program of \S\ref{sec:implementation} to (attempt to) compute Gr\"obner bases for the ideals $\mathfrak{a}$ generated by various sets of relations $\{f_i\}$. For this we will use the command
\begin{verbatim}
	$ grobner basis 'f1' 'f2' ...
\end{verbatim}
from \S\ref{sec:implementation}. In the examples where we obtain a finite Gr\"obner basis, we also give an explicit vector space basis of $T\langle x, y\rangle / \mathfrak{a}$ for use in \S\ref{sec:assoc-graded}.

\begin{enumerate}
	\item For the set $\{y^2-x^2\}$, we obtain the Gr\"obner basis
	\begin{equation*}
		\{y^2-x^2,yx^2-x^2y\}.
	\end{equation*}
    By Corollary~\ref{cor:grobner}, a basis for the quotient algebra $T\langle x,y\rangle/\mathfrak{a}$ is therefore given by
    \begin{equation}
      \label{eqn:basisone}
    \left\{[x^i(yx)^jy^\ell]\mid i,j\ge 0,\text{ }\ell\in\{0,1\}\right\},
    \end{equation}
    as these are the equivalence classes of monomials in $T\langle x,y\rangle$ not divisible by any of the monomials $m_i\in\{y^2,yx^2\}$ corresponding to this Gr\"obner basis.
	\item For the set $\{y^2-x,yx-y\}$, we obtain the Gr\"obner basis
    \begin{equation*}
      \{y^2 - x,yx - y,xy - y,x^2 - x\}.
    \end{equation*}
    As before, the set of monomials $m_i$ is given by $\{y^2,yx,xy,x^2\}$, and therefore a basis for the quotient algebra is given by
    \begin{equation*}
      \{[1],[x],[y]\}.
    \end{equation*}
    \item\label{item:infinite-case-yxy} For the set $\{yxy-xyx\}$, the program does not terminate. However, one can check by hand that the set
    \[
    	\{yxy - xyx\} \cup \{yx^{i+1}yx = xyx^2 y^i \mid i \geq 1\}
    \]
    is a Gr\"obner basis. We check the overlap $(yx^{i+1}yx)x^j yx = yx^{i+1}(yx^{j+1}yx)$ as an example and leave the rest to the reader. 
    \begin{align*}
    	(yx^{i+1}yx)x^j yx &\equiv xyx^2 y^i x^j yx\\
    	yx^{i+1}(yx^{j+1}yx) &\equiv (yx^{i+2}yx)xy^j\\
    	&\equiv xyx^2 y^i (yxy) y^{j-1}\\
    	&\equiv xyx^2 y^i x (yxy) y^{j-2}\\
    	&\hspace{6pt} \vdots\\
    	&\equiv xyx^2 y^i x^j yx.
    \end{align*}
	Thus this overlap is weakly consistent.
	\item For the set $\{yxy-x\}$, we obtain the Gr\"obner basis
	\begin{equation*}
		\{yxy-x,yx^2-x^2y\}.
	\end{equation*}
    The set of monomials $m_i$ is given by $\{yxy,yx^2\}$, and therefore a basis for the quotient algebra is given by
    \begin{equation*}
      \left\{[x^iy^jx^\ell]\mid i,j\ge 0,\text{ }\ell\in\{0,1\}\right\}.
    \end{equation*}
	\item For the set $\{y^2-ax^2-bx\}$, where $a,b\in k$, we observe by testing various values of $a$ and $b$ that the Gr\"obner bases look like
      \begin{equation*}
\begin{cases}
\{y^2-ax^2-bx,yx^2-x^2y-\frac{b}{a}yx+\frac{b}{a}xy\}&\text{if }a\ne 0,\\
\{y^2-bx,yx-xy\}&\text{if }a=0\text{ and }b\ne 0,\\
\{y^2\}&\text{if }a=b=0.\\
\end{cases}
        \end{equation*}
      This is easily proven to hold in general. The set of monomials $m_i$ is therefore given by $\{y^2,yx^2\}$, $\{y^2,yx\}$, and $\{y^2\}$, respectively, in these three cases. Thus, a basis for the quotient algebra is given by
      \begin{equation*}
\begin{cases}
\eqref{eqn:basisone}&\text{if }a\ne 0,\\
\left\{[x^iy^\ell]\mid i\ge 0,\text{ }\ell\in\{0,1\}\right\}&\text{if }a=0\text{ and }b\ne 0,\\
\left\{[x^{i_1}yx^{i_2}y\cdots yx^{i_n}]\mid n\ge 1,\text{ }i_1,i_n\ge 0,\text{ }i_2,\ldots,i_{n-1}\ge 1\right\}&\text{if }a=b=0.\\
\end{cases}
      \end{equation*}
	\item For the set $\{yxyx-xyxy\}$, the program does not terminate. However, as in \eqref{item:infinite-case-yxy}, one can check by hand that the set
	\[
		\{yxyx - xyxy\} \cup \{yx^{i+1}yxy - xyxy^2 x^i \mid i \geq 1\}
	\]
	is a Gr\"obner basis.
    \item The project description suggests that we consider when $X = \{x,y,z\}$ (with $x<y<z$) and we have a set of relations
\begin{equation*}
\{yx-\phi_1,zx-\phi_2,zy-\phi_3\},
\end{equation*}
where $\phi_1$ (resp.\ $\phi_2$, $\phi_3$) is lower than $yx$ (resp.\ $zx$, $zy$). In \S\ref{sec:applications}, we prove that these relations are consistent in the case where they describe the universal enveloping algebra of a $3$-dimensional Lie algebra. For instance, if
\begin{equation*}
\phi_1=xy+z,\quad\phi_2=xz-2x,\quad\phi_3=yz+2y,
\end{equation*}
then consistency is achieved; these relations correspond to the special linear Lie algebra $\mathfrak{sl}_2(k)$. A basis for the quotient algebra is given in Theorem~\ref{thm:ugbasis}.
\end{enumerate}

\subsection{The associated graded algebra}\label{sec:assoc-graded}
Given an ideal $\mathfrak{a} \subset T(V)$, we may also ask about the ``size'' of $T(V)/\mathfrak{a}$. The dimension alone is too crude of a measurement, as $T(V)/\mathfrak{a}$ will almost always be infinite-dimensional. We can get a more refined sense of size by considering its associated graded algebra, which we construct now.

The tensor algebra $T(V)$ is naturally graded by degree. As $\mathfrak{a}$ need not be homogeneous, in general we do not get a grading of the quotient $T(V)/\mathfrak{a}$. However, if we take the grading filtration on $T(V)$ defined by
\[
	F_s T(V) = \bigoplus_{i=0}^s V^{\otimes i}
\]
with $F_{-1} = 0$, then the quotient map $T(V) \to T(V)/\mathfrak{a}$ induces a filtration on $T(V)/\mathfrak{a}$:
\[
	F_s(T(V)/\mathfrak{a}) = \img (F_s T(V)).
\]
This filtration respects multiplication in the sense that $F_s F_t \subset F_{s+t}$ for all $s,t \geq 0$. Hence if we take
\[
	\gr_s(T(V)/\mathfrak{a}) \coloneqq F_s/F_{s-1},
\]
then $\gr (T(V)/\mathfrak{a})$ is an algebra: the \emph{associated graded algebra} for the filtration. Using this, we can give a more nuanced description of the size of $T(V)/\mathfrak{a}$.
\begin{defn}
	Let $A$ be a graded vector space. Its \emph{Hilbert function} is defined to be
	\[
		h(d) \coloneqq \dim_k (A)_d
	\]
	and its \emph{Poincar\'e series} is defined to be
	\[
		H(t) \coloneqq \sum_{d\geq 0} h(d) t^d.
	\]
\end{defn}
For $\gr (T(V) / \mathfrak{a})$, we already have the tools to compute $h$ (and thus $H$ as well):
\begin{lem}
	$h(d)$ is the number of reduced monomials of degree $d$.
\end{lem}
\begin{proof}
	This is an immediate consequence of Corollary \ref{cor:grobner}.
\end{proof}
Now we compute the Hilbert function and Poincar\'e series for each of the examples in Parts II, III of \S\ref{sec:assignment-examples} where we obtained a finite Gr\"obner basis. We will write out only the Poincar\'e series---from them, the Hilbert functions are easily seen.
\begin{enumerate}
	\item Of the monomials $\left\{x^i(yx)^jy^\ell\mid i,j\ge 0,\text{ }\ell\in\{0,1\}\right\}$, exactly $d+1$ of them have degree $d$. The Poincar\'e series is
	\begin{equation}
		H(t) = 1 + 2t + 3t^2 + 4t^3 + \cdots.\label{eq:poincare-y2-x2}
	\end{equation}
	Note that $H - tH = 1 + t + t^2 + t^3 + \cdots$ is the series expansion for $1/(1-t)$, so we can write $H$ in closed form as (the series expansion of)
	\[
		H(t) = \frac{1}{(1-t)^2}.
	\]
	\item The only reduced monomials are $\{1,x,y\}$ so the Poincar\'e series is
	\[
		H(t) = 1 + 2t.
	\]
	\setcounter{enumi}{3}
	\item There are $2d$ monomials of the form $\left\{x^iy^jx^\ell\mid i,j\ge 0,\text{ }\ell\in\{0,1\}\right\}$ which have degree $d$, for $d \geq 1$. Hence
	\[
		H(t) = 1 + 2t + 4t^2 + 6t^3 + \cdots
	\]
	which can be written in closed form as
	\[
		H(t) = \frac{1+t^2}{(1-t)^2}.
	\]
	\item We have three sub-cases in this scenario:
	\begin{enumerate}
		\item If $a\neq 0$, then $H(t)$ is \eqref{eq:poincare-y2-x2}.
		\item If $a = 0$ and $b\neq 0$ then there are $2$ reduced monomials of degree $d$ for $d \geq 1$, so
		\[
			H(t) = 1 + 2t + 2t^2 + 2t^3 + \cdots
		\]
		which can be written as
		\[
			H(t) = \frac{1+t}{1-t}.
		\]
		\item If $a = b = 0$, then the reduced monomials are monomials which are not divisible by $y^2$. Note that a reduced monomial of degree $d$ either ends in $x$, meaning it is equal to $ux$ for some other reduced monomial $u$ of degree $d-1$, or it ends in $xy$, meaning that it is equal to $vxy$ for some other reduced monomial $v$ of degree $d-2$. Thus $h(d) = h(d-1) + h(d-2)$, and since $h(0) = 1$ and $h(1) = 2$, we get
		\[
			H(t) = 1 + 2t + 3t^2 + 5t^3 + 8t^4 + \cdots
		\] 
		where the coefficients are the Fibonacci sequence. This can be written as
		\[
			H(t) = \frac{1 + t}{1 - t - t^2}.
		\]
	\end{enumerate}
\end{enumerate}
The computations done for the examples above are rather ad-hoc, but we give a general method of computing $H(t)$ below. Let $X$ be an finite well-ordered set of variables, and let $M$ be the set of leading monomials in a finite Gr\"obner basis. Denote the set of reduced monomials by $B_M$. Elements of $B_M$ are the monomials which are not divisible by any element of $M$, and $h(d)$ is the number of monomials in $B_M$ of degree $d$.

Let $D = \min_{m_i \in M} \deg(m_i) - 1$. We define a $|X|^D \times |X|^D$ matrix $A$ whose rows and columns are indexed by the monomials of degree $D$. Let $u,v$ be two such monomials. We set $A_{u,v} = 1$ if there exist $x_i, x_j \in X$ such that $x_i u = v x_j$ and $x_i u \in B_M$. This condition means that if $sv \in B_M$ for some monomial $s$, then $svx_j \in B_M$ as well. Otherwise we set $A_{u,v} = 0$.

Define $h(d,u)$ to be the number of monomials $s$ such that $su\in B_M$ and $\deg(su) = d$. Then the matrix $A$ satisfies
\[
	A\begin{bmatrix}
	\vdots\\
	h(d,u)\\
	\vdots
	\end{bmatrix}
	=\begin{bmatrix}
	\vdots\\
	h(d+1,u)\\
	\vdots
	\end{bmatrix}
\]
for $d \geq D$. If we define
\[
	H(t,u) = \sum_{d\geq D} h(d,u)t^d
\]
then we also get the identity
\[
	\begin{bmatrix}
	\vdots\\
	H(t,u)\\
	\vdots
	\end{bmatrix}=
	tA\begin{bmatrix}
	\vdots\\
	H(t,u)\\
	\vdots
	\end{bmatrix}+
	\begin{bmatrix}
	t^D\\
	\vdots\\
	t^D
	\end{bmatrix}.
\]
Then,
\begin{align}
	H(t) &= \sum_{d=0}^{D-1} h(d)t^d + \sum_u H(t,u) \nonumber\\
	&= \sum_{d=0}^{D-1} h(d)t^d + [1 \cdots 1]
	t^D (I - tA)^{-1} \begin{bmatrix}
		1\\
		\vdots\\
		1
	\end{bmatrix} \label{eq:Poincare-series-formula}
\end{align}
where the second term is the sum of the entries of $t^D(I-tA)^{-1}$. It is a fact that the asymptotic behavior of $h(d)$ is determined by the poles of $H(t)$ with minimal modulus; see \cite[Theorem 5.2.1]{wilf}. Observe that if $z \in \mathbb{C}$ is a pole of $H(t)$, then $\det(I-zA) = 0$. Since $z \neq 0$, this means $\det(I/z - A) = 0$ and $1/z$ is an eigenvalue of $A$. Sadly the converse is not true in general. Even so, we suspect that the asymptotic behavior of $h(d)$ is related to the eigenvalues of $A$ with maximal modulus, as the following examples suggest.
\begin{example}
	Let $X = \{x,y\}$ and $M = \{yx\}$. Then $D = 1$ and
	\[
		A = 
		\begin{blockarray}{ccc}
		x & y &\\
		\begin{block}{[cc]c}
		1 & 0 & x\\
		1 & 1 & y\\
		\end{block}
		\end{blockarray}
	\]
	which has $1$ as its only eigenvalue. Equation \eqref{eq:Poincare-series-formula} gives
	\[
		H(t) = 1 + \frac{t(2-t)}{(1-t)^2} = 1 + 2t + 3t^2 + \cdots
	\]
	and we see that $h(d)$ is linear. Also, $\lim_{d\to \infty} (h(d+1)/h(d)) = 1$.
	
	If we instead consider $M = \{yy\}$, then we get
	\[
		A = 
		\begin{blockarray}{ccc}
		x & y &\\
		\begin{block}{[cc]c}
		1 & 1 & x\\
		1 & 0 & y\\
		\end{block}
		\end{blockarray}
	\]
	which has $(1+\sqrt{5})/2$ as its eigenvalue with maximal modulus. As we saw before, $h(d)$ in this case is the Fibonacci sequence, and $\lim_{d\to \infty} (h(d+1) / h(d)) = (1+\sqrt{5})/2$.
\end{example}

We conclude this section by formulating the following conjecture, motivated by the examples that we have seen so far.
\begin{conj}
	For a finite $X$ and a finite Gr\"obner basis, $h(d)$ is either eventually polynomial, with degree at most $|X| - 1$, or it is asymptotically exponential. The former occurs when $1$ is the only pole of $H(t)$.
\end{conj}
For example, in the case $|X|=2$, we saw instances where $h(d)$ was eventually linear, constant, or zero, and an instance where it was asymptotically exponential.
%%% Local Variables:
%%% mode: latex
%%% TeX-master: t
%%% End:
