\section{Computations}\label{sec:examples}
Now we will put the machinery of \S\ref{sec:implementation} to use by computing Gr\"obner bases for the examples provided in the project description. Afterwards, in \S\ref{sec:assoc-graded}, we will quantify the size of $T(V)/\mathfrak{a}$ in each case, using Hilbert functions and Poincar\'e series.

\subsection{Examples from the assignment}\label{sec:assignment-examples}
For all of the following examples, our set of variables $X$ will be $\{x,y\}$, with the order $x < y$.
\subsection*{Part I}

The first part of the assignment was to determine whether a given polynomial $p$ is contained in the two-sided ideal generated by a given relation $f$, in five different cases. For this we will use the command
\begin{verbatim}
	$ grobner reduce 'f' -p 'p'
\end{verbatim}
from \S\ref{sec:implementation}, which first computes a Gr\"obner basis for $\mathfrak{a} = (f)$, and then prints the reduction outcome of $p$ using that basis. By Theorem~\ref{thm:grobner-thm}, we have $p\in\mathfrak{a}$ if and only if $p_\red=0$.

\begin{enumerate}
	\item If $f = x^2y$ and $p = x^4 y x - x^3 y x^2 + xyx^4$, then $p_\red = xyx^4$.
	\item If $f = yx - xy - 1$ and $p = yx^2 - 2x$, then $p_\red = x^2 y$.
    \item If $f = yx - 2xy$ and $p = yx^2 - xyx$, then $p_\red = 2x^2y$.
    \item If $f = y^2 - x^2$ and $p = yx^2 - x^2y$, then $p_\red = 0$.
    \item If $f = y^2 - x + 1$ and $p=y^3 - xy$, then $p_\red - y$.
\end{enumerate}

\subsection*{Parts II, III}

The second and third parts of the assignment were to use the program of \S\ref{sec:implementation} to (attempt to) compute Gr\"obner bases for the ideals $\mathfrak{a}$ generated by various sets of relations $\{f_i\}$. For this we will use the command
\begin{verbatim}
	$ grobner basis 'f1' 'f2' ...
\end{verbatim}
from \S\ref{sec:implementation}.

\begin{enumerate}
	\item For the set $\{y^2-x^2\}$, we obtain the Gr\"obner basis
	\begin{equation*}
		\{y^2-x^2,yx^2-x^2y\}.
	\end{equation*}
    By Corollary~\ref{cor:grobner}, a basis for the quotient algebra $T(x,y)/\mathfrak{a}$ is therefore given by
    \begin{equation}
      \label{eqn:basisone}
    \left\{x^i(yx)^jy^\ell\mid i,j\ge 0,\text{ }\ell\in\{0,1\}\right\}.
    \end{equation}
	\item For the set $\{y^2-x,yx-y\}$, we obtain the Gr\"obner basis
    \begin{equation*}
      \{y^2 - x,yx - y,xy - y,x^2 - x\}.
    \end{equation*}
    A basis for the quotient algebra is therefore
    \begin{equation*}
      \{1,x,y\}.
    \end{equation*}
    \item For the set $\{yxy-xyx\}$, the program does not terminate.
	\item For the set $\{yxy-x\}$, we obtain the Gr\"obner basis
	\begin{equation*}
		\{yxy-x,yx^2-x^2y\}.
	\end{equation*}
    A basis for the quotient algebra is therefore given by
    \begin{equation*}
      \left\{x^iy^jx^\ell\mid i,j\ge 0,\text{ }\ell\in\{0,1\}\right\}.
    \end{equation*}
	\item For the set $\{y^2-ax^2-bx\}$, where $a,b\in k$, we observe by testing various values of $a$ and $b$ that the Gr\"obner bases look like
      \begin{equation*}
\begin{cases}
\{y^2-ax^2-bx,yx^2-x^2y-\frac{b}{a}yx+\frac{b}{a}xy\}&\text{if }a\ne 0,\\
\{y^2-bx,yx-xy\}&\text{if }a=0\text{ and }b\ne 0,\\
\{y^2\}&\text{if }a=b=0.\\
\end{cases}
        \end{equation*}
      This is easily proven to hold in general. Thus, a basis for the quotient algebra is given by
      \begin{equation*}
\begin{cases}
\eqref{eqn:basisone}&\text{if }a\ne 0,\\
\left\{x^iy^\ell\mid i\ge 0,\text{ }\ell\in\{0,1\}\right\}&\text{if }a=0\text{ and }b\ne 0,\\
\left\{x^{i_1}yx^{i_2}y\cdots yx^{i_n}\mid n\ge 1,\text{ }i_1,i_n\ge 0,\text{ }i_2,\ldots,i_{n-1}\ge 1\right\}&\text{if }a=b=0.\\
\end{cases}
      \end{equation*}
	\item For the set $\{yxyx-xyxy\}$, the program does not terminate.
    \item The project description suggests that we consider when $X = \{x,y,z\}$ (with $x<y<z$) and we have a set of relations
\begin{equation*}
\{yx-\phi_1,zx-\phi_2,zy-\phi_3\},
\end{equation*}
where $\phi_1$ (resp.\ $\phi_2$, $\phi_3$) is lower than $yx$ (resp.\ $zx$, $zy$). In \S\ref{sec:applications}, we prove that these relations are consistent in the case where they describe the universal enveloping algebra of a $3$-dimensional Lie algebra. For instance, if
\begin{equation*}
\phi_1=xy+z,\quad\phi_2=xz-2x,\quad\phi_3=yz+2y,
\end{equation*}
then consistency is achieved; these relations correspond to the special linear Lie algebra $\mathfrak{sl}_2(k)$. A basis for the quotient algebra is given in Theorem~\ref{thm:ugbasis}.
\end{enumerate}

\subsection{The associated graded}\label{sec:assoc-graded}
Given an ideal $\mathfrak{a} \subset T(V)$, we may also ask about the ``size'' of $T(V)/\mathfrak{a}$. The dimension alone is too crude of a measurement, as $T(V)/\mathfrak{a}$ will almost always be infinite-dimensional. We can get a more refined sense of size by considering its associated graded algebra, which we construct now.

The tensor algebra $T(V)$ is naturally graded by degree. As $\mathfrak{a}$ need not be homogeneous, in general we do not get a grading of the quotient $T(V)/\mathfrak{a}$. However, if we take the grading filtration on $T(V)$ defined by
\[
	F_s T(V) = \bigoplus_{i=0}^s V^{\otimes i}
\]
with $F_{-1} = 0$, then the quotient map $T(V) \to T(V)/\mathfrak{a}$ induces a filtration on $T(V)/\mathfrak{a}$:
\[
	F_s(T(V)/\mathfrak{a}) = \img (F_s T(V)).
\]
This filtration respects multiplication in the sense that $F_s F_t \subset F_{s+t}$ for all $s,t \geq 0$. Hence if we take
\[
	\gr_s(T(V)/\mathfrak{a}) \coloneqq F_s/F_{s-1},
\]
then $\gr (T(V)/\mathfrak{a})$ is an algebra: the \emph{associated graded algebra} for the filtration. Using this, we can give a more nuanced description of the size of $T(V)/\mathfrak{a}$.
\begin{defn}
	Let $A$ be a graded vector space. Its \emph{Hilbert function} is defined to be
	\[
		h(d) \coloneqq \dim_k (A)_d
	\]
	and its \emph{Poincar\'e series} is defined to be
	\[
		H(t) \coloneqq \sum_{d\geq 0} h(d) t^d.
	\]
\end{defn}
For $\gr (T(V) / \mathfrak{a})$, we already have the tools to compute $h$ (and thus $H$ as well):
\begin{lem}
	$h(d)$ is the number of reduced monomials of degree $d$.
\end{lem}
\begin{proof}
	This is an immediate consequence of Corollary \ref{cor:grobner}.
\end{proof}
Now we compute the Hilbert function and Poincar\'e series for each of the examples in Parts II, III of \S\ref{sec:assignment-examples} where we successfully obtained a finite Gr\"obner basis. We will write out only the Poincar\'e series---from them, the Hilbert functions are easily seen.
\begin{enumerate}
	\item Of the monomials $\left\{x^i(yx)^jy^\ell\mid i,j\ge 0,\text{ }\ell\in\{0,1\}\right\}$, exactly $d+1$ of them have degree $d$. The Poincar\'e series is
	\begin{equation}
		H(t) = 1 + 2t + 3t^2 + 4t^3 + \cdots.\label{eq:poincare-y2-x2}
	\end{equation}
	Note that $H - tH = 1 + t + t^2 + t^3 + \cdots$ is the series expansion for $1/(1-t)$, so we can write $H$ in closed form as (the series expansion of)
	\[
		H(t) = \frac{1}{(1-t)^2}.
	\]
	\item The only reduced monomials are $\{1,x,y\}$ so the Poincar\'e series is
	\[
		H(t) = 1 + 2t.
	\]
	\setcounter{enumi}{3}
	\item There are $2d$ monomials of the form $\left\{x^iy^jx^\ell\mid i,j\ge 0,\text{ }\ell\in\{0,1\}\right\}$ which have degree $d$, for $d \geq 1$. Hence
	\[
		H(t) = 1 + 2t + 4t^2 + 6t^3 + \cdots
	\]
	which can be written in closed form as
	\[
		H(t) = \frac{1+t^2}{(1-t)^2}.
	\]
	\item We have three sub-cases in this scenario:
	\begin{enumerate}
		\item If $a\neq 0$, then $H(t)$ is \eqref{eq:poincare-y2-x2}.
		\item If $a = 0$ and $b\neq 0$ then there are $2$ reduced monomials of degree $d$ for $d \geq 1$, so
		\[
			H(t) = 1 + 2t + 2t^2 + 2t^3 + \cdots
		\]
		which can be written as
		\[
			H(t) = \frac{1+t}{1-t}.
		\]
		\item If $a = b = 0$, then the reduced monomials are monomials which are not divisible by $y^2$. Note that a reduced monomial of degree $d$ either ends in $x$, meaning it is equal to $ux$ for some other reduced monomial $u$ of degree $d-1$, or it ends in $xy$, meaning that it is equal to $vxy$ for some other reduced monomial $v$ of degree $d-2$. Thus $h(d) = h(d-1) + h(d-2)$, and since $h(0) = 1$ and $h(1) = 2$, we get
		\[
			H(t) = 1 + 2t + 3t^2 + 5t^3 + 8t^4 + \cdots
		\] 
		where the coefficients are the Fibonacci sequence. This can be written as
		\[
			H(t) = \frac{1 + t}{1 - t - t^2}.
		\]
	\end{enumerate}
\end{enumerate}
Note that the last sub-case of the final example differs from all the other examples: $h(d)$ is asymptotically exponential rather than (eventually) linear, and $H(t)$ has a denominator of $1 - t - t^2$ rather than $(1-t)^\delta$ for some $\delta$.

We conclude this section with some additional observations and conjectures regarding reduced monomial bases and the eventual behavior of $h(d)$. To make sense of $h(d)$, we continue assuming that $X$ is finite, although we no longer assume $|X| = 2$. In what follows, we use some nomenclature from computability theory. A \emph{language} is a set of strings (monomials) formed out of letters (variables) from $X$, which we call our \emph{alphabet}. As an example, we have the language $M = \{m_i\}$ consisting of the leading monomials from our Gr\"obner basis. The linear basis of reduced monomials is then precisely the language
\[
	B_M = \{u \mid \text{for all }m_i \in M\text{, }u \text{ does not contain }m_i\text{ as a substring}\}.
\]  
In the below, we assume that $M$ is finite. We examine the structure of languages $B_M$ with this assumption.
\begin{prop}\label{prop:BM-characterization}
	A language $B$ is of the form $B_M$ for some finite set of strings $M$ if and only if $B$ satisfies the following:
	\begin{enumerate}
		\item\label{item:substringcondition} If $b \in B$ and $b'$ is a substring of $b$, then $b' \in B$.
		\item\label{item:gluingcondition} There exists a non-negative integer $D$ such that if $b_1 u$ and $u b_2$ are elements of $B$ and $\ell(u) = D$, then $b_1 u b_2 \in B$.
	\end{enumerate}
\end{prop}
\begin{proof}
	Given a language of the form $B_M$ where $M=\{m_i\}$, we can take $D = \max_i(\ell(m_i))-1$ and it is straightforward to check that the conditions are satisfied.
	
	For the other direction, let $M$ be the set of all (non-empty) strings of length at most $D + 1$ which are \emph{not} elements of $B$. The two conditions entail that, in order to test if a given string $b$ is an element of $B$, it suffices to check all of its substrings of length at most $D$. Hence $B = B_M$ as desired.
\end{proof}
This result is readily used to show that certain languages are \emph{not} of the form $B_M$ for any $M$, and thus do not define a set of reduced monomials (for any Gr\"obner basis). In this sense, it is similar to the ``pumping lemmas'' in computability theory.

Note that all of the sets of reduced monomials from \S\ref{sec:assignment-examples} can be written as regular expressions (regex). Take \eqref{eqn:basisone} for instance, where we have 
\[
	\left\{x^i(yx)^jy^\ell\mid i,j\ge 0,\text{ }\ell\in\{0,1\}\right\} = \{ x^* (yx)^* \} \cup \{ x^* (yx)^* y \}.
\]
This is true in general, as the following proposition shows.
\begin{prop}
	Any language of the form $B_M$ is regular, meaning that $B_M$ can be characterized by a regular expression.
\end{prop}
\begin{proof}
	For a fixed string $m$, consider the language $R_{m}$ consisting of all strings which contain $m$ as a substring. This is a regular language as it can be expressed by the regex $X^* m X^*$, where $X$ is the alphabet. Note that
	\[
	B_M = \bigcap_{m \in M} R_{m}^C
	\]
	where $R_m^C$ denotes the complement of $R_m$. This is a finite intersection, and it is a well-known fact that regular languages are closed under set-theoretic Boolean operations. Therefore $B_M$ is also regular.
\end{proof}

We finish this section with a few questions. If $X = \{x_1,\ldots,x_N\}$, note that for $0 \leq i \leq N$, the language
\[
	B_i = \{ x_1^* x_2^* \cdots x_i^* \}
\]
satisfies the conditions of Proposition \ref{prop:BM-characterization} and thus is a family of reduced monomials for some finite Gr\"obner basis. The Hilbert function $h(n)$ for these reduced monomials is eventually polynomial, with degree $i - 1$.

The degree is maximal when we have $B_N$. If we attempt to introduce more Kleene stars (the asterisks) to the regex for $B_N$, it seems that rather than becoming a polynomial of higher degree, the Hilbert function explodes exponentially. As an illustrative special case, suppose our set $B$ of reduced monomials contains the set
\[
	\{x_i^* x_j^* x_i^*\}
\]
where $i\neq j$.
Let $D$ be as in Proposition \ref{prop:BM-characterization} applied to $B$. Using conditions \eqref{item:substringcondition} and \eqref{item:gluingcondition}, we deduce that for any $\alpha\colon \mathbb{N} \to \{i,j\}$, the string
\[
	x_i^D x_{\alpha(0)} x_i^D x_{\alpha(1)} x_i^D \cdots x_i^D x_{\alpha(n)} x_i^D 
\]
is always in $B$. But this results in exponential growth of $h(d)$. Also, if we look back at the example where $M = \{y^2\}$, the reduced monomials are characterized by the regex $x^* (yxx^*)^*$. The Kleene star applied to $yxx^*$ again results in exponential growth of $h(d)$ for similar reasons.

From what we've seen so far, we formulate the following conjecture.
\begin{conj}
	For a finite $X$ and a finite Gr\"obner basis, $h(d)$ is either eventually polynomial (with degree at most $N-1$) or it is asymptotically exponential.
\end{conj}
For example, in the case $N = 2$, we saw instances where $h(d)$ was eventually linear (or constant, or zero), and an instance where it was asymptotically exponential.
Ideally, one would like to discern the behavior of $h(d)$ from the monomials $m_i$ directly, but it is unclear to us how one would do so.

%%% Local Variables:
%%% mode: latex
%%% TeX-master: t
%%% End:
