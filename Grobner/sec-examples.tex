\section{Computations}\label{sec:examples}
\textcolor{red}{TODO: move (what is appropriate) about assoc. graded into this section, and finish this section}

In this section, we compute Gr\"obner bases for the examples given in the project description. Throughout, our set of variables $X$ will be $\{x,y\}$, with the order $x < y$.

\subsection{Examples from the assignment}
\subsection*{Part I}

The first part of the assignment was to determine whether a given polynomial $p$ is contained in the two-sided ideal generated by a given relation $f$, in five different cases. For this we will use the command
\begin{verbatim}
	$ grobner reduce 'f' -p 'p'
\end{verbatim}
from \S\ref{sec:implementation}, which first computes a Gr\"obner basis for $\mathfrak{a} = (f)$, and then prints the reduction outcome of $p$ using that basis.

\begin{enumerate}
	\item If $f = x^2y$ and $p = x^4 y x - x^3 y x^2 + xyx^4$, then $p_\red = xyx^4$.
	\item If $f = yx - xy - 1$ and $ p = yx^2 - 2x$, then $p_\red = x^2 y$.
\end{enumerate}

\subsection*{Part II}

The second part of the assignment was to use the program of \S\ref{sec:implementation} to (attempt to) compute Gr\"obner bases for the following sets of relations:

\begin{enumerate}
	\item $\{y^2-x^2\equiv 0\}$. The program returns the Gr\"obner basis
	\begin{equation*}
		\{y^2-x^2\equiv 0,yx^2-x^2y\equiv 0\}.
	\end{equation*}
	\item $\{y^2-x\equiv 0,yx-y\equiv 0\}$. The program returns the Gr\"obner basis
\end{enumerate}

\subsection*{Part III}

\begin{enumerate}
	\item $\{yxy-x\equiv 0\}$. The program returns the Gr\"obner basis
	\begin{equation*}
		\{yxy-x\equiv 0,yx^2-x^2y\equiv 0\}
	\end{equation*}
	\item $\{y^2-ax^2-bx\equiv 0\}$, where $a,b\in k$. We observe using our program that
	\item $\{yxyx-xyxy\equiv 0\}$. The program does not terminate.
\end{enumerate}

\subsection{The associated graded}\label{sec:assoc-graded}
Given an ideal $\mathfrak{a} \subset T(V)$, we may also ask about the ``size'' of $T(V)/\mathfrak{a}$. The dimension alone is too crude of a measurement, as $T(V)/\mathfrak{a}$ will almost always be infinite-dimensional. We can get a more refined sense of size by considering its associated graded algebra, which we construct now.

The tensor algebra $T(V)$ is naturally graded by degree. As $\mathfrak{a}$ need not be homogeneous, in general we do not get a grading of the quotient $T(V)/\mathfrak{a}$. However, if we take the grading filtration on $T(V)$ defined by
\[
	F_s T(V) = \bigoplus_{i=0}^s V^{\otimes s}
\]
then the quotient map $T(V) \to T(V)/\mathfrak{a}$ induces a filtration on $T(V)/\mathfrak{a}$:
\[
	F_s(T(V)/\mathfrak{a}) = \img (F_s T(V)).
\]
This filtration respects multiplication in the sense that $F_s F_t \subset F_{s+t}$ for all $s,t \geq 0$.

%%% Local Variables:
%%% mode: latex
%%% TeX-master: t
%%% End:
