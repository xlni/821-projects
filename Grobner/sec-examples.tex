\section{Computations}\label{sec:examples}
\textcolor{red}{TODO: move (what is appropriate) about assoc. graded into this section, and finish this section}

In this section, we compute Gr\"obner bases for the examples given in the project description. Throughout, our set of variables $X$ will be $\{x,y\}$, with the order $x < y$.

\subsection{Examples from the assignment}
\subsection*{Part I}

The first part of the assignment was to determine whether a given polynomial $p$ is contained in the two-sided ideal generated by a given relation $f$, in five different cases. For this we will use the command
\begin{verbatim}
	$ grobner reduce 'f' -p 'p'
\end{verbatim}
from \S\ref{sec:implementation}, which first computes a Gr\"obner basis for $\mathfrak{a} = (f)$, and then prints the reduction outcome of $p$ using that basis. By Theorem~\ref{thm:grobner-thm}, we have $p\in\mathfrak{a}$ if and only if $p_\red=0$.

\begin{enumerate}
	\item If $f = x^2y$ and $p = x^4 y x - x^3 y x^2 + xyx^4$, then $p_\red = xyx^4$.
	\item If $f = yx - xy - 1$ and $p = yx^2 - 2x$, then $p_\red = x^2 y$.
    \item If $f = yx - 2xy$ and $p = yx^2 - xyx$, then $p_\red = 2x^2y$.
    \item If $f = y^2 - x^2$ and $p = yx^2 - x^2y$, then $p_\red = 0$.
    \item If $f = y^2 - x + 1$ and $p=y^3 - xy$, then $p_\red - y$.
\end{enumerate}

\subsection*{Part II}

The second part of the assignment was to use the program of \S\ref{sec:implementation} to (attempt to) compute Gr\"obner bases for the ideals $\mathfrak{a}$ generated by three different sets of relations $\{f_i\}$. For this we will use the command
\begin{verbatim}
	$ grobner basis 'f1' 'f2' ...
\end{verbatim}
from \S\ref{sec:implementation}.

\begin{enumerate}
	\item For the set $\{y^2-x^2\}$, we obtain the Gr\"obner basis
	\begin{equation*}
		\{y^2-x^2,yx^2-x^2y\}.
	\end{equation*}
    By Corollary~\ref{cor:grobner}, a basis for the quotient algebra $T(x,y)/\mathfrak{a}$ is therefore given by
    \begin{equation}
      \label{eqn:basisone}
    \left\{x^i(yx)^jy^\ell\mid i,j\ge 0,\text{ }\ell\in\{0,1\}\right\}.
    \end{equation}
	\item For the set $\{y^2-x,yx-y\}$, we obtain the Gr\"obner basis
    \begin{equation*}
      \{y^2 - x,yx - y,xy - y,x^2 - x\}.
    \end{equation*}
    A basis for the quotient algebra is therefore
    \begin{equation*}
      \{1,x,y\}.
    \end{equation*}
    \item For the set $\{yxy-xyx\}$, the program does not terminate.
\end{enumerate}

\subsection*{Part III}

The third part of the assignment was a continuation of the second; we therefore preserve the setup of Part II.

\begin{enumerate}
	\item For the set $\{yxy-x\}$, we obtain the Gr\"obner basis
	\begin{equation*}
		\{yxy-x,yx^2-x^2y\}.
	\end{equation*}
    A basis for the quotient algebra is therefore given by
    \begin{equation*}
      \left\{x^iy^jx^\ell\mid i,j\ge 0,\text{ }\ell\in\{0,1\}\right\}.
    \end{equation*}
	\item For the set $\{y^2-ax^2-bx\}$, where $a,b\in k$, we observe by testing various values of $a$ and $b$ that the Gr\"obner bases look like
      \begin{equation*}
\begin{cases}
\{y^2-ax^2-bx,yx^2-x^2y-\frac{b}{a}yx+\frac{b}{a}xy\}&\text{if }a\ne 0,\\
\{y^2-bx,yx-xy\}&\text{if }a=0\text{ and }b\ne 0,\\
\{y^2\}&\text{if }a=b=0.\\
\end{cases}
        \end{equation*}
      This is easily proven to hold in general. Thus, a basis for the quotient algebra is given by
      \begin{equation*}
\begin{cases}
\eqref{eqn:basisone}&\text{if }a\ne 0,\\
\left\{x^iy^\ell\mid i\ge 0,\text{ }\ell\in\{0,1\}\right\}&\text{if }a=0\text{ and }b\ne 0,\\
\left\{x^{i_1}yx^{i_2}y\cdots yx^{i_n}\mid n\ge 1,\text{ }i_1,i_n\ge 0,\text{ }i_2,\ldots,i_{n-1}\ge 1\right\}&\text{if }a=b=0.\\
\end{cases}
      \end{equation*}
	\item For the set $\{yxyx-xyxy\}$, the program does not terminate.
    \item The project description suggests that we consider a set of relations of the form
\begin{equation*}
\{yx-\phi_1,zx-\phi_2,zy-\phi_3\},
\end{equation*}
where $\phi_1$ (resp.\ $\phi_2$, $\phi_3$) is lower than $yx$ (resp.\ $zx$, $zy$), and where our set of variables $X$ is now $\{x,y,z\}$, with the order $x<y<z$. In \S\ref{sec:applications}, we prove that these relations are consistent in the case where they describe the universal enveloping algebra of a $3$-dimensional Lie algebra. For instance, if
\begin{equation*}
\phi_1=xy+z,\quad\phi_2=xz-2x,\quad\phi_3=yz+2y,
\end{equation*}
then consistency is achieved; these relations correspond to the special linear Lie algebra $\mathfrak{sl}_2(k)$. A basis for the quotient algebra is given in Theorem~\ref{thm:ugbasis}.
\end{enumerate}

\subsection{The associated graded}\label{sec:assoc-graded}
Given an ideal $\mathfrak{a} \subset T(V)$, we may also ask about the ``size'' of $T(V)/\mathfrak{a}$. The dimension alone is too crude of a measurement, as $T(V)/\mathfrak{a}$ will almost always be infinite-dimensional. We can get a more refined sense of size by considering its associated graded algebra, which we construct now.

The tensor algebra $T(V)$ is naturally graded by degree. As $\mathfrak{a}$ need not be homogeneous, in general we do not get a grading of the quotient $T(V)/\mathfrak{a}$. However, if we take the grading filtration on $T(V)$ defined by
\[
	F_s T(V) = \bigoplus_{i=0}^s V^{\otimes s}
\]
then the quotient map $T(V) \to T(V)/\mathfrak{a}$ induces a filtration on $T(V)/\mathfrak{a}$:
\[
	F_s(T(V)/\mathfrak{a}) = \img (F_s T(V)).
\]
This filtration respects multiplication in the sense that $F_s F_t \subset F_{s+t}$ for all $s,t \geq 0$.

%%% Local Variables:
%%% mode: latex
%%% TeX-master: t
%%% End:
