\usepackage[latin1]{inputenc}
\usepackage{mathtools} 
\usepackage{amssymb} 
\usepackage{mathrsfs} % gives mathscr font 
\usepackage{graphicx}
%\usepackage[headsep=0.15in, left=0.5in, right=0.5in, top=0.6in, bottom=0.5in]{geometry} 
%\usepackage[textlf,mathlf]{MinionPro} 
\usepackage{fancyhdr}
%\usepackage{xypic}
\usepackage{tikz-cd} 
\usepackage{todonotes}
\usepackage{verbatim}
\usepackage{filecontents}
\usepackage[colorlinks=true]{hyperref}
\usepackage{tabu}
\usepackage{multicol}
\usepackage{booktabs}
\usepackage{centernot}
\usepackage{bbm} % for blackboard bold 1 indicator functions
\usepackage{multirow}
\usepackage{enumerate}
\usepackage{stmaryrd} % for \mapsfrom

\definecolor{zaffre}{rgb}{0.0, 0.08, 0.66}
\hypersetup{colorlinks=true,allcolors=zaffre}

%Environments
\numberwithin{equation}{section} 
\numberwithin{figure}{section}
\numberwithin{table}{section}
\theoremstyle{definition} 
\theoremstyle{plain} 
\newtheorem{thm}{\protect\theoremname}[section]
\theoremstyle{definition}
\newtheorem{example}[thm]{\protect\examplename}
\theoremstyle{remark} 
\newtheorem{rem}[thm]{\protect\remarkname}
\theoremstyle{plain}
\newtheorem{cor}[thm]{\protect\corollaryname}
\theoremstyle{definition} 
\newtheorem{defn}[thm]{\protect\definitionname}
\theoremstyle{plain} 
\newtheorem{prop}[thm]{\protect\propositionname}
\theoremstyle{plain} 
\newtheorem{lem}[thm]{\protect\lemmaname}
\newtheorem{conj}[thm]{Conjecture}
\theoremstyle{definition}
\newtheorem{notation}[thm]{Notation}

\providecommand{\definitionname}{Definition}
\providecommand{\examplename}{Example} 
\providecommand{\lemmaname}{Lemma}
\providecommand{\propositionname}{Proposition}
\providecommand{\remarkname}{Remark} 
\providecommand{\corollaryname}{Corollary}
\providecommand{\theoremname}{Theorem}

\setcounter{tocdepth}{1}

%macros, etc.
\newcommand{\legendre}[2]{\genfrac{(}{)}{}{}{#1}{#2}}
\renewcommand{\arraystretch}{1.2} 

\newcommand{\A}{\mathbb{A}}
\renewcommand{\P}{\mathbb{P}} 
\newcommand{\F}{\mathbb{F}}
\newcommand{\Z}{\mathbb{Z}}
\newcommand{\Q}{\mathbb{Q}}
\newcommand{\R}{\mathbb{R}}
\newcommand{\C}{\mathbb{C}}
\newcommand{\GL}{\operatorname{GL}}
\newcommand{\Span}{\operatorname{Span}}
\newcommand{\trsp}[1]{{{#1}^{\top}\!\!}}
\newcommand{\ndiv}{\centernot|}
\newcommand{\IND}{\mathbbm{1}}
\DeclareMathOperator{\img}{im}
\newcommand{\red}{\mathrm{red}}